\documentclass{article}
\usepackage[english]{babel}
\usepackage{geometry,amssymb,enumerate,theorem,amsmath}
\geometry{a4paper, bottom=20mm, top=20mm}

%%%%%%%%%% Start TeXmacs macros
\newcommand{\TeXmacs}{T\kern-.1667em\lower.5ex\hbox{E}\kern-.125emX\kern-.1em\lower.5ex\hbox{\textsc{m\kern-.05ema\kern-.125emc\kern-.05ems}}}
\newcommand{\assign}{:=}
\newcommand{\mathd}{\mathrm{d}}
\newcommand{\tmtextbf}[1]{{\bfseries{#1}}}
\newenvironment{enumeratealpha}{\begin{enumerate}[a{\textup{)}}] }{\end{enumerate}}
{\theorembodyfont{\rmfamily\small}\newtheorem{exercise}{Exercise}}
%%%%%%%%%% End TeXmacs macros

\newcommand{\renderexercise}[2]{{\paddednormal{0.5fn}{0.5fn}{{\exercisename{#1{\exercisesep}}}#2}}}

\begin{document}

\setlength{\parindent}{0cm}{\small{\tmtextbf{Institute for Applied Mathematics
-- SS2020}}}\hspace{4cm}\hbox{}

{\small{Massimiliano Gubinelli}}

\


{\center{{\LARGE{\tmtextbf{V4F1 Stochastic Analysis -- Problem Sheet 8}}}

\ }}


{\scriptsize{Version 1, 2020.06.10. Tutorial classes: Mon June 22nd 16--18 \
(Zoom) Min Liu \textbar  Wed June 24th 16--18 (Zoom) Daria Frolova.

Solutions in groups of 2 (at most). To be handled in {\LaTeX} or {\TeXmacs}
format via eCampus not later than 8pm Thursday June 18th. Use this sheet for
your solutions and write them under the corresponding exercise. Fill out your
names below.

\ }}

\tmtextbf{Names: XXXXXXXXXXXX/YYYYYYYYYYYYYY}

\hrulefill

\begin{exercise}
  [Pts 2+2+2+2] Assume that $\Omega = C (R_{\geqslant 0} ; \mathbb{R}^d)$,
  $\mathbb{P}$ is the $d$--dimensional Wiener measure and that $X$ is the
  canonical process on $\Omega$ and that the filtration
  $\mathcal{F}_{\bullet}$ is generated by $X$. Consider a predictable \
  $\mathbb{R}^d$-valued drift $b$ given by a function $b :
  \mathbb{R}_{\geqslant 0} \times \Omega \rightarrow \mathbb{R}^d$. By tilting
  $\mathbb{P}$ via $Z =\mathcal{E} \left( \int_0^{\cdot} b (X) \mathd X
  \right)$ we obtain that, under the tilted measure $\mathbb{P}^b$ the process
  $X$ is a solution of the SDE
  \[ \mathd X_t = b_t (X) + \mathd W_t, \qquad t \geqslant 0 \]
  where $W$ is a $\mathbb{P}^b$--Brownian motion.
  \begin{enumeratealpha}
    \item Prove that if
    \[ | b_t (x) | \leqslant C (1 + | x_t |), \qquad t \geqslant 0, x \in
       \Omega, \]
    then Novikov's condition holds conditionally on $\mathcal{F}_s$ for
    intervals $[s, t]$ such that $| t - s |$ is small enough, i.e.
    \[ \mathbb{E} \left[ \exp \left( \frac{1}{2} \int_s^t | b_u (X) |^2 \mathd
       u \right) |\mathcal{F}_s \right] < + \infty . \]
    \item Deduce that $Z$ is a martingale. [Hint: prove that $\mathbb{E} [Z_t
    |\mathcal{F}_s] = Z_s$ for small time intervals $[s, t]$ and the
    conclude].
    
    \item Prove that
    \[ \mathbb{P} (\| X \|_{[0, t]} > r) \leqslant 2 d e^{- r^2 / 2 d t}
       \qquad t \geqslant 0, r \geqslant 0. \]
    where $\| X \|_{[0, t]}$ denotes the supremum wrt. the Euclidean norm of
    $(X_s)_{s \in [0, t]}$.
    
    [Hint: use Doob's inequality for the submartingale $e^{\lambda X^i_t}$ and
    optimize over $\lambda > 0$]
    
    \item Prove the same result as in (a) under the more general assumption
    that $b$ is a previsible drift such that
    \[ | b_t (x) | \leqslant C (1 + \| x \|_{\infty, [0, t]}), \qquad t
       \geqslant 0, x \in \Omega \]
    where $C < + \infty$.
  \end{enumeratealpha}
\end{exercise}

\hrulefill

\begin{exercise}
  [Pts 2+2+2] Consider the one dimensional SDE
  \[ \mathd X_t = - X_t^3 \mathd t + \mathd B_t, \qquad X_0 = x, \]
  where $B$ is a standard Brownian motion.
  \begin{enumeratealpha}
    \item Let $f (t, x) = (1 + | x |^2)$ and $T_L = \inf \{ t \geqslant 0 : |
    X_t | > L \}$. Use Ito formula to show that there exists a constant
    $\lambda$ such that the process $Z_t \assign e^{- \lambda (t \wedge T_L)}
    f (X_{t \wedge T_L})$ is a supermartingale.
    
    \item Deduce that $\mathbb{P} (T_L \leqslant t) \rightarrow 0$ as $L
    \rightarrow \infty$.
    
    \item Conclude that solutions of the SDE cannot explode (that is $\zeta
    \assign \sup_L T_L = \infty$ a.s.).
  \end{enumeratealpha}
\end{exercise}

\hrulefill

\begin{exercise}
  [Pts 2+2+2] If $c (t) = (x (t), y (t))$ is a smooth curve in $\mathbb{R}^2$
  with c(0) = 0,
  \[ A_t = \int_0^t (x (s) y' (s) - y (s) x' (s)) \mathd s \]
  describes the area that is covered by the secant from the origin to $c (s)$
  in the interval $[0, t]$. Analogously, for a two-dimensional Brownian motion
  $B_t = (X_t, Y_t)$ with $B_0 = 0$, one defines the L{\'e}vy Area
  \[ A_t = \int_0^t (X_s \mathd Y_s - Y_s \mathd X_s) . \]
  \begin{enumeratealpha}
    \item Let $\alpha (t)$, $\beta (t)$ be $C^1$-functions, $p \in
    \mathbb{R}$, and
    \[ V_t = ipA_t - \frac{\alpha (t)}{2} (X^2_t + Y^2_t) + \beta (t) . \]
    Use It{\^o} formula to show that $e^{V_t}$ is a local martingale provided
    $\alpha' (t) = \alpha (t)^2 - p^2$ and $\beta' (t) = \alpha (t)$
    
    \item Let $t_0 \geqslant 0$. Solutions to the equations for $\alpha,
    \beta$ with $\alpha (t_0) = \beta (t_0) = 0$ are
    \[ \alpha (t) = p \tanh (p (t_0 - t)), \qquad \beta (t) = - \log \cosh (p
       (t_0 - t)) . \]
    Conclude that
    \[ \mathbb{E} \left[ e^{ipA_{t_0}} \right] = (\cosh (p t_0))^{- 1} . \]
    \item Show that the distribution of $A_t$ is absolutely continuous with
    respect to the Lebesgue measure with density
    \[ f_{A_t} (x) = (2 t \cosh (\pi x / 2 t))^{- 1}, \qquad x \in \mathbb{R}.
    \]
  \end{enumeratealpha}
\end{exercise}

\hrulefill

\

\end{document}
