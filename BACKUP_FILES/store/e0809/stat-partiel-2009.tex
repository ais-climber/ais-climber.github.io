%!TEX encoding =  IsoLatin
\documentclass[a4paper,10pt]{article}
\usepackage[noheadfoot,bottom=1cm,top=2cm,left=3cm,right=3cm]{geometry}
\usepackage[french]{babel}
\usepackage[latin1]{inputenc}
%\usepackage[parfill]{parskip}    % Activate to begin paragraphs with an empty line rather than an 
\usepackage{amsfonts,amsmath}

\newcommand{\ds}{\displaystyle}
\newcommand{\vs}{\vspace{0.25cm}}
%\renewcommand{\vss}{\vspace{0.25cm}}
\newcommand{\E}{\mathbb{E}}
\newcommand{\V}{\mathbb{V}}
\newcommand{\RR}{\mathbb{R}}
\renewcommand{\P}{\mathbb{P}}

\newcommand{\Pto}{\stackrel{P}{\longrightarrow}}
\newcommand{\Lto}{\stackrel{\cal L}{\longrightarrow}}
\newcommand{\Leq}{\stackrel{\cal L}{=}}
\newcommand{\PSto}{\stackrel{p.s.}{\longrightarrow}}
%\makeatletter
%\let\accent@spacefactor\relax
%\makeatother

\newcounter{exo}
\setcounter{exo}{1}

\newcommand{\exercise}{{\vspace{0.4cm}\noindent\textbf{Exercice \theexo .\hspace{1em}}\addtocounter{exo}{1}}}
\pagestyle{empty}

\begin{document}

 \renewcommand{\labelenumi}{\alph{enumi})}
 
\hrule
%\parindent 0cm
\vspace{0.2cm}
\noindent Universit� Paris Dauphine \hfill Ann\'ee 2009

\noindent D�partement MIDO \hfill
DU 2 - Statistiques 

\vspace{0.2cm}\hrule

\vspace{0.2cm}

\begin{center}
{\Large \bf \vs Partiel 2009}
\end{center}

\vspace{0.2cm}

{\small
\noindent [Dur�e deux heures. Aucun document n'est autoris�. Les  exercices sont ind�pendants. Seule les r�ponses soigneusement justifi�es seront prises en compte. ]
}

\vspace{0.2cm}


\exercise  Soit $(X,Y)$ un couple al�atoire � valeurs dans $\RR^2$ admettant une densit� $$
f_{(X,Y)}(x,y)=\begin{cases} C & \text{si $\max(|x|,|y|)\le 2$}\\ 0 & \text{sinon}\end{cases}
$$ 

\begin{enumerate}
\item
D�terminer $C$.
\item $X$ et $Y$ sont-elles ind�pendantes?
\item Calculer la loi de la v.a. $X+Y$.
\end{enumerate}



\exercise Soit $(X,Y)$ le vecteur gaussien centr� de matrice de covariance 
$$
\left(
\begin{array}{cc}1 & 1 \\ 1 & 4\end{array}\right)
$$
Soit $Z=Y-\alpha X$.

\begin{enumerate}
%\item Calculer $\P(2 X \ge 2 + 3Y)$ en termes de la fonction de repartition $\Phi$ de la loi gaussienne standard (centr�e et reduite).
\item Quelle est la loi de $Z$? Pr\'eciser ses param\`etres.
\item D\'eterminer $\alpha$ tel que $X$ et $Z$ soient ind\'ependantes.
\item Calculer le coefficient de corr\'elation entre $X$ et $Y$.
\item Calculer le coefficient de corr\'elation entre $X^2$ et $Y^2$.

%\item Montrer qu'il n'existent pas des r�els $\beta,\gamma,\rho$ tels que $\beta X + \gamma Y = \rho$.
%\item Calculer $\E[e^{Z}]$ pour $\alpha=2$.
\end{enumerate}

\exercise Soient $X \sim \mathcal{N}(0,1)$, et $K$ une v.a. discrete telle que 
$$\P(K=-1)=\P(K=1)=1/2$$
et $K$ est ind\'ependante de $X$.  On consi\`ere $Y=K X$. 

\begin{enumerate}
\item Calculer $\E(Y), \text{Var}(Y)$ et $\text{Cov}(X, Y)$.
\item Calculer la fonction de r�partition de $Y$ et en d\'eduire que $Y \sim \mathcal{N}(0,1)$.
\item Montrer (par un argument simple) que le vecteur $(X,Y)$ n'est pas gaussien.
\end{enumerate}



\exercise Soit $U_1, ..., U_n$ une suite i.i.d. $\sim \mathcal{U}[0,1]$. On pose $X_n = \min_{1\le k \le n} U_k$ et $Y_n = n X_n$.
\begin{enumerate}
\item Calculer la fonction de r\'epartition de $X_n$ et sa densit\'e. Identifier la loi de $X_n$.
\item Donner la fonction de r\'epartition de $Y_n$.
\item Montrer que $(Y_n)_{n \ge 1}$ converge en loi en pr\'ecisant cette loi limite.

\end{enumerate}


\exercise Soit $(X_n)_{n \ge 1}$ une suite de v.a. telle que $X_n \sim \chi^2_n$ (une loi Khi-Deux de $n$ de degr\'es de libert\'e).

\medskip

\begin{enumerate}
\item Rappeler la d\'efinition d'une variable al\'eatoire $\sim \chi^2_m, m \in \mathbb N^*$. 
\item Pour $n \ge 1$, calculer $\E(X_n)$ et $\textrm{Var}(X_n)$.
\item Montrer que $X_n/n$ converge presque s�rement en pr\'ecisant le th\'eor\`eme utilis\'e et identifier la limite.
\item Montrer que $X_n/\sqrt{n}-\sqrt{n}$ converge en loi en pr\'ecisant le th\'eor\`eme utilis\'e et identifier la limite.
\end{enumerate}

\end{document}


\exercise  Soit $(X,Y)$ un couple al�atoire � valeurs dans $\RR^2$ admettant une densit� $$
f_{(X,Y)}(x,y)=\begin{cases} C & \text{si $|x|^2+|y|^2 \le 1$}\\ 0 & \text{sinon}\end{cases}
$$ 

\begin{enumerate}
\item
D�terminer $C$.
\item Montrer que $X,Y$ ne sont pas ind�pendantes.
\item Calculer $\P(X+Y\le 0)$ et $\P(X\ge 0, Y\ge 0)$.
\item Calculer $\text{Var}(X|Y) = \E[(X-\E[X|Y])^2|Y]$. 
\item Soient $(R,\Theta)$ tels que $R\ge 0$, $\Theta\in[0,2\pi)$ et  $X=R \sin(\Theta)$, $Y= R \cos(\Theta)$. Montrer que $R,\Theta$ sont ind�pendantes et calculer leurs lois marginales.
\end{enumerate}



\exercise
Soit $(X,Y)$ un vecteur al�atoire dans $\RR^2$ tel que $X \sim\mathcal{N}(1,1)$ et la loi conditionelle de $Y$ sachant $X=x$ est $\mathcal{N}(3x,4)$. 

\begin{enumerate}
\item Calculer la moyenne et la matrice de variance/covariance du couple $(X,Y)$. 
\item Donner la densit� du couple $(X,Y)$.
\item Montrer que $Y$ est une gaussienne.
\item Montrer que $(X,Y)$ est un vecteur gaussien.
\item Montrer que la loi conditionelle de $X$ sachant $Y=y$ est gaussienne.
\end{enumerate}


\exercise Soit $(X,Y)$ un vecteur al�atoire gaussien dans $\RR^2$ centr� et de matrice de covariance l'indentit� $I_2$. Soit $(Z,Q)$ le vecteur al�atoire defini par $Z=(X+Y)/2$ et $Q=(X-Y)/2$. On pose
$$
U = \frac{1}{2} (X-Z)^2 + \frac{1}{2} (Y-Z)^2
$$
 

\begin{enumerate}
\item Calculer la fonction caract�ristique du vecteur $(Z,Q)$ et montrer qu'il est un vecteur gaussien.
\item Montrer que $Z,Q$ sont ind�pendantes.
\item Calculer $\E[ U ]$ et $\text{Var}(U)$.
\item Montrer que $Z$ et $U$ sont ind�pendantes.
\item Donner la loi de $U$.
\end{enumerate}

\end{document}

