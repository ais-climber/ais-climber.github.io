\documentclass[12pt]{article}  
\usepackage{amssymb,amsthm,amsmath}
\usepackage{lscape}

\usepackage{bigstrut}
%\usepackage{MnSymbol}
\usepackage{bbm}
\usepackage{proof}
\usepackage{bussproofs}
\usepackage{tikz}
\usepackage{lingmacros}


\usepackage{hyperref}
\hypersetup{
    colorlinks,
    citecolor=black,
    filecolor=black,
    linkcolor=black,
    urlcolor=black
}


\newcommand{\existsgeq}{\mbox{\sf AtLeast}}
\newcommand{\Pol}{\mbox{\emph{Pol}}}
  \newcommand{\nonered}{\textcolor{red}{=}}
  \newcommand{\equalsred}{\nonered}
  \newcommand{\redstar}{\textcolor{red}{\star}}
    \newcommand{\dred}{\textcolor{red}{d}}
    \newcommand{\dmark}{\dred}
    \newcommand{\redflip}{\textcolor{red}{flip}}
        \newcommand{\flipdred}{\textcolor{red}{\mbox{\scriptsize \em flip}\ d}}
        \newcommand{\mdred}{\textcolor{blue}{m}\textcolor{red}{d}}
        \newcommand{\ndred}{\textcolor{blue}{n}\textcolor{red}{d}}
\newcommand{\arrowm}{\overset{\textcolor{blue}{m}}{\rightarrow} }
\newcommand{\arrown}{\overset{\textcolor{blue}{n}}{\rightarrow} }
\newcommand{\arrowmn}{\overset{\textcolor{blue}{mn}}{\longrightarrow} }
\newcommand{\arrowmonemtwo}{\overset{\textcolor{blue}{m_1 m_2}}{\longrightarrow} }
\newcommand{\bluen}{\textcolor{blue}{n}}
\newcommand{\bluem}{\textcolor{blue}{m}}
\newcommand{\bluemone}{\textcolor{blue}{m_1}}
\newcommand{\bluemtwo}{\textcolor{blue}{m_2}}
\newcommand{\blueminus}{\textcolor{blue}{-}}

\newcommand{\bluedot}{\textcolor{blue}{\cdot}}
\newcommand{\bluepm}{\textcolor{blue}{\pm}}
\newcommand{\blueplus}{\textcolor{blue}{+ }}
\newcommand{\translate}[1]{{#1}^{tr}}
\newcommand{\Caba}{\mbox{\sf Caba}} 
\newcommand{\Set}{\mbox{\sf Set}} 
\newcommand{\Pre}{\mbox{\sf Pre}} 
\newcommand{\wmarkpolarity}{\scriptsize{\mbox{\sf W}}}
\newcommand{\wmarkmarking}{\scriptsize{\mbox{\sf Mon}}}
\newcommand{\smark}{\scriptsize{\mbox{\sf S}}}
\newcommand{\bmark}{\scriptsize{\mbox{\sf B}}}
\newcommand{\mmark}{\scriptsize{\mbox{\sf M}}}
\newcommand{\jmark}{\scriptsize{\mbox{\sf J}}}
\newcommand{\kmark}{\scriptsize{\mbox{\sf K}}}
\newcommand{\tmark}{\scriptsize{\mbox{\sf T}}}
\newcommand{\greatermark}{\mbox{\tiny $>$}}
\newcommand{\lessermark}{\mbox{\tiny $<$}}
%%{\mbox{\ensuremath{>}}}
\newcommand{\true}{\top}
\newcommand{\false}{\bot}
\newcommand{\upred}{\textcolor{red}{\uparrow}}
\newcommand{\downred}{\textcolor{red}{\downarrow}}
\usepackage[all,cmtip]{xy}
\usepackage{enumitem}
\usepackage{fullpage}
\usepackage[authoryear]{natbib}
\usepackage{multicol}
\theoremstyle{definition}
\newtheorem{definition}{Definition}
\newtheorem{theorem}{Theorem}
\newtheorem{lemma}[theorem]{Lemma}
%\theoremstyle{case}


\newcounter{cases}
\newcounter{subcases}
\newenvironment{mycases}
  {%
    \setcounter{cases}{0}%
    \def\case
      {\bigskip
        \par\noindent
        \refstepcounter{cases}%
        \textbf{Case \thecases\ }
      }%
  }
  {%
    \par
  }
\newenvironment{subcases}
  {
    \setcounter{subcases}{0}%
    \def\subcase
      {\bigskip
        \par\noindent
        \refstepcounter{subcases}%
        \textbf{Subcase \thesubcases\ }
      }%
  }
  {%
  }
\renewcommand*\thecases{\arabic{cases}}
\renewcommand*\thesubcases{\roman{subcases}}


\newtheorem{claim}{Claim}
\newtheorem{corollary}{Corollary}
%\newtheorem{theorem}{Theorem}
\newtheorem{proposition}{Proposition}
\newtheorem{example}{Example}
\newtheorem{remark}[theorem]{Remark}
\newcommand{\semantics}[1]{[\![\mbox{\em $ #1 $\/}]\!]}
\newcommand{\abovearrow}[1]{\rightarrow\hspace{-.14in}\raiseonebox{1.0ex}
{$\scriptscriptstyle{#1}$}\hspace{.13in}}
\newcommand{\toplus}{\abovearrow{r}}
\newcommand{\tominus}{\abovearrow{i}} 
\newcommand{\todestroy}{\abovearrow{d}}
\newcommand{\tom}{\abovearrow{m}}
\newcommand{\tomprime}{\abovearrow{m'}}
\newcommand{\A}{\textsf{App}}
\newcommand{\At}{\textsf{At}}
\newcommand{\Emb}{\textsf{Emb}}
\newcommand{\EE}{\mathbb{E}}
\newcommand{\DD}{\mathbb{D}}
\newcommand{\PP}{\mathbb{P}}
\newcommand{\QQ}{\mathbb{Q}}
\newcommand{\LL}{\mathbb{L}}
\newcommand{\MM}{\mathbb{M}}
\usepackage{verbatim}
\newcommand{\TT}{\mathcal{T}}
\newcommand{\Marking}{\mbox{Mar}}
\newcommand{\Markings}{\Marking}
\newcommand{\Mar}{\Marking}
\newcommand{\Model}{\mathcal{M}}
\renewcommand{\SS}{\mathcal{S}}
\newcommand{\TTM}{\TT_{\Markings}}
\newcommand{\CC}{\mathbb{C}}
\newcommand{\erase}{\mbox{\textsf{erase}}}
\newcommand{\set}[1]{\{ #1 \}}
\newcommand{\arrowplus}{\overset{\blueplus}{\rightarrow} }
\newcommand{\arrowminus}{\overset{\blueminus}{\rightarrow} }
\newcommand{\arrowdot}{\overset{\bluedot}{\rightarrow} }
\newcommand{\arrowboth}{\overset{\bluepm}{\rightarrow} }
\newcommand{\arrowpm}{\arrowboth}
\newcommand{\arrowplusminus}{\arrowboth}
\newcommand{\arrowmone}{\overset{m_1}{\rightarrow} }
\newcommand{\arrowmtwo}{\overset{m_2}{\rightarrow} }
\newcommand{\arrowmthree}{\overset{m_3}{\rightarrow} }
\newcommand{\arrowmcomplex}{\overset{m_1 \orr m_2}{\longrightarrow} }
\newcommand{\arrowmproduct}{\overset{m_1 \cdot m_2}{\longrightarrow} }
\newcommand{\proves}{\vdash}
\newcommand{\Dual}{\mbox{\sc dual}}
\newcommand{\orr}{\vee}
\newcommand{\uar}{\uparrow}
\newcommand{\dar}{\downarrow}
\newcommand{\andd}{\wedge}
\newcommand{\bigandd}{\bigwedge}
\newcommand{\arrowmprime}{\overset{m'}{\rightarrow} }
\newcommand{\quadiff}{\quad \mbox{ iff } \quad}
\newcommand{\Con}{\mbox{\sf Con}}
\newcommand{\type}{\mbox{\sf type}}
\newcommand{\lang}{\mathcal{L}}
\newcommand{\necc}{\Box}
\newcommand{\vocab}{\mathcal{V}}
\newcommand{\wocab}{\mathcal{W}}
\newcommand{\Types}{\mathcal{T}_\mathcal{M}}
\newcommand{\mon}{\mbox{\sf mon}}
\newcommand{\anti}{\mbox{\sf anti}}
\newcommand{\FF}{\mathcal{F}}
\newcommand{\rem}[1]{\relax}


\newcommand{\raiseone}{\mbox{raise}^1}
\newcommand{\raisetwo}{\mbox{raise}^2}
\newcommand{\wrapper}[1]{{#1}}
\newcommand{\sfa}{\wrapper{\mbox{\sf a}}}
\newcommand{\sfb}{\wrapper{\mbox{\sf b}}}
\newcommand{\sfv}{\wrapper{\mbox{\sf v}}}
\newcommand{\sfw}{\wrapper{\mbox{\sf w}}}
\newcommand{\sfx}{\wrapper{\mbox{\sf x}}}
\newcommand{\sfy}{\wrapper{\mbox{\sf y}}}
\newcommand{\sfz}{\wrapper{\mbox{\sf z}}}
  \newcommand{\sff}{\wrapper{\mbox{\sf f}}}
    \newcommand{\sft}{\wrapper{\mbox{\sf t}}}
      \newcommand{\sfc}{\wrapper{\mbox{\sf c}}}
      \newcommand{\sfu}{\wrapper{\mbox{\sf u}}}
            \newcommand{\sfs}{\wrapper{\mbox{\sf s}}}
  \newcommand{\sfg}{\wrapper{\mbox{\sf g}}}

\newcommand{\sfvomits}{\wrapper{\mbox{\sf vomits}}}
\newlength{\mathfrwidth}
  \setlength{\mathfrwidth}{\textwidth}
  \addtolength{\mathfrwidth}{-2\fboxrule}
  \addtolength{\mathfrwidth}{-2\fboxsep}
\newsavebox{\mathfrbox}
\newenvironment{mathframe}
    {\begin{lrbox}{\mathfrbox}\begin{minipage}{\mathfrwidth}\begin{center}}
    {\end{center}\end{minipage}\end{lrbox}\noindent\fbox{\usebox{\mathfrbox}}}
    \newenvironment{mathframenocenter}
    {\begin{lrbox}{\mathfrbox}\begin{minipage}{\mathfrwidth}}
    {\end{minipage}\end{lrbox}\noindent\fbox{\usebox{\mathfrbox}}} 
 \renewcommand{\hat}{\widehat}
 \newcommand{\nott}{\neg}
  \newcommand{\preorderO}{\mathbb{O}}
 \newcommand{\PreorderP}{\mathbb{P}}
  \newcommand{\preorderE}{\mathbb{E}}
\newcommand{\preorderP}{\mathbb{P}}
\newcommand{\preorderN}{\mathbb{N}}
\newcommand{\preorderQ}{\mathbb{Q}}
\newcommand{\preorderX}{\mathbb{X}}
\newcommand{\preorderA}{\mathbb{A}}
\newcommand{\preorderR}{\mathbb{R}}
\newcommand{\preorderOm}{\mathbb{O}^{\bluem}}
\newcommand{\preorderPm}{\mathbb{P}^{\bluem}}
\newcommand{\preorderQm}{\mathbb{Q}^{\bluem}}
\newcommand{\preorderOn}{\mathbb{O}^{\bluen}}
\newcommand{\preorderPn}{\mathbb{P}^{\bluen}}
\newcommand{\preorderQn}{\mathbb{Q}^{\bluen}}
 \newcommand{\PreorderPop}{\mathbb{P}^{\blueminus}}
  \newcommand{\preorderEop}{\mathbb{E}^{\blueminus}}
\newcommand{\preorderPop}{\mathbb{P}^{\blueminus}}
\newcommand{\preorderNop}{\mathbb{N}^{\blueminus}}
\newcommand{\preorderQop}{\mathbb{Q}^{\blueminus}}
\newcommand{\preorderXop}{\mathbb{X}^{\blueminus}}
\newcommand{\preorderAop}{\mathbb{A}^{\blueminus}}
\newcommand{\preorderRop}{\mathbb{R}^{\blueminus}}
 \newcommand{\PreorderPflat}{\mathbb{P}^{\flat}}
  \newcommand{\preorderEflat}{\mathbb{E}^{\flat}}
\newcommand{\preorderPflat}{\mathbb{P}^{\flat}}
\newcommand{\preorderNflat}{\mathbb{N}^{\flat}}
\newcommand{\preorderQflat}{\mathbb{Q}^{\flat}}
\newcommand{\preorderXflat}{\mathbb{X}^{\flat}}
\newcommand{\preorderAflat}{\mathbb{A}^{\flat}}
\newcommand{\preorderRflat}{\mathbb{R}^{\flat}}
\newcommand{\pstar}{\preorderBool^{\preorderBool^{E}}}
\newcommand{\pstarplus}{(\pstar)^{\blueplus}}
\newcommand{\pstarminus}{(\pstar)^{\blueminus}}
\newcommand{\pstarm}{(\pstar)^{\bluem}}
\newcommand{\Reals}{\preorderR}
\newcommand{\preorderS}{\mathbb{S}}
\newcommand{\preorderBool}{\mathbbm{2}}
 \renewcommand{\o}{\cdot}
 \newcommand{\NP}{\mbox{\sc np}}
 \newcommand{\NPplus}{\NP^{\blueplus}}
  \newcommand{\NPminus}{\NP^{\blueminus}}
   \newcommand{\NPplain}{\NP}
    \newcommand{\npplus}{np^{\blueplus}}
  \newcommand{\npminus}{np^{\blueminus}}
   \newcommand{\npplain}{np}
   \newcommand{\np}{np}
   \newcommand{\Term}{\mbox{\sc t}}
  \newcommand{\N}{\mbox{\sc n}}
   \newcommand{\X}{\mbox{\sc x}}
      \newcommand{\Y}{\mbox{\sc y}}
            \newcommand{\V}{\mbox{\sc v}}
    \newcommand{\Nbar}{\overline{\mbox{\sc n}}}
    \newcommand{\Pow}{\mathcal{P}}
    \newcommand{\powcontravariant}{\mathcal{Q}}
    \newcommand{\Id}{\mbox{Id}}
    \newcommand{\pow}{\Pow}
   \newcommand{\Sent}{\mbox{\sc s}}
   \newcommand{\lookright}{\slash}
   \newcommand{\lookleft}{\backslash}
   \newcommand{\dettype}{(e \to t)\arrowminus ((e\to t)\arrowplus t)}
\newcommand{\ntype}{e \to t}
\newcommand{\etttype}{(e\to t)\arrowplus t}
\newcommand{\nptype}{(e\to t)\arrowplus t}
\newcommand{\verbtype}{TV}
\newcommand{\who}{\infer{(\nptype)\arrowplus ((\ntype)\arrowplus (\ntype))}{\mbox{who}}}
\newcommand{\iverbtype}{IV}
\newcommand{\Nprop}{\N_{\mbox{prop}}}
\newcommand{\VP}{{\mbox{\sc vp}}}
\newcommand{\CN}{{\mbox{\sc cn}}}
\newcommand{\Vintrans}{\mbox{\sc iv}}
\newcommand{\Vtrans}{\mbox{\sc tv}}
\newcommand{\Num}{\mbox{\sc num}}
%\newcommand{\S}{\mathbb{A}}
\newcommand{\Det}{\mbox{\sc det}}
\newcommand{\preorderB}{\mathbb{B}}
\newcommand{\simA}{\sim_A}
\newcommand{\simB}{\sim_B}
\newcommand{\polarizedtype}{\mbox{\sf poltype}}
\newcommand{\Diag}{\mbox{Diag}}
\newcommand{\OffDiag}{\mbox{Off-diag}}
\newcommand{\Pairs}{\mbox{Pairs}}
\newcommand{\Bad}{\mbox{Bad}}
\newcommand{\argmax}{\mbox{argmax}}
\newcommand{\Clamp}{\mbox{Clamp}}
\newcommand{\sClamp}{\mbox{subset-Clamp}}
\newcommand{\ordercanonical}{<_{\scriptstyle can}}
\newcommand{\lex}{\ordercanonical}
\newcommand{\lexcanonical}{\ordercanonical}
\newcommand{\precsubseteq}{\preceq_{\scriptsize subset}}
\newcommand{\approxsubset}{\approx_{\scriptsize subset}}

\begin{document}
\section{Producing antichain familes from suitable orders}

\begin{definition}
Let $n\geq 1$.   We write $[n]$ for $\set{1,\ldots, n}$.
An \emph{$n$-family} is  a family of sets
\[ S = (S_1, \ldots, S_n)\]
\rem{
A \emph{suitable $n$-family}
is
such that for all $i \neq j$ in $\set{1,\ldots, n}$,
the set-theoretic difference
$S_i\setminus S_j$ is non-empty (and hence so is $S_j\setminus S_i$).
}
For a family $S$, we write $s_{i,j}$
for $|S_i\cup S_j|$.  Note that $s_{i,j}$
is a number, not a set.
We also write $s_i$ for $s_{i,i}$.
For a family $T$, we would of course use the notation $t_{i,j}$.

$S$ is an \emph{antichain family} if whenever $i \neq j$,
$S_i$ is not a subset of $S_j$.

The \emph{size class} of $(i,j)$ is $\set{(a,b): s_{a,b}= s_{i,j}}$.

 We write $\mathcal{C}$ for the
set of all $n$-families $S$ of sets
\end{definition}



\begin{definition}
Let $n\geq 1$.
We define  sets $\Diag(n)$, $\OffDiag(n)$, and $\Pairs(n)$ as follows:
\[
\begin{array}{lcl}
\Diag(n) & = & \set{(i,j)\in [n]^2:  i = j}\\
\OffDiag(n) & = & \set{(i,j)\in [n]^2:  i < j}\\
\Pairs(n) & = & \Diag(n)\cup\OffDiag(n)\\
m  & = & |\Pairs(n)|\\
\end{array}
\]
Observe that $|\Diag(n)| = n$,
$|\OffDiag(n)| = \binom{n}{2}$,
so that $m =  |\Pairs(n)| = \binom{n+1}{2}$.

Frequently we drop the $n$ and just write
$\Diag$, $\OffDiag$, and $\Pairs$.
\end{definition}

\rem{
\begin{definition} Let $<$ be a   suitable linear preorder of $\Pairs(n)$.
We fix a well-order $\ordercanonical$ on $\Pairs(n)$
to be any well-order on $\Pairs(n)$ with the property that if $(a,b) < (c,d)$ then $(a,b) \ordercanonical (c,d)$.
\end{definition}
}

\begin{definition}\label{def-suitable}
A \emph{suitable linear preorder} on $\Pairs(n)$
is a relation $\preceq$ such that
\begin{enumerate}
    \item $\preceq$ is reflexive and transitive.
    \item For all $(i,j), (k,l)\in \Pairs(n)$,
 either $(i,j) \preceq (k,l)$ or $(k,l) \prec (i,j)$,
 where 
 \[ (i,j) \prec(k,l) \quadiff  (i,j) \preceq (k,l) \mbox{ but not } (k,l)\preceq (i,j)
 \]
    \item For $i < j$, $(i,i) \prec (i,j)$.
    For $i > j$, $(i,i) \prec (j,i)$.
 \end{enumerate}
We write $(i,j)\equiv (k,l)$ if
$(i,j)\preceq (k,l) \preceq (i,j)$.
\end{definition}

\begin{example} Let  $S$  be a family of sets,
and assume that $S$ is an antichain family.
Then the relation 
$\preceq$ is suitable, where $\preceq$ is defined by 
\[ (i,j) \preceq (k,l) \quadiff s_{i,j} \leq s_{k,l}
\]
\label{example-suitable}
\end{example}




\begin{theorem}
Let $\preceq$ be a suitable linear preorder.
Then there is a family of sets $S$
such that for all $(i,j)$, $(k,l)\in\Pairs(n)$,
\begin{equation}
    \label{goal}
 (i,j) \preceq  (k,l) \quadiff 
 s_{i,j}\leq s_{k,l}.
 \end{equation}
 \label{theorem-thoughts}
 \end{theorem}
 
 This representation theorem is tantamount to the completeness of the
 associated logical system.
 
\subsection{The Clamp Construction on Families of Sets}
 Let $S$ be a  family,
 and let $(i,j)\in\Pairs(n)$.
 Let $r\in \omega$.
 We define a new family 
 \[ \Clamp(S,i,j,r)\]
 from $S$, $i$, $j$, and $r$, as follows:
$*_1,\ldots, *_r$ be   fresh points.
For $a\in[n]$, let 
\[ \begin{array}{lcl}
\Clamp(S,i,j,p)_a & = & \left\{
\begin{array}{ll}
S_a \cup \set{*_1,\ldots, *_r} & \mbox{if $a\neq i$ and $a\neq j$}\\
 S_a & \mbox{if $a= i$ or $a = j$}\\
 \end{array}
 \right.
\end{array}
\]
In words, we are adding $r$ new points
simultaneously to all sets $S_a$, except
for $S_i$ and $S_j$.  In words, we clamp $S_i$ and $S_j$, and raise all other sets
by simultaneously adding points to them.    Note that the clamping raises all unions,
except for $S_i\cup S_i$, $S_j \cup S_j$, and $S_i\cup S_j$.

\begin{proposition}
Let $S$ be a family on $n$, and fix $i,j\in[n]$
and $r\in\omega$.
Write $T$ for $\Clamp(S,i,j,r)$.
\begin{enumerate}
    %\item  $T$ is an antichain family.
    \item For $(a,b)\notin \set{(i,j),(i,i),(j,j)}$, $t_{a,b} = s_{a,b} + r$.
   % $|T_{a} \cup T_{b}| =|S_{a} \cup S_{b}| +p $.
    \item For $(a,b)\in \set{(i,j),(i,i),(j,j)}$, $t_{a,b} = s_{a,b}$.
%    $|T_{a} \cup T_{b}| =|S_{a} \cup S_{b}| $.
    \item For all $(a,b), (c,d)$
    in \[\Pairs(n)\setminus
    \set{(i,j),(i,i),(j,j)} \]
we have 
\[ s_{a,b} - s_{c,d} = t_{a,b} - t_{c,d}.\]
%|S_{a}\cup S_{b}| - |S_{c}\cup S_{d}|
%= |T_{a}\cup T_{b}| - |T_{c}\cup T_{d}|.\]
\rem{\item $\Bad(T)\setminus\Bad(S)$ is a subset of
\[\set{((a,b),(c,d))\in \Pairs(n)^2 : \mbox{either $(a,b)$ or $(c,d)$ belongs to $ \set{(i,j),(i,i),(j,j)}$}}.\]
\item  \marginpar{This point is slightly off, in 
addition to being awkward.}
$(\Bad(T)\cap \OffDiag(n)^2)\setminus
(\Bad(S)\cap \OffDiag(n)^2)$ is exactly
\[ 
\begin{array}{lcl}
& \set{(i,j),(c,d))\in\OffDiag(n)^2 :0\leq |S_i\cup S_j| - |S_c \cup S_d| \leq p} \\
\cup & \set{(c,d),(i,j))\in\OffDiag(n)^2  :0\leq |S_i\cup S_j| - |S_c \cup S_d| \leq p} \\
\end{array}
\]
    \item If $i = j$ and $f(S) = 0$, then $f(T) = 0$.
}
\item If $s_{k,l}, s_{m,n} > s_{i,j}$,
then $s_{m,n}\leq s_{k,l}$ iff
 $t_{m,n}\leq t_{k,l}$.

\end{enumerate}
\label{proposition-clamp}
\end{proposition}

\subsection{Equalizing the sizes of some pairs in a family}

\begin{lemma}
Let $S$ be a family.  Let $k\geq 2$, and let 
\[p_1 = (a_1,b_1), \ldots, p_k =(a_k,b_k)\]
be pairs,
and assume that if $p_n = (a_n, b_n)$, 
then neither $(a_n,a_n)$ nor  $(b_n,b_n)$ is  on the list 
$p_1, \ldots, p_k$.
Then there is a family $T$ such that 
\begin{enumerate}
    \item $t_{a_1, b_1} = t_{a_2, b_2} = \cdots = t_{a_k, b_k}$.
    \item If $(k,l)$ and $(m,n)$ are any pairs with $s_{k,l}, s_{m,n} >
    \max_n s_{a_n,b_n}$,
    then 
    \[ \mbox{$t_{k,l} \leq t_{m,n} $ if and only if $s_{k,l} \leq s_{m,n} $}.\]
\label{equalize2}
\end{enumerate}
\label{lemma-equalize}
\end{lemma}

\begin{proof}
Reorder the given pairs so that
\[  s_{a_1, b_1} \leq s_{a_2, b_2} \leq \cdots\leq s_{a_k, b_k}.
\]
 Let \[ \begin{array}{lcl}
 T^1  & = &  \Clamp(S,a_2,b_2,s_{a_2, b_2} - s_{a_1, b_1})\\
T^2 & = & \Clamp(T^1,a_3, b_3, s_{a_3, b_3} - s_{a_2, b_2} )\\
  & \vdots   & \\
T^{k-1} & = & \Clamp(T^{k-2},a_k,b_k,
s_{a_k, b_k} - s_{a_{k-1}, b_{k-1}})\\
\end{array}
\]
Let $T = T^{k-1}$

To save on a lot of notation, let us write $s_i$ for $s_{a_i, b_i}$
and similarly for $t^j_i$.

An induction on $1\leq i \leq k -1$ shows that
for $1 \leq j \leq i+1$,
\begin{equation}
\label{equalization}
s_{i+1}  = t^i_{1} = t^i_{i} = \cdots t^i_{i} =  t^i_{i+1}.
\end{equation}
For $i = 1$, recall that $s_1 \leq s_2$.
Also, $t^1_1 = s_1 + (s_2 - s_1) = s_2$.
Moreover, $t^1_2= s_2$, since 
the definition of $T^1$ uses $\Clamp$ at $p_2$.



Assume (\ref{equalization}) for  $i$.
Let   $1\leq j \leq i+1$.
Then \[ t^{i+1}_j = s_{i+1} + (s_{i+2} - s_{i+1}) 
= s_{i+2}\]
Also, $t^{i+1}_{i+2} = s_{i+2} $ since $T^{i+1}$ uses $\Clamp$
at $p_{i+2}$.

Taking $i = k -1$ in (\ref{equalization}) proves our result.
\end{proof}
 
\subsection{Making a list of pairs larger than all $\prec$-predecessors
of it}

\begin{lemma}
Let $S$ be a family, and let $q_1\preceq q_2 \preceq \cdots \preceq q_k$ be a sequence 
from $\Pairs(n)$.  
Then there is a family $T$ such that
\begin{enumerate} 
    \item For $1\leq i,j \leq k$, $s_i \leq s_j$ iff $t_i \leq t_j$.
    \label{competitor1}
    \item for all 
pairs $p \prec q_1$, 
$t_{p}  <  t_{q_i}$ for all $i$.
\label{competitor2}
\end{enumerate}
\label{lemma-competitor}
\end{lemma}
 

 \begin{proof}
 Let $m =  \min_i s_{i} = \min_i s_{q_i}$.
We call a pair $p$ a \emph{size competitor} if 
$p\prec q_1$ and $t_p \geq m$.
 
List the size competitors as $p_1, \ldots, p_k$. 
 Note that if $p_i = (a_i,b_i)$, then none of the original
 points $q_j$ are $(a_i,a_i)$ or $(b_i,b_i)$
or $p_i$.   This
is because $p_i \prec q_j$, and 
$(a_i,a_i), (b_i,b_i) \prec p_i$.
(Recall Definition~\ref{def-suitable}.)
 Let \[ \begin{array}{lcl}
 T^1  & = &  \Clamp(S,p_1,  s_{p_1} -m + 1)\\

T^2 & = & \Clamp(T^1,p_2,  s_{p_2}-m + 1 )\\
  & \vdots   & \\
T^{k} & = & \Clamp(T^{k-1},p_k, s_{p_k}-m + 1 )\\
\end{array}
\]
Let $T = T^{k}$.
We claim that the original points $q_j$ have $t_{q_j} > t_{p}$
for all $p \prec q_1$.
The reason is that 
\[ t_{q_j} = s_{q_j} + ( s_{p_1} -m + 1) + (s_{p_2}-m + 1) + \cdots 
+ ( s_{p_k} -m + 1)
\]
On the other hand, for one of the size competitors, say $p_i$
\[\begin{array}{lcl}
t_{p_i} & = & s_{p_i} + ( s_{p_1}-m  + 1) + \cdots +
(s_{p_{i-1}} -m  + 1)
+ (s_{p_{i+1}} -m  + 1) + \cdots +(s_{p_k}-m + 1)\\
& > &  \\
\end{array}
\]
That is, $p_i$ is clamped as we move from $T^{i-1}$ to $T^i$.
The upshot is that \[t_{q_j} - t_{p_i} \geq s_{q_j} + (s_{p_i}-m + 1)  -
s_{p_i} = s_{q_j} - m + 1
\geq  s_{q_j} -s_{q_i} +1  = 1.\]
The reason that the first $\geq$ is not an equals sign $=$
is that it may be the case that $p_i$ is of the form $(a,a)$
and some other $p_{i'}$ is $(a,b)$ for some $b$.
At the end, we used the fact that $m \leq s_{q_i}$.
And so $t_{q_j} > t_{p_i}$.


For all $p \prec q_1$ which are not  size competitors,
the calculations are easier.  For such $p$,
$s_p < s_{q_j}$ for all $j$.
So we get 
$t_{q_j} - t_{p} \geq s_{q_j} - s_{p} > 0$.

This completes the proof.
\end{proof}

\section{Algorithm}

We prove Theorem~\ref{theorem-thoughts} by designing an algorithm
which represents a suitable linear preorder $\preceq$ on $\Pairs(n)$ by a family
of sets.   We construct the family using Lemmas~\ref{lemma-equalize}
a[~\ref{lemma-competitor} on each of the 
size classes of $\preceq$.


Consider the given ordering $\preceq$.
A \emph{size class} is a set of $\equiv$
 pairs $p$.   We list the size classes in order, from $\prec$-largest to 
 $\prec$-smallest.   Let's say the size classes in this order are 
 \[  C_1, C_2, \ldots, C_K \]
 Since we are listing them from 
 $\prec$-largest to 
 $\prec$-smallest, we have the following fact:
 if $(a,b) \prec (c,d)$, and also  $(a,b)\in C_i$, and finally
 $(c,d)\in C_j$,
 then $j < i$.
 
 Our algorithm has $K+1$ steps, one to start
 and one for each size class $C^i$. 
 We construct families $S^0, S^1, \ldots, S^K$.
 In Step $i$,
 we assure the following two assertions:
 \begin{enumerate}
 \item For all $(a,b)$, $(c,d)\in \bigcup_{1\leq j\leq i} C_j$,
\begin{equation}
    \label{goal-in-alg}
 (a,b) \preceq  (c,d) \quadiff 
 s^i_{a,b}\leq s^i_{c,d}.
 \end{equation}
%     \item  For $1 \leq j \leq i$,
 %the sizes of all pairs in $C_j$ are the same.
 %That is, for $p, q\in C_j$,  $s^i_p = s^i_q$.
     \item 
The sizes of all pairs in $\bigcup_{j< i} C_j$ are larger than the sizes of all pairs in $\bigcup_{j\geq i} C_j$.
   That is, for  $j < i$, $q\in C_j$ and $p\in C_i$, $s^i_q > s^i_p$.
 \end{enumerate}
 If we do this for $i = 0, 1, \ldots, K$, then $S^K$ will
 prove Theorem~\ref{theorem-thoughts}.
 
 We begin by taking $S^0$ to be the empty family\footnote{This
 choice of the empty family is not really needed.
 The proof in fact shows that we can take $S^0$ to be \emph{any}
 family on $n$}   on $n$.
 Assertions (1) and (2) from above are trivially satisfied.
 
 \paragraph{Step i ($1 \leq i \leq K$)}
 At the start of this step, we have a family $S^{i-1}$.
 We assume (1) and (2) for $i -1$.
 
 \paragraph{Substep $a$}
 Let $C_i$ be listed as $p_1, \ldots, p_k$.
 If $k = 1$, set $T = S^{i-1}$ and go to  Substep $b$ in the 
 next paragraph.  Otherwise, 
  use Lemma~\ref{lemma-equalize} with these pairs 
  $p_1, \ldots, p_k$ and with the family
  $S^{i-1}$.
  By the lemma, we get a new family which we'll call $T$.
  In it, all pairs in $C_i$ have the same size.
 By (2) for $i$, and by part~\ref{equalize2} of Lemma~\ref{lemma-equalize},
 we have (1) for $T$.
  
 \paragraph{Substep $b$}  
Enumerate $\bigcup_{1\leq j \leq i} C_j$ as 
$q_1 \preceq q_2\preceq \cdots \preceq q_k$.
(Note that elements of $C_i$
appear in the beginning of this list.)
Apply Lemma~\ref{lemma-competitor}
to this sequence and to the family $T$
from Substep $a$.  We get a new family, say 
$S^i$.  
Lemma~\ref{lemma-competitor}, part~\ref{competitor1}, insures that (1) holds for $S^i$, since it held for $T$.
 And Lemma~\ref{lemma-competitor}, part~\ref{competitor2},
 insures that (2) holds for $S^i$.
 
 \subsection{Example}
 
 \rem{
 task6 = [[(5,5),(6,6)], [(5,6),(4,4),(7,7)], 
         [(7,4),(4,5),(2, 2),(1,1),(0,0),(8,8), (3,3)], [(3, 2),  (2, 1),(3, 1),(7,0),(3, 0), (2, 0)],
	    [(1, 0), (4,0),(7,1),(7,2), (8,2),(8,1), (8,3), (8,7)], 
         [(7,3), (7,5),(7,6),(4,1),(4,2), (4,3), (8,6)],
	    [(6,0),(6,1),(6,2),(6,3),(5,1), (8,5), (8,4)],
         [(6,4),(5,0),(5,2),(5,3)]]
insertionSort(task6)
}




 Let $n = 9$, and let $\prec$ have size classes as shown in lists below:
 \[
 \begin{array}{l}
\ [(5,5),(6,6)] \\
 \  [(5,6),(4,4),(7,7)] \\
  \  [(7,4),(4,5),(2, 2),(1,1),(0,0),(8,8), (3,3)],\\
  \ [(3, 2),  (2, 1),(3, 1),(7,0),(3, 0), (2, 0)]\\
\	    [(1, 0), (4,0),(7,1),(7,2), (8,2),(8,1), (8,3), (8,7)]\\
 \        [(7,3), (7,5),(7,6),(4,1),(4,2), (4,3), (8,6)] \\
\	    [(6,0),(6,1),(6,2),(6,3),(5,1), (8,5), (8,4)]\\
 \        [(6,4),(5,0),(5,2),(5,3)]
         \end{array}
 \]

 
 We illustrate with step $6$.
 We begin with a family $S$ with cardinalities as shown below.
 \[
 \begin{array}[t]{l@{\qquad\qquad}l@{\qquad\qquad}l}
 \begin{array}{l}
|S[0]| = 65 \\
|S[1]| = 63  \\
|S[2]| = 63 \\
|S[3]| = 64 \\
|S[4]| = 64 \\
|S[5]| = 70 \\
|S[6]| = 71 \\
|S[7]| = 60 \\
|S[8]| = 68 \\
\hline
|S[5] \cup S[5]| = 70  \\
|S[6] \cup S[6]| = 71 \\
  \\
|S[5] \cup S[6]| = 79 \\
|S[4] \cup S[4]| = 64 \\
|S[7] \cup S[7]| = 60 \\
  \\
|S[7] \cup S[4]| = 79 \\
|S[4] \cup S[5]| = 79 \\
|S[2] \cup S[2]| = 63 \\
|S[1] \cup S[1]| = 63 \\
|S[0] \cup S[0]| = 65 \\
|S[8] \cup S[8]| = 68 \\
|S[3] \cup S[3]| = 64 \\
  \end{array}
&
  \begin{array}{l}
|S[3] \cup S[2]| = 80 \\
|S[2] \cup S[1]| = 80 \\ 
|S[3] \cup S[1]| = 80 \\
|S[7] \cup S[0]| = 80 \\
|S[3] \cup S[0]| = 80 \\
|S[2] \cup S[0]| = 80 \\
  \\
  
|S[1] \cup S[0]| = 81 \\
|S[4] \cup S[0]| = 81 \\
|S[7] \cup S[1]| = 81 \\
|S[7] \cup S[2]| = 81 \\
|S[8] \cup S[2]| = 81 \\
|S[8] \cup S[1]| = 81 \\
|S[8] \cup S[3]| = 81 \\
|S[8] \cup S[7]| = 81 \\
  \\
|S[7] \cup S[3]| = 82 \\
|S[7] \cup S[5]| = 82 \\
|S[7] \cup S[6]| = 82 \\
|S[4] \cup S[1]| = 82 \\
|S[4] \cup S[2]| = 82 \\
|S[4] \cup S[3]| = 82 \\
|S[8] \cup S[6]| = 82 \\
 \end{array}
 &
  \begin{array}{l}
|S[6] \cup S[0]| = 83 \\
|S[6] \cup S[1]| = 83 \\
|S[6] \cup S[2]| = 83 \\
|S[6] \cup S[3]| = 83 \\
|S[5] \cup S[1]| = 83 \\
|S[8] \cup S[5]| = 83 \\
|S[8] \cup S[4]| = 83 \\
  \\
|S[6] \cup S[4]| = 84 \\
|S[5] \cup S[0]| = 84 \\
|S[5] \cup S[2]| = 84 \\
|S[5] \cup S[3]| = 84 \\
 \end{array} 
 \end{array}
 \]
Step $6$ concerns the sixth size class, starting from the highest one.
So  we are working on the size class 
 \[  (7,4),(4,5),(2, 2),(1,1),(0,0),(8,8), (3,3)
 \]
 The first step is to equalize the sizes in this class, using Lemma~\ref{lemma-equalize}.
 We reorder this in size order in our family above, obtaining
  \[  (1,1), (2, 2), (3,3),  (0,0),(8,8), (7,4),(4,5)
 \]
 We therefore calculate:
\[ \begin{array}{lcl}
 T^1  & = &  \Clamp(S,(2,2), 63-63)\\

T^2 & = & \Clamp(T^1,(3,3),  64-63)\\

T^{3} & = & \Clamp(T^{2}, (0,0), 65-64 )\\
T^{4} & = & \Clamp(T^{3}, (8,8), 68-65 )\\
T^{5} & = & \Clamp(T^4, (7,4),79-69 )\\
T^{6} & = & \Clamp(T^{5}, (4,5), 79-79)\\
\end{array}
\]
 We use $T^6$.
 
After equalizing, we get 
\[
 \begin{array}[t]{l@{\qquad\qquad}l@{\qquad\qquad}l}
\begin{array}{l}
|S[0]| = 103 \\
|S[1]| = 103 \\
|S[2]| = 103 \\
|S[3]| = 103 \\
|S[4]| = 72 \\
|S[5]| = 94 \\
|S[6]| = 111 \\
|S[7]| = 84 \\
|S[8]| = 103 \\
\hline
|S[5] \cup S[5]| = 94 \\
|S[6] \cup S[6]| = 111 \\
  \\
|S[5] \cup S[6]| = 119 \\
|S[4] \cup S[4]| = 72 \\
|S[7] \cup S[7]| = 84 \\
  \\
|S[7] \cup S[4]| = 103 \\
|S[4] \cup S[5]| = 103 \\
|S[2] \cup S[2]| = 103 \\
|S[1] \cup S[1]| = 103 \\
|S[0] \cup S[0]| = 103 \\
|S[8] \cup S[8]| = 103 \\
|S[3] \cup S[3]| = 103 \\
 \end{array}
&
  \begin{array}{l}
|S[3] \cup S[2]| = 120 \\
|S[2] \cup S[1]| = 120 \\
|S[3] \cup S[1]| = 120 \\
|S[7] \cup S[0]| = 120 \\
|S[3] \cup S[0]| = 120 \\
|S[2] \cup S[0]| = 120 \\
  \\
|S[1] \cup S[0]| = 121 \\
|S[4] \cup S[0]| = 121 \\
|S[7] \cup S[1]| = 121 \\
|S[7] \cup S[2]| = 121 \\
|S[8] \cup S[2]| = 121 \\
|S[8] \cup S[1]| = 121 \\
|S[8] \cup S[3]| = 121 \\
|S[8] \cup S[7]| = 121 \\
  \\
|S[7] \cup S[3]| = 122 \\
|S[7] \cup S[5]| = 122 \\
|S[7] \cup S[6]| = 122 \\
|S[4] \cup S[1]| = 122 \\
|S[4] \cup S[2]| = 122 \\
|S[4] \cup S[3]| = 122 \\
|S[8] \cup S[6]| = 122 \\
 \end{array}
 &
  \begin{array}{l}
|S[6] \cup S[0]| = 123 \\
|S[6] \cup S[1]| = 123 \\
|S[6] \cup S[2]| = 123 \\
|S[6] \cup S[3]| = 123 \\
|S[5] \cup S[1]| = 123 \\
|S[8] \cup S[5]| = 123 \\
|S[8] \cup S[4]| = 123 \\
  \\
|S[6] \cup S[4]| = 124 \\
|S[5] \cup S[0]| = 124 \\
|S[5] \cup S[2]| = 124 \\
|S[5] \cup S[3]| = 124 \\
  \end{array}
    \end{array}
  \]
Note that for classes above the classes of interest in this step,
the sizes stay larger during the equalization.

  \pagebreak
  
At this point, the size competitors are $(5,6)$ and $(6,6)$.
We want to make the sizes of the sets in our current size class larger than the sizes of $(5,6)$ and $(6,6)$.
So we use Lemma~\ref{lemma-competitor}.
That is, we clamp $(5,6)$ and $(5,5)$, increasing all sets by
one more than the difference of the sizes of those sets with $103$, 
We get

\[
\begin{array}[t]{l@{\qquad\qquad}l@{\qquad\qquad}l}
\begin{array}{l}
|S[0]| = 120 \\
|S[1]| = 120 \\
|S[2]| = 120 \\
|S[3]| = 120 \\
|S[4]| = 89 \\
|S[5]| = 94 \\
|S[6]| = 111 \\
|S[7]| = 101 \\
|S[8]| = 120 \\
\hline
|S[5] \cup S[5]| = 94 \\
|S[6] \cup S[6]| = 111 \\
  \\
|S[5] \cup S[6]| = 119 \\
|S[4] \cup S[4]| = 89 \\
|S[7] \cup S[7]| = 101 \\
  \\
|S[7] \cup S[4]| = 120 \\
|S[4] \cup S[5]| = 120 \\
|S[2] \cup S[2]| = 120 \\
|S[1] \cup S[1]| = 120 \\
|S[0] \cup S[0]| = 120 \\
|S[8] \cup S[8]| = 120 \\
|S[3] \cup S[3]| = 120 \\
 \end{array}
&
  \begin{array}{l}

|S[3] \cup S[2]| = 137 \\
|S[2] \cup S[1]| = 137 \\
|S[3] \cup S[1]| = 137 \\
|S[7] \cup S[0]| = 137 \\
|S[3] \cup S[0]| = 137 \\
|S[2] \cup S[0]| = 137 \\
  \\
|S[1] \cup S[0]| = 138 \\
|S[4] \cup S[0]| = 138 \\
|S[7] \cup S[1]| = 138 \\
|S[7] \cup S[2]| = 138 \\
|S[8] \cup S[2]| = 138 \\
|S[8] \cup S[1]| = 138 \\
|S[8] \cup S[3]| = 138 \\
|S[8] \cup S[7]| = 138 \\
  \\
|S[7] \cup S[3]| = 139 \\
|S[7] \cup S[5]| = 139 \\
|S[7] \cup S[6]| = 139 \\
|S[4] \cup S[1]| = 139 \\
|S[4] \cup S[2]| = 139 \\
|S[4] \cup S[3]| = 139 \\
|S[8] \cup S[6]| = 139 \\
   \end{array}
 &
  \begin{array}{l}
|S[6] \cup S[0]| = 140 \\
|S[6] \cup S[1]| = 140 \\
|S[6] \cup S[2]| = 140 \\
|S[6] \cup S[3]| = 140 \\
|S[5] \cup S[1]| = 140 \\
|S[8] \cup S[5]| = 140 \\
|S[8] \cup S[4]| = 140 \\
  \\
|S[6] \cup S[4]| = 141 \\
|S[5] \cup S[0]| = 141 \\
|S[5] \cup S[2]| = 141 \\
|S[5] \cup S[3]| = 141 \\
\end{array}
\end{array}
\]
Note that it wasn't really necessary to clamp $(5,6)$ after we clamped $(5,5)$.
So our algorithm does a bit of work that is not necessary.   It could be elaborated to 
produce slightly smaller sets in the end.  But it is correct.

\vfil\eject


\section{Subset Information}

{\bf starting here, it's a self-contained presentation}

\begin{definition}
Let $n\geq 1$.   We write $[n]$ for $\set{1,\ldots, n}$.
An \emph{$n$-family} is  a family of  finite sets
\[ S = (S_1, \ldots, S_n)\]
For a family $S$, we write $s_{i,j}$
for $|S_i\cup S_j|$.  Note that $s_{i,j}$
is a number, not a set.
We also write $s_i$ for $s_{i,i}$.
For a family $T$, we would of course use the notation $t_{i,j}$.
\end{definition}

\begin{definition}
Let $n\geq 1$.
We define  sets $\Diag(n)$, $\OffDiag(n)$, and $\Pairs(n)$ as follows:
\[
\begin{array}{lcl}
\Diag(n) & = & \set{(i,j)\in [n]^2:  i = j}\\
\OffDiag(n) & = & \set{(i,j)\in [n]^2:  i < j}\\
\Pairs(n) & = & \Diag(n)\cup\OffDiag(n)\\
m  & = & |\Pairs(n)|\\
\end{array}
\]
\end{definition}

\paragraph{Notation}
For the elements of $\Pairs(n)$, we use two kinds of notation.
We could denote pairs by, well, pairs, writing $(a,b)$, or $(i,j)$, or something similar.
But sometimes we just want to say: let $p$ be a pair.  And so at times we
use notation like $p$, $q$, etc. for pairs.
I'm sure that this will be confusing, but the fact is that each
kind of notation has its place in what we're going to do.

\begin{definition}
A \emph{suitable pair of relations on $\Pairs(n)$} is a pair of relations\footnote{I know that
the notation $\precsubseteq$ is lousy.   I don't think we can use the letter $s$ alone there,
since we use that all over the place for other things.    But surely there is a better notation.
Anyways, it's all a macro that can be changed.}
\[ (\preceq, \precsubseteq) \]
such that 
\begin{enumerate}
\item $\preceq$ and $\precsubseteq$ are preorders on $\Pairs(n)$.
\item  
 For all $(i,j), (k,l)\in \Pairs(n)$,
 either $(i,j) \preceq (k,l)$ or $(k,l) \prec (i,j)$.
 \item If $i < j$, then $(i,i) \precsubseteq (i,j)$.  If $j < i$, then $(i,i) \precsubseteq (j,i)$. 
 \item If $(i,i) \precsubseteq (k,\ell)$ and $(j,j) \precsubseteq (k,\ell)$ and $i < j$,
 then $(i,j) \precsubseteq (k,\ell)$.
\item If $p \precsubseteq q$, then $p\preceq q$.
\item If $p \precsubseteq q$ and $q\preceq p$, then $q \precsubseteq p$.
\end{enumerate}
\label{def-suitable-pair}
\end{definition}


%(Compare with Definition~\ref{def-suitable}.)

\begin{definition}
An $n$-family $S$ is \emph{$\precsubseteq$-preserving}
if $p\precsubseteq q$ implies that $S_p \subseteq S_q$.

$S$ is \emph{$\precsubseteq$-reflecting} if 
$S_p \subseteq S_q$ implies that $p\precsubseteq q$. 
\end{definition}


\begin{example} Let  $S$  be a family of sets.
Then the pair of relations $(\preceq, \precsubseteq)$
is suitable, where 
\[ \begin{array}{lcl}
(i,j) \preceq (k,l) & \quadiff  & s_{i,j} \leq s_{k,l}\\
(i,j) \precsubseteq (k,l) & \quadiff  & S_{i,j} \subseteq S_{k,l}\\
\end{array}
\]
Moreover, $S$ preserves and reflects $\precsubseteq$.
\label{example-suitable-pair}
\end{example}

Here is our main representation result:

\begin{theorem}
Let $(\preceq, \precsubseteq)$ be a suitable pair of relations.
Then there is a family of sets $S$
such that for all $(i,j)$, $(k,l)\in\Pairs(n)$,
\begin{equation}
    \label{goal-main1}
 (i,j) \preceq  (k,l) \quadiff 
 s_{i,j}\leq s_{k,l}.
 \end{equation}
 And also
 \begin{equation}
    \label{goal-main2}
 (i,j) \precsubseteq  (k,l) \quadiff 
S_{i,j}\subseteq S_{k,l}.
 \end{equation}
 That is, (\ref{goal}) holds for $\preceq$, and $S$ preserves and reflects $\precsubseteq$.
 \label{theorem-thoughts-subset}
 \end{theorem}
 
 
\subsection{The s-Clamp Construction on Families  $S$  and Preorders
$\precsubseteq$}
 Let $S$ be an $n$-family.
 Let $(i,j)\in\Pairs(n)$.
 Let $r\in \omega$.
 We define a new family 
 \[ \sClamp(S,i,j,r)\]
 from $S$, $i$, $j$, and $r$, as follows:
$*_1,\ldots, *_r$ be   fresh points.
For $a\in[n]$, let 
\[ \begin{array}{lcl}
\sClamp(S,i,j,r)_a & = & \left\{
\begin{array}{ll}
S_a \cup \set{*_1,\ldots, *_r} & \mbox{if $\nott ((a,a)\precsubseteq (i,j))$}\\
 S_a & \mbox{if $(a,a)\precsubseteq (i,j)$}\\ 
 \end{array}
 \right.
\end{array}
\]
In words, we are adding $r$ new points
simultaneously to all sets $S_a$, except when 
$\precsubseteq$ ``wants 
$S_a$ to be a subset of  $S_i\cup S_j$.''

Note that $\precsubseteq$ is involved in the definition of $\sClamp$, but it is not
part of the notation.


\begin{proposition}
Let $S$ be a family on $n$, and fix $i,j\in[n]$
and $r\in\omega$.
Write $T$ for $\sClamp(S,i,j,r)$.
\begin{enumerate}
    %\item  $T$ is an antichain family.
 %   \item For $(a,b)\notin \set{(i,j),(i,i),(j,j)}$, 
   % $|T_{a} \cup T_{b}| =|S_{a} \cup S_{b}| +p $.
    \item For $(a,b)\precsubseteq (i,j)$, $T_{a,b} = S_{a,b}$.
    \label{part-easy}
\item For $(a,b)$ such that $(i,j) \prec (a,b)$, $T_{a,b} =  S_{a,b}\cup\set{*_1,\ldots, *_r}$.
\label{part-bigger}
\rem{    \item For all $(a,b), (c,d)$
    in \[\Pairs(n)\setminus
    \set{(i,j),(i,i),(j,j)} \]
we have 
\[ s_{a,b} - s_{c,d} = t_{a,b} - t_{c,d}.\]
%|S_{a}\cup S_{b}| - |S_{c}\cup S_{d}|
%= |T_{a}\cup T_{b}| - |T_{c}\cup T_{d}|.\]
}
%item If $s_{k,l}, s_{m,n} > s_{i,j}$,
%then $s_{m,n}\leq s_{k,l}$ iff
% $t_{m,n}\leq t_{k,l}$.
\item If $S$ preserves $\precsubseteq$,
then $T$  preserves $\precsubseteq$.
\label{part-preserve}
\item If $S$  reflects $\precsubseteq$,
then $T$  reflects $\precsubseteq$.
\label{part-reflect}
\end{enumerate}
\label{proposition-sClamp}
\end{proposition}

\begin{proof}
Here is the proof of part (\ref{part-easy}).
If $(a,b)\precsubseteq (i,j)$, then both $(a,a)\precsubseteq (i,j)$ and $(b,b)\precsubseteq (i,j)$.
And so $T_a = S_a$ and $T_b = S_b$.
 Thus, $T_{a,b} = T_a \cup T_b = S_a \cup S_b = S_{a,b}$.
 
 Turning to part (\ref{part-bigger}), let  $(i,j) \prec (a,b)$.
 We claim that either $\nott ((a,a) \precsubseteq (i,j))$ or $\nott ((b,b) \precsubseteq (i,j))$.
To see this, suppose towards a contradiction that both $(a,a) \precsubseteq(i,j)$ and $(b,b) \precsubseteq (i,j)$.
 Then $(a,b) \precsubseteq (i,j)$, and so $(a,b) \preceq (i,j)$.  And this contradicts 
$(i,j) \prec (a,b)$.

Without loss of generality, 
$\nott ((a,a) \precsubseteq (i,j))$.
Then $T_a = S_a \cup \set{*_1,\ldots, *_r}$.
And so
\[ T_{a,b} = T_a \cup T_b = S_a\cup\set{*_1,\ldots, *_r} \cup S_b = S_{a,b}\cup\set{*_1,\ldots, *_r}.\]
This completes the proof.


In  part (\ref{part-preserve}),
let $(a,b) \precsubseteq (c,d)$.  We show that $T_{a,b} \subseteq T_{c,d}$.

If $(a,b)\precsubseteq (i,j)$, then $T_{a,b} = S_{a,b}$.
By our assumption that $S$ preserves $\precsubseteq$, $S_{a,b} \subseteq S_{c,d}$.
And clearly $S_{c,d} \subseteq T_{c,d}$.
So in this case we easily get $T_{a,b} \subseteq T_{c,d}$.

It remains to argue the case when
$\nott((a,b)\precsubseteq (i,j))$.
So in this case,  
$T_{a,b} =  S_{a,b} \cup \set{*_1,\ldots, *_r}$.
We claim that in this case,
$T_{c,d} =  S_{c,d} \cup \set{*_1,\ldots, *_r}$.
This again would imply $T_{a,b} \subseteq T_{c,d}$.

Suppose towards a contradiction that 
$T_{c,d} \neq  S_{c,d} \cup \set{*_1,\ldots, *_r}$.
Then $T_{c,d} =  S_{c,d} $.
In this case, $T_{c,c} = S_{c,c}$ and $T_{d,d} = S_{d,d}$.
So  $(c,c)\precsubseteq (i,j)$,
and   $(d,d)\precsubseteq (i,j)$.
And now we have   $(c,d)\precsubseteq (i,j)$.
Recall that 
$(a,b) \precsubseteq (c,d)$. 
And so we have $(a,b) \precsubseteq (i,j)$. 
This is a contradiction to the assumption in this case that 
$\nott((a,b)\precsubseteq (i,j))$.

Part (\ref{part-reflect}) is easier:  if we take any family which reflects $\precsubseteq$ and
add fresh points in any way whatsoever, the resulting family will reflect $\precsubseteq$.
\end{proof}

\subsection{Equalizing the sizes in a size class}

In this section, we fix a suitable pair 
$(\preceq, \precsubseteq)$.

Recall that a \emph{size class} of $\preceq$ is an equivalence class $C$
of the induced equivalence relation $\approx$ induced by $\preceq$.

The size classes of $\preceq$ have an induced strict order.

\begin{lemma}
Let $S$ be a family which preserves and reflects $\precsubseteq$.  Let $k\geq 2$, and let 
\[ C = \set{p_1, \ldots, p_k} \]
be a size class of  $\preceq$.
 
Then there is a family $T$ such that 
\begin{enumerate}
    \item 
 For $1\leq r,s \leq k$, $t_{p_r} = t_{p_s}$.
 In words, the  unions corresponding to the pairs in $C$ have equal size in $T$.   
%    $t_{a_1, b_1} = t_{a_2, b_2} = \cdots = t_{a_k, b_k}$.
    \item If $(k,l)$ and $(m,n)$ are any pairs 
    which belong to larger size classes than $C$,
    then 
    \[ \mbox{$t_{k,l} \leq t_{m,n} $ if and only if $s_{k,l} \leq s_{m,n} $}.\]
    \item $T$ preserves and reflects $\precsubseteq$. 
\label{equalize2}
\end{enumerate}
\label{lemma-equalize-subset}
\end{lemma}


\begin{proof}
Before we begin the construction of $T$,
we have an observation.
$C$, being a size class of $\approx$, splits into one or more
$\approxsubset$ classes, where $\approxsubset$ is the equivalence relation induced by $\precsubseteq$.
The observation is that if $q_1$ and $q_2$ are members of $C$ which are in different $\approxsubset$
classes, then neither $q_1 \precsubseteq q_2$ nor  $q_2 \precsubseteq q_1$.
To see this, suppose towards a contradiction that  $q_1 \precsubseteq q_2$.
Then since we also have $q_2 \preceq q_1$, we have 
$q_1 \precsubseteq q_2$ by one of the properties of the suitable pair $(\preceq, \precsubseteq)$.



Let us chose one pair in each $\approxsubset$-class of 
$C$, and list the chosen pairs in size order according to $S$.
That is, we have 
\[ (a_1, b_1), \ldots, (a_k,b_k) \]
so that every element of $C$ is $\approxsubset$ to exactly one pair on this list,
and the order is chosen so that as numbers,
\[  s_{a_1, b_1} \leq s_{a_2, b_2} \leq \cdots\leq s_{a_k, b_k}.
\]
 Let \[ \begin{array}{lcl}
 T^1  & = &  \sClamp(S,a_2,b_2,s_{a_2, b_2} - s_{a_1, b_1})\\
T^2 & = & \sClamp(T^1,a_3, b_3, s_{a_3, b_3} - s_{a_2, b_2} )\\
  & \vdots   & \\
T^{k-1} & = & \sClamp(T^{k-2},a_k,b_k,
s_{a_k, b_k} - s_{a_{k-1}, b_{k-1}})\\
\end{array}
\]
Let $T = T^{k-1}$

To save on a lot of notation, let us write $s_i$ for $s_{a_i, b_i}$
and similarly for $t^j_i$.

An induction on $1\leq i \leq k -1$ shows that
for $1 \leq j \leq i+1$,
\begin{equation}
\label{equalization}
s_{i+1}  = t^i_{1} = t^i_{i} = \cdots t^i_{i} =  t^i_{i+1}.
\end{equation}
For $i = 1$, recall that $s_1 \leq s_2$.
Also, $t^1_1 = s_1 + (s_2 - s_1) = s_2$.
This is one place where we use our observation at the beginning of the proof.
That is, $\nott ((a_1,b_1) \precsubseteq (a_2,b_2))$.
Moreover, $t^1_2= s_2$, since 
the definition of $T^1$ uses $\sClamp$ at $p_2$.



Assume (\ref{equalization}) for  $i$.
Let   $1\leq j \leq i+1$.
Then \[ t^{i+1}_j = s_{i+1} + (s_{i+2} - s_{i+1}) 
= s_{i+2}\]
Again, we are using  our observation at the beginning of the proof.
Also, $t^{i+1}_{i+2} = s_{i+2} $ since $T^{i+1}$ uses $\sClamp$
at $p_{i+2}$.

Taking $i = k -1$ in (\ref{equalization}) proves the first part of our result.

The second is an easy induction using Proposition~\ref{proposition-sClamp}, part (\ref{part-bigger}).

The last part is also an easy induction, this time using 
 Proposition~\ref{proposition-sClamp}, parts (\ref{part-preserve}) and (\ref{part-reflect}).
\end{proof}

\subsection{Making a list of pairs larger than all $\prec$-predecessors
of it}
\begin{lemma}
Let $S$ be a family which  preserves and reflects $\precsubseteq$.  
Let $q_1$, $\ldots$, $q_{\ell}$ be a sequence from 
%$q_1\preceq q_2 \preceq \cdots \preceq q_k$ be a sequence 
from $\Pairs(n)$.  
Then there is a family $T$ such that
\begin{enumerate} 
    \item For $1\leq i,j \leq k$, $s_i \leq s_j$ iff $t_i \leq t_j$.
    (Here we are writing $s_i$ and $s_j$ for $s_{q_i}$ and $s_{q_j}$, respectively.)
    \label{competitor1}
    \item For all 
pairs $p $ which are $\prec$-below all $q_j$, we also have 
$t_{p}  <  t_{q_i}$ for all $i$.
\label{competitor2}
  \item $T$ preserves and reflects $\precsubseteq$. 

\end{enumerate}
\label{lemma-competitor-subset}
\end{lemma}
 

 \begin{proof}
 Let $m =  \min_i s_{i} = \min_i s_{q_i}$.
We call a pair $p$ a \emph{size competitor} if 
$p\prec q_j$ for all $j$, and yet  $t_p \geq m$.
 
List the size competitors as $p_1, \ldots, p_k$. 
 Note that for all size competitors $p_i$ and all of the original points
 $q_j$,  we have $\nott (q_j \precsubseteq p_i)$.
For if we did have $q_j \precsubseteq p_i$, then we would have 
$q_j \preceq p_i$; and the definition of a size competitor
insures that that $p_i \prec q_j$ for all $i, j$.


 Let \[ \begin{array}{lcl}
 T^1  & = &  \sClamp(S,p_1,  s_{p_1} -m + 1)\\

T^2 & = & \sClamp(T^1,p_2,  s_{p_2}-m + 1 )\\
  & \vdots   & \\
T^{k} & = & \sClamp(T^{k-1},p_k, s_{p_k}-m + 1 )\\
\end{array}
\]
Let $T = T^{k}$.
We claim that the original points $q_j$ have $t_{q_j} > t_{p}$
for all $p \prec q_1$.
The reason is that 
\[ t_{q_j} = s_{q_j} + ( s_{p_1} -m + 1) + (s_{p_2}-m + 1) + \cdots 
+ ( s_{p_k} -m + 1)
\]
On the other hand, for one of the size competitors, say $p_i$, we have
\begin{equation}
\label{nearendsubset}
\begin{array}{lcl}
t_{p_i} & = & s_{p_i} + ( s_{p_1}-m  + 1) + \cdots +
(s_{p_{i-1}} -m  + 1)
+ (s_{p_{i+1}} -m  + 1) + \cdots +(s_{p_k}-m + 1)\\
%& > &  \\
\end{array}
\end{equation}
That is, $p_i$ is $s$-clamped as we move from $T^{i-1}$ to $T^i$.
The upshot is that 
\begin{equation}\label{upshot}
t_{q_j} - t_{p_i} \geq s_{q_j} + (s_{p_i}-m + 1)  -
s_{p_i} = s_{q_j} - m + 1
\geq  s_{q_j} -s_{q_i} +1  = 1.
\end{equation}
The reason that the first $\geq$ is not an equals sign $=$
is that it may be the case that $p_i\precsubseteq p_{i'}$.

At the end of (\ref{upshot}), we used the fact that $m \leq s_{q_i}$.
By (\ref{upshot}), we see that $t_{q_j} > t_{p_i}$.


For all $p \prec q_1$ which are not  size competitors,
the calculations are easier.  For such $p$,
$s_p < s_{q_j}$ for all $j$.
So we get 
$t_{q_j} - t_{p} \geq s_{q_j} - s_{p} > 0$.


This completes the proof of the first part of our lemma, and the other parts follow as in Lemma~\ref{lemma-equalize-subset}.
This completes the proof.
\end{proof}

\section{Algorithm}

We prove Theorem~\ref{theorem-thoughts-subset} by designing an algorithm
which represents a suitable pair $(\preceq,\precsubseteq)$ on $\Pairs(n)$ by a family
of sets.   We construct the required family using Lemmas~\ref{lemma-equalize-subset}
and~\ref{lemma-competitor-subset} on each of the 
size classes of $\preceq$.


Consider the given ordering $\preceq$.
Recall that a \emph{size class} is a set of $\equiv$
 pairs $p$.   We list the size classes in order, from $\prec$-largest to 
 $\prec$-smallest.   Let's say the size classes in this order are 
 \[  C_1, C_2, \ldots, C_K \]
 Since we are listing them from 
 $\prec$-largest to 
 $\prec$-smallest, we have the following fact:
 if $(a,b) \prec (c,d)$, and also  $(a,b)\in C_i$, and finally
 $(c,d)\in C_j$,
 then $j < i$.
 
 Our algorithm has $K+1$ steps, one to start
 and one for each size class $C_i$. 
 We construct families $S^0, S^1, \ldots, S^K$.
 In Step $i$,
 we assure the following two assertions:
 \begin{enumerate}
 \item For all $(a,b)$, $(c,d)\in \bigcup_{1\leq j\leq i} C_j$,
\begin{equation}
    \label{goal-in-alg}
 (a,b) \preceq  (c,d) \quadiff 
 s^i_{a,b}\leq s^i_{c,d}.
 \end{equation}
%     \item  For $1 \leq j \leq i$,
 %the sizes of all pairs in $C_j$ are the same.
 %That is, for $p, q\in C_j$,  $s^i_p = s^i_q$.
     \item 
The sizes of all pairs in $\bigcup_{j< i} C_j$ are larger than the sizes of all pairs in $\bigcup_{j\geq i} C_j$.
   That is, for  $j < i$, $q\in C_j$ and $p\in C_i$, $s^i_q > s^i_p$.
   \item $S^i$ preserves and reflects $\precsubseteq$.
 \end{enumerate}
 If we do this for $i = 0, 1, \ldots, K$, then $S^K$ will
 prove Theorem~\ref{theorem-thoughts}.
 
 We begin by taking $S^0$ to be any family
 which preserves and reflects $\precsubseteq$.
 The canonical choice is to take $S^0_i$ to be the join-prime up-closed
 subsets of $(X,\preceq)$ that contain $(i,i)$ as an element, where $X$ is the set of all unary and binary union terms.
 

 Assertions (1) and (2) from above are trivially satisfied.
 
 \paragraph{Step i ($1 \leq i \leq K$)}
 At the start of this step, we have a family $S^{i-1}$.
 We assume (1) and (2) for $i -1$.
 
 \paragraph{Substep $a$}
 Let $C_i$ be listed as $p_1, \ldots, p_k$.
 If $k = 1$, set $T = S^{i-1}$ and go to  Substep $b$ in the 
 next paragraph.  Otherwise, 
  use Lemma~\ref{lemma-equalize-subset} with these pairs 
  $p_1, \ldots, p_k$ and with the family
  $S^{i-1}$.
  By the lemma, we get a new family which we'll call $T$.
  In it, all pairs in $C_i$ have the same size.
 By (2) for $i$, and by part~\ref{equalize2} of Lemma~\ref{lemma-equalize-subset},
 we have (1) for $T$.
 (3) holds by  Lemma~\ref{lemma-equalize-subset}.
  
 \paragraph{Substep $b$}  
Write $\bigcup_{1\leq j \leq i} C_j$ as 
$\set{q_1, q_2,\ldots, q_k}$.
Apply Lemma~\ref{lemma-competitor-subset}
to this set and to the family $T$
from Substep $a$.  We get a new family, say 
$S^i$.  
Lemma~\ref{lemma-competitor-subset}, part~\ref{competitor1}, insures that (1) holds for $S^i$, since it held for $T$.
 And Lemma~\ref{lemma-competitor-subset}, part~\ref{competitor2},
 insures that (2) holds for $S^i$.
  (3) holds by  Lemma~\ref{lemma-competitor-subset}.
 
\end{document}

 \begin{definition}
 Let $<$ be a suitable linear
 preorder on $\Pairs(n)$.
 Let $S$ be a  family on $n$.
 The pair $(a,b)$ is \emph{bad} in $S$
 if either of the following conditions holds:
 \begin{enumerate}
     \item 
     There is some $(c,d)$ such that $s_{c,d} = s_{a,b}$, but $(a,b) > (c,d)$.
     \item
     There is some $(c,d)$ such that $s_{c,d} > s_{a,b}$, but $(a,b) \geq (c,d)$.
 \end{enumerate}
 In either case, we say that $(c,d)$
 is \emph{bad for $(a,b)$}.
 
 We say that $(a,b)$ is \emph{good  in $S$} if
 it is not bad in $S$.  And $(a,b)$
 is \emph{very good in $S$} if $(a,b)$ is good in $S$, and for
 all $(c,d)$ such that $s_{c,d}  >  s_{a,b}$,
 $(c,d)$ is also good in $S$.
 \label{definition-bad}
 \end{definition}

\begin{proposition}
If there are no bad pairs for $S$,
then (\ref{goal}) holds.
Thus, if (\ref{goal}) does not hold,
then some pair is not very good.
\end{proposition} 

\begin{proposition}
If $(e,f)$ and $(g,h)$ are both good in $S$, and $s_{e,f} = s_{g,h}$, then $(e,f) \equiv (g,h)$.
\label{prop-bothgood}
\end{proposition}
 
 \begin{definition}
 Let $<$ be a suitable linear preorder order on $\Pairs(n)$.  Let $m = \binom{n+1}{2}$.

Suppose that $S$ does not meet the goal for $<$.
Let $p(S)$
be a pair $(a,b)$ which is bad and with
$s_{a,b}$ maximal among the bad  pairs.
And in case of ties, we take the $\ordercanonical$-least such pair.
Further let 
 \[
 g(S) = \mbox{the number of pairs which
 are bad for $p(S)$ in $S$}.
\]
If $f(S) = 0$, then we set $g(S) = 0$ as well.

We further let \[ h(S)= (p(S),g(S)).\]
Note that $h(S) \in \Pairs(n) \times [m]$.
We regard $\Pairs(n) \times [m]$ as a well-ordered set using the lexicographic order on the product,
where $\Pairs(n)$ is well-ordered by $\lex$, and $[m]$ by the order on natural numbers.
 \end{definition}
 
 
 \begin{lemma} [Main Lemma]
 If $h(S) \neq (0,0)$, then there is some $T$ such that 
 either $T$ meets the goal, or else
 $h(T)\prec h(S)$.
 \end{lemma}
 
 This implies that  we can find one that satisfies
 (\ref{goal}) by starting with the empty family
 (or any other family) and
 repeatedly finding families with smaller and smaller value of $h$.
 Eventually, we reach some $T$ such   achieves the goal in (\ref{goal}).
 
 The rest of this section is devoted to the proof.
 We assume that $h(S) \neq (0,0)$.
 Let $p(S)$ be $(a,b)$.
 Recall that $(a,b)$ is bad and of maximal size in $S$ for the bad pairs.
 Thus
 \begin{equation}\label{verygood}
 \mbox{$(e,f)$ is very good in $S$ iff $s_{e,f} > s_{a,b}$}
 \end{equation}
 
 \begin{mycases}
 
 \section{Case 1}
 
 
 \case \label{case1}
  There is some $(c,d)$ such that $s_{c,d} = s_{a,b}$, but $(a,b) > (c,d)$.
  
 Let $T = \Clamp(S,c,d,1)$.
 Then for all $(e,f)$ except $(c,d)$, $(c,c)$, and $(d,d)$, 
 $t_{e,f} = s_{e,f}+1$.
 But when $(e,f)$ is either $(c,d)$, or $(c,c)$, or $(d,d)$, 
 $t_{e,f} = s_{e,f}$.
 In particular, 
 \[t_{a,b} = s_{a,b}+1 = s_{c,d} + 1 = t_{c,d}+ 1> t_{c,d}\]
 That is, $(c,d)$ is not bad for $(a,b)$ in $T$.
 
Note as well that for all $(e,f)$ such that $s_{e,f} > s_{a,b}$,
$(e,f)$ is not one of the pairs $(c,d)$, $(c,c)$, or $(d,d)$.
So $(e,f)$ is very good for $S$.  Indeed, $(e,f)$ is very good for $T$ as well.

\begin{subcases}
\subcase  $(a,b)$ is good for $T$.

Then, $(a,b)$ would be very good for $T$.
This is because the size classes above $(a,b)$ are the same in $S$ as in $T$.

In this subcase,  we claim that
$p(T) \lex p(S)$.  
The reason is that $p(T)$ would be the lexicog

And since we are using the lexicographic
order in connection with $h$, we see that $h(T) \prec h(S)$.

\subcase 
$(a,b)$ is bad for $T$.
Then it is not very good, either.   
In this case, we 
have (\ref{verygood}) for $T$.   This implies that
$f(T) = f(S)$. 
We claim that $p(T) = p(S) = (a,b)$.  This is because $(a,b)$
is of maximal size for a bad pair for $T$, and in case of a tie,
it would be lexicographically least in that set.    (That is, the
size class of $(a,b)$ in $T$ is a subset of the size class of $(a,b)$ in $S$.
Thus, the the lexicographically
least element of the second set, $(a,b)$, is also 
he lexicographically
least element of the first set.)
But we also have $g(T) < g(S)$, since 
$(c,d)$ is not bad for $(a,b)$ in $T$, and (it is easy to check)
that every pair which is bad for $(a,b)$ in $T$ is already bad for $(a,b)$
in $S$.

This again implies that $h(T) \prec h(S)$.

\end{subcases}


\section{Case 2}


 \case \label{case2}
 Case~\ref{case1} does not hold, and
 there is some $(c,d)$ such that $s_{c,d} > s_{a,b}$, but $(a,b) \equiv (c,d)$.
 We fix such a pair
  $(c,d)$ which minimizes $s_{c,d}$.
  
\begin{claim}\label{sizeclassOfab}
All pairs $(e,f)$ in the size class   
of $(a,b)$ in $S$ have $(e,f) \leq (a,b)$.
\end{claim}

\begin{proof}
If $(a,b) > (e,f)$, then we would be in  Case~\ref{case1},
contrary to the assumption in this case.
\end{proof}


\begin{claim}\label{nopairs}
There are no pairs $(e,f)$ with 
$s_{a,b} < s_{e,f} < s_{c,d}$.
\end{claim}

\begin{proof}
Suppose towards a contradiction that $(e,f)$ exists.
By (\ref{verygood}), $(e,f)$ is good in $S$.
So $(e,f) < (c,d)$.  By the choice of $(c,d)$, either $(a,b) < (e,f)$,
or $(e,f) > (a,b)$.
If $(a,b) < (e,f)$, we  would have $(a,b)< (c,d)$, contradicting the 
choice of $(c,d)$ in Case 2.
If $(e,f) > (a,b)$, then since we also have $(a,b) \equiv (c,d)$,
we contradict our earlier observation that $(e,f) < (c,d)$.
\end{proof}


\rem{
We claim that for all $(e,f)$ such that  $s_{e,f} = s_{a,b}$,
we must have  $(e,f) \geq (c,d)$.
Suppose that $(e,f) < (c,d)$.  Then since $(a,b) \equiv (c,d)$,
we have  $(a,b)>(e,f)$.  And then $(e,f)$ would put us in Case~\ref{case1},
contrary to the assumption in this case.
}

Let $r = s_{c,d} - s_{a,b}$.
 Enumerate the size class of $(c,d)$ by
 \[ (c_1,d_1), \ldots, (c_p, d_p)\]
 Let \[ \begin{array}{lcl}
 T_1  & = &  \Clamp(S,c_1,d_1,r)\\
T_2 & = & \Clamp(T_1,c_2, d_2, r)\\
  & \vdots   & \\
T_p & = & \Clamp(T_{p-1},c_{p},c_{p}, r)\\
\end{array}
\]
Let $T = T_p$.


\begin{claim}\label{sizeclaimCase1}
\begin{enumerate}
    \item For all $(e,f)$, $t_{e,f} = s_{e,f} +qr$ for some $q\leq p$.
        \label{sizeclaimCase1Part4}
    \item For $i = 1, \ldots, p$, $t_{c_i,d_i} =s_{c_i,d_i} + (p-1)r$.
    \label{sizeclaimCase1Part1}
    \item $t_{a,b} =s_{a,b} + pr = (s_{c,d}-r) + pr = t_{c,d}$.
        \label{sizeclaimCase1Part2}
    \item If $s_{e,f} > s_{c,d}$, then $t_{e,f} = s_{e,f} +pr$.
        \label{sizeclaimCase1Part3}

    \item If $s_{e,f} < s_{a,b}$, then $t_{e,f} < t_{c,d}$.
        \label{sizeclaimCase1Part5}
            \item If $t_{c,d} < t_{e,f}$, then $s_{c,d} < s_{e,f}$.
        \label{sizeclaimCase1Part5a}
    \item The size class of $(c,d)$ in $T$ is the 
    size class of $(c,d)$ in $S$ together with the size class of $(a,b)$ in $S$.
     \label{sizeclaimCase1Part6}
    \item If $s_{e,f} > s_{c,d}$, then the size
     class of $(e,f)$ in $T$ is the 
    size class of $(e,f)$ in $S$. 
        \label{sizeclaimCase1Part7}
\end{enumerate}
\end{claim}


\begin{proof}
Parts~\ref{sizeclaimCase1Part4},
\ref{sizeclaimCase1Part1}, and~\ref{sizeclaimCase1Part2}
are clear from the definition of $\Clamp$ and the choice of $r$.
Part~\ref{sizeclaimCase1Part3} comes from the observation that
if $s_{e,f} > s_{c,d}$, then $(e,f)$ is not among
the following paris: $(c_1,d_1)$, $(c_1, c_1)$, $(d_1,d_1)$,
$\ldots$, $(c_p,d_p)$, $(c_p, c_p)$, $(d_p,d_p)$.

Here is the proof of 
Part~\ref{sizeclaimCase1Part5}.
Let $s_{e,f} < s_{a,b}$.   Let $q$ be as in Parts~\ref{sizeclaimCase1Part1} for
$(e,f)$.
Then $t_{e,f} = s_{e,f} +qr \leq   s_{e,f} +pr  < s_{a,b} + pr = t_{c,d}$.

Part~\ref{sizeclaimCase1Part5a} follows from Parts~\ref{sizeclaimCase1Part3} and~\ref{sizeclaimCase1Part5}.

Parts~\ref{sizeclaimCase1Part6} and~\ref{sizeclaimCase1Part7}
follow easily from the previous parts.
\end{proof}


\begin{claim} $(c,d)$ is good for $T$.
\label{claimcdgood}
\end{claim}

\begin{proof}
Suppose not.  
As listed in Definition~\ref{definition-bad},
there are two ways that $(c,d)$ could be bad for $T$.
There might be an element $(e,f)$ in the size class of $(c,d)$ in $T$
such that $(e,f) > (c,d)$.
By Claim~\ref{sizeclaimCase1}, part~\ref{sizeclaimCase1Part6},
$(e,f)$ is either in the size class of $(c,d)$ in $S$, or else it is in the size class of $(a,b)$ in $S$.
The first case would imply that $(c,d)$ is bad in $S$, contrary to
(\ref{verygood}).
In the second case, we have $(e,f) \leq (a,b)$ by Claim~\ref{sizeclassOfab}.
But $(a,b) \equiv (c,d)$, and so $(e,f) \leq (c,d)$.   This contradicts $(e,f) > (c,d)$.

The second way that $(c,d)$ could be bad for $T$ is that there is some
$(e,f)$ in a larger size class than $(c,d)$ in $T$ such that $(c,d) \geq (e,f)$.
But by Claim~\ref{sizeclaimCase1}, part~\ref{sizeclaimCase1Part7},
such $(e,f)$ would belong to a larger size class than $(c,d)$ in $S$.
And so $(c,d)$ would be bad in $S$; we know that this is a contradiction.
\end{proof}

\begin{claim}\label{claimPropagateVeryGood}
If $(e,f)$ is very good for $S$, then $(e,f)$ is very good for $T$.
Thus $f(T) \leq f(S)$.
\end{claim}

\begin{proof}
By (\ref{verygood}), $s_{e,f}>s_{a,b}$.
By Claim~\ref{nopairs},  $s_{e,f} \geq s_{c,d}$.

Let us first consider the case when $s_{e,f} = s_{c,d}$.
By Proposition~\ref{prop-bothgood},  $(e,f) \equiv (c,d)$.
If either
Looking back at Definition~\ref{definition-bad}, we see that
if  $(e,f)$ were bad for $T$, then $(c,d)$ would also be bad for $T$, 
contradicting  Claim~\ref{claimcdgood}.



If $s_{e,f}  > s_{c,d}$, then the size class of $(e,f)$ in $T$
is the same as that in $S$; and the same holds for $(g,h)$ with $s_{g,h} > s_{e,f}$.
It follows easily that $(e,f)$ is good in $T$.


This completes the proof.
\end{proof}

We conclude Case~\ref{case2} by checking that $h(T)\prec h(S)$.


\begin{subcases}
\subcase  
Some pair $(a',b')$ in the size class of 
$(a,b)$  in $S$
 is also  good for $T$.
We show that every such $(a',b')$ is very good for $T$.
Let $(e,f)$ be such that  $t_{e,f} >t_{a',b'} = t_{c,d}$.
By Claim~\ref{sizeclaimCase1},
Part~\ref{sizeclaimCase1Part5a}, 
$s_{e,f}  > s_{c,d}$.
By (\ref{verygood}), 
 $(e,f)$ is very good for $S$.
By Claim~\ref{claimPropagateVeryGood},  $(e,f)$ is good for $T$.

At this point in the subcase, we know that $(a',b')$ is very good for $T$, and that it is bad for $S$.
Recall also Claim~\ref{claimPropagateVeryGood}.
We see that  $f(T) < f(S)$.  Thus $h(T)\prec h(S)$.


\subcase
No pair $(a',b')$ in the size class of 
$(a,b)$  in $S$
 is also  good for $T$.
We show in this subcase that every $(e,f)$ which is very good for $T$ is very good for $S$.
For if $(e,f)$ were very good for $T$, then $t_{e,f} \geq t_{a,b}$.
If $t_{e,f} = t_{a,b}$, then $(e,f)$ is in the size class of $(e,f)$ in $T$.
By Claim~\ref{sizeclaimCase1},
Part~\ref{sizeclaimCase1Part6}, $(e,f)$ is either in the size class of $(c,d)$ in $S$
(and then it is clearly very good in $S$)
or else  $(e,f)$ is either in the size class of $(a,b)$ in $S$
(and then we would be in Subcase i).
 
As we saw in Claim~\ref{claimPropagateVeryGood}, $f(S) \leq f(T)$.
In this subcase, we actually have equality: $f(S) = f(T)$.
This is because $(a,b)$ is bad for $T$.

Continuing with this subcase, $(a,b)$ is a pair which is bad in $T$
and of maximal size.
As in Subcase ii of Case~\ref{case1}, 
 we claim that $p(T) = p(S) = (a,b)$.  This is because $(a,b)$
is of maximal size for a bad pair for $T$, and in case of a tie,
it would be lexicographically least in that set. 
And  we again have $g(T) < g(S)$, since 
$(c,d)$ is not bad for $(a,b)$ in $T$, and (it is easy to check)
that every pair which is bad for $(a,b)$ in $T$ is already bad for $(a,b)$
in $S$.

Finally, we check $g(T) < g(S)$, and this will conclude the proof
that  
 $h(T)\prec h(S)$.
We know that $(c,d)$ is not bad for $(a,b)$ in $T$,
since $t_{c,d} = t_{a,b}$.
Suppose that $(e,f)$ is bad for 
 $(a,b)$ in $T$.
 If $t_{e,f} > t_{a,b}$, then also 
  $s_{e,f} > s_{a,b}$, due to 
Claim~\ref{sizeclaimCase1},
Part~\ref{sizeclaimCase1Part5a}.
And then $(e,f)$ would be bad for 
 $(a,b)$ in $S$.
 If $t_{e,f} = t_{a,b}$, then $(e,f)$ is either in the size class of $(a,b)$ in $S$
 or in the size class of $(c,d)$ in $S$.   In the former case, 
 $(e,f)$ would be bad for 
 $(a,b)$ in $S$.
 In the latter case, we get a contradiction.
 We would have $(e,f) < (a,b)$ by badness in $S$, and yet 
 $(e,f) \equiv (c,d)$ (since the two pairs are in the same size class),
 and $(c,d) \equiv (a,b)$ (by the statement of Case~\ref{case2}).
 
 We conclude that the pairs which are bad for $(a,b)$ in $T$
 are all bad for $(a,b)$ in $S$, and $(c,d)$ is not among them.
 This implies the desired conclusion: $g(T) < g(S)$.

\end{subcases}

\section{Case 3}
\label{case3}

\case Cases~\ref{case1} and~\ref{case2} do not hold, and
 there is some $(c,d)$ such that $s_{c,d} > s_{a,b}$, but $(a,b) > (c,d)$.
 Note that 
(\ref{verygood}) still holds, and so $(c,d)$ if very good in $S$.
 
 
 \begin{claim}\label{sizeclassOfabCase3}
All pairs $(e,f)$ in the size class   
of $(a,b)$ in $S$ have $(e,f) \leq (a,b)$.
All pairs $(e,f)$ in the size class   
of $(c,d)$ in $S$ have $(c,d) \equiv (e,f)$.
\end{claim}

\begin{proof}
For the first assertion: 
if $(a,b) > (e,f)$, then we would be in  Case~\ref{case1},
contrary to the assumption in this case.
The second assertion comes from Proposition~\ref{prop-bothgood}.
\end{proof}


 
\begin{claim}\label{nopairsCase3}
There are no pairs $(e,f)$ with 
$s_{a,b} < s_{e,f} < s_{c,d}$.
\end{claim}

\begin{proof}
Suppose towards a contradiction that $(e,f)$ exists.
By (\ref{verygood}), $(e,f)$ is good in $S$.
So $(e,f) < (c,d)$.  By the choice of $(c,d)$, either
$(e,f) > (a,b)$
or else $(a,b)\equiv (e,f)$.
If $(a,b) < (e,f)$, we  would have $(a,b)< (c,d)$, contradicting the 
choice  $(a,b) > (c,d) $ as stated in Case 3.
If $(e,f) \equiv (a,b)$,
we would be in Case 2 rather than Case 3.
\end{proof}
 
 
Let $r = s_{c,d} - s_{a,b}+1$.
 Enumerate the size class of $(c,d)$ by
 \[ (c_1,d_1), \ldots, (c_p, d_p)\]
 Let \[ \begin{array}{lcl}
 T_1  & = &  \Clamp(S,c_1,d_1,r)\\
T_2 & = & \Clamp(T_1,c_2, d_2, r)\\
  & \vdots   & \\
T_p & = & \Clamp(T_{p-1},c_{p},c_{p}, r)\\
\end{array}
\]
Let $T = T_p$.


\begin{claim}\label{sizeclaimCase3}
\begin{enumerate}
    \item For all $(e,f)$, $t_{e,f} = s_{e,f} +qr$ for some $q\leq p$.
        \label{sizeclaimCase3Part4}
    \item For $i = 1, \ldots, p$, $t_{c_i,d_i} =s_{c_i,d_i} + (p-1)r$.
    \label{sizeclaimCase3Part1}
    \item $t_{a,b} =s_{a,b} + pr = (s_{c,d}-r +1) + pr  >  (s_{c,d}-r ) + pr  = t_{c,d}$.
        \label{sizeclaimCase3Part2}
    \item If $s_{e,f} > s_{c,d}$, then $t_{e,f} = s_{e,f} +pr$.
        \label{sizeclaimCase3Part3}

    \item If $s_{e,f} < s_{a,b}$, then $t_{e,f} < t_{c,d}$.
        \label{sizeclaimCase3Part5}
            \item If $t_{c,d} < t_{e,f}$, then $s_{c,d} < s_{e,f}$.
        \label{sizeclaimCase3Part5a}
    \item The size class of $(c,d)$ in $T$ is the 
    size class of $(c,d)$ in $S$.
    % together with the size class of $(a,b)$ in $S$.
     \label{sizeclaimCase3Part6}
        \item The size class of $(a,b)$ in $T$ is the 
    size class of $(a,b)$ in $S$.
    % together with the size class of $(a,b)$ in $S$.
     \label{sizeclaimCase3Part6a}
    \item If $s_{e,f} > s_{c,d}$, then the size
     class of $(e,f)$ in $T$ is the 
    size class of $(e,f)$ in $S$. 
        \label{sizeclaimCase3Part7}
\end{enumerate}
\end{claim}


\begin{proof}
Parts~\ref{sizeclaimCase1Part4},
\ref{sizeclaimCase1Part1}, and~\ref{sizeclaimCase1Part2}
are clear from the definition of $\Clamp$ and the choice of $r$.

Part~\ref{sizeclaimCase1Part3} comes from the observation that
if $s_{e,f} > s_{c,d}$, then $(e,f)$ is not among
the following paris: $(c_1,d_1)$, $(c_1, c_1)$, $(d_1,d_1)$,
$\ldots$, $(c_p,d_p)$, $(c_p, c_p)$, $(d_p,d_p)$.

Here is the proof of 
Part~\ref{sizeclaimCase1Part5}.
Let $s_{e,f} < s_{a,b}$.   Let $q$ be as in Parts~\ref{sizeclaimCase1Part1} for
$(e,f)$.
Then $t_{e,f} = s_{e,f} +qr \leq   s_{e,f} +pr  < s_{a,b} + pr = t_{c,d}$.

Part~\ref{sizeclaimCase1Part5a} follows from Parts~\ref{sizeclaimCase1Part3} and~\ref{sizeclaimCase1Part5}.

Parts~\ref{sizeclaimCase1Part6} and~\ref{sizeclaimCase1Part7}
follow easily from the previous parts.
\end{proof}


 
 
 
 \end{mycases}
 
 
 
 \end{document}


\begin{definition}
Let $<$ be a suitable linear preorder order on $\Pairs(n)$.
Let $S$ be a suitable family.  We define several
  functions from $\mathcal{C}$ to $[n]\cup\set{0}$ and to $[\binom{n+1}{2}]\cup \set{0}$:

\begin{enumerate}
\item $\Bad(S) = \set{((i,j),(k,l))\in \Pairs^2 : \nott
\bigl( (i,j) \leq  (k,l) \mbox{ iff } |S_i \cup S_j | \leq |S_k \cup S_l |\bigr)}$.
\item $f(S) = |\Bad(S)\cap \OffDiag(n)^2|$.
\item $W(S) = \set{i\in[n]: \mbox{for some
$(k,l)\in \Pairs$, $((i,i),(k,l))\in \Bad(S)$}}$.
\item \[\begin{array}{lcl}g(S) & = & 
\left\{\begin{array}{ll}
\argmax_{i\in W(S)} |S_i| &  \mbox{if $W(S)\neq\emptyset$}\\
0 & \mbox{if $W(S) = \emptyset$}
\end{array}
\right.
\end{array}
\]
If there are multiple $i\in W(S)$ which maximize
$|S_i|$, $g(S)$ may be taken to be any of these.
For concreteness, we take the least in the natural order on $[n]$.
\item If $g(S)\neq 0$, then
\[h(S) = 
|\set{(k,l)\in \Pairs : \nott
\bigl( (g(S),g(S)) \leq  (k,l) \mbox{ iff } |S_{g(S)}| \leq |S_k \cup S_l |\bigr)}|\]
And if $g(S) = 0$, $h(S) = 0$ as well.
\item $i(S) = (f(S),g(S),h(S))$.
\end{enumerate}
\end{definition}

Note that $<$ figures into the definitions of all of these,
but it is left out of the notation.
We fix $<$ for the rest of this section.


\begin{proposition}
\begin{enumerate}
    \item $\Bad(S) = \emptyset$ iff condition (\ref{goal}) holds.
    \item \label{twoconditionsBad}
    $((i,j),(k,l))\in\Bad(S)$ iff one of the
    following conditions holds:
  \begin{enumerate}
    \item  $(i,j) <  (k,l)$, but  $|S_i \cup S_j | \geq |S_k \cup S_l|$.
    \item 
    $(i,j) \equiv (k,l)$, but  $|S_k \cup S_l | < |S_i \cup S_j|$.
    \item  $(k,l) <  (i,j)$, but  $|S_k \cup S_l | \geq |S_i \cup S_j|$.
 \item 
    $(k,l) \equiv (i,j)$, but  $|S_i \cup S_j | < |S_k \cup S_l|$.
\end{enumerate} 
Note that (c) and (d) are the same as (a)
and (b), but with the pairs $(i,j)$ and $(k,l)$
interchanged.
\item $\Bad(S)$ is a symmetric relation on $\Pairs(n)$.
    \item If $\Bad(S) \neq \emptyset$, then either $f(S) > 0$ or $W(S)\neq \emptyset$.
    \item If $W(S)\neq \emptyset$, then $g(S) \in [n]$.
    (That is, if $W(S)\neq \emptyset$, then $g(S)> 0 $.)
    \item If $\Bad(S) \neq \emptyset$ and $f(S) = 0$,
    so that $W(S)\neq \emptyset$ and $g(S)\in [n]$, then $h(S) >0$.
\item $i(S) \in ([m]\cup\set{0}) \times ([n]\cup\set{0}) \times
([n]\cup\set{0})$, where $m = \binom{n+1}{2}$ is $|\Pairs(n)|$.
\end{enumerate}
\end{proposition}

\subsection{The Clamp Construction on Families of Sets}
 Let $S$ be a suitable family,
 and let $(i,j)\in\Pairs(n)$.
 Let $r\in \omega$.
 We define a new family 
 \[ \Clamp(S,i,j,r)\]
 from $S$, $i$, $j$, and $r$, as follows:
$*_1,\ldots, *_r$ be   fresh points.
For $a\in[n]$, let 
\[ \begin{array}{lcl}
\Clamp(S,i,j,r)_a & = & \left\{
\begin{array}{ll}
S_a \cup \set{*_1,\ldots, *_r} & \mbox{if $a\neq i$ and $a\neq j$}\\
 S_a & \mbox{if $a= i$ or $a = j$}\\
 \end{array}
 \right.
\end{array}
\]
In words, we are adding $r$ new points
simultaneously to all sets $S_a$, except
for $S_i$ and $S_j$.

\begin{proposition}
Let $S$ be an antichain family, and fix $i,j\in[n]$
and $p\in\omega$.
Write $T$ for $\Clamp(S,i,j,p)$.
\begin{enumerate}
    \item  $T$ is suitable.
    \item For $(a,b)\notin \set{(i,j),(i,i),(j,j)}$,
    $|T_{a} \cup T_{b}| =|S_{a} \cup S_{b}| +p $.
    \item For $(a,b)\in \set{(i,j),(i,i),(j,j)}$,
    $|T_{a} \cup T_{b}| =|S_{a} \cup S_{b}| $.
    \item For all $(a,b), (c,d)$
    in \[\Pairs(n)\setminus
    \set{(i,j),(i,i),(j,j)} \]
we have 
\[ |S_{a}\cup S_{b}| - |S_{c}\cup S_{d}|
= |T_{a}\cup T_{b}| - |T_{c}\cup T_{d}|.\]
\item $\Bad(T)\setminus\Bad(S)$ is a subset of
\[\set{((a,b),(c,d))\in \Pairs(n)^2 : \mbox{either $(a,b)$ or $(c,d)$ belongs to $ \set{(i,j),(i,i),(j,j)}$}}.\]
\item  \marginpar{This point is slightly off, in 
addition to being awkward.}
$(\Bad(T)\cap \OffDiag(n)^2)\setminus
(\Bad(S)\cap \OffDiag(n)^2)$ is exactly
\[ 
\begin{array}{lcl}
& \set{(i,j),(c,d))\in\OffDiag(n)^2 :0\leq |S_i\cup S_j| - |S_c \cup S_d| \leq p} \\
\cup & \set{(c,d),(i,j))\in\OffDiag(n)^2  :0\leq |S_i\cup S_j| - |S_c \cup S_d| \leq p} \\
\end{array}
\]
    \item If $i = j$ and $f(S) = 0$, then $f(T) = 0$.
\end{enumerate}
\end{proposition}


\subsection{The order $\prec$, and the main result}
\begin{definition}
Let $\preceq$ be the lexicographic order on 
\[ ([ m]\cup\set{0}) \times ([n]\cup\set{0}) \times
([n]\cup\set{0})
\]
where $m = \binom{n+1}{2}$, and where 
each factor ordered as in $\omega$.
\end{definition}

\begin{lemma}
\label{lemma-main}
For every $S\in\mathcal{C}$
such that $i(S)\neq (0,0,0)$, 
there is some $T\in\mathcal{C}$
such that $i(T)\prec i(S)$. 
\end{lemma}

\begin{mycases}
\case
$f(S) > 0$.

{\bf the argument below is totally off}

In this case, take $(i,j)$ and $(k,l)$
in $\OffDiag(n)$ such that 
(without loss of generality)
one of the four 
conditions in part (\ref{twoconditionsBad})
holds.  We shall assume that it is the 
first condition, since the arguments in
the other three cases are nearly the same.
So we have  $(i,j) <  (k,l)$ but
 $|S_i \cup S_j | \geq |S_k \cup S_l|$.

 Let $p =|S_k \cup S_l| - |S_i \cup S_j |+1 $.
 Let $T = \Clamp(S,i,j,p)$.
Then it is fairly easy to check that $((i,j),(k,l))\notin f(T)$,
but that for pairs $(a,b)\in \OffDiag(n)$
which are different from $(i,j)$,
\[ |T_a \cup T_b| = |S_a \cup S_b|\]
This implies that 
\[ \Bad(T)\cap \OffDiag(n)^2 \subseteq (\Bad(S)\cap \OffDiag(n)^2 )\setminus
 \set{((i,j),(k,l)),((k,l),(i,j))}.
\]
From this, $f(T)< f(S)$.   So $i(T)\prec i(S)$.


\case
$f(S)= 0$.

In this case, $W(S)\neq \emptyset$. 
Let $i^*$ maximize $|S_{i^*}|$ for all elements of $W(S)$.


\begin{subcases}
\subcase
 There is some $(k,l)\in\OffDiag(n)$
 such that $|S_{k}\cup S_l| = |S_{i^*}|$
 and $(k,l) < (i^*,i^*)$.
 Note that $i^*$ must be different from $k$ and $l$
 (and it is possible that $k =l$).
 
 Enumerate the pairs $(k,l)\in\OffDiag$ with
 the properties in the last paragraph by
 \[ (k_1,l_1), \ldots, (k_c, l_c)\]
 Let \[ \begin{array}{lcl}
 T_1  & = &  \Clamp(S,k_1,l_1,1)\\
T_2 & = & \Clamp(T_1,k_2,l_2, 1)\\
  & \vdots   & \\
T_c & = & \Clamp(T_{c-1},k_{c},l_{c}, 1)\\
\end{array}
\]
 Check that for $d, e \in [c]$,
 $|T_{k_d}| = |T_{k_e}| $.
 From this, it follows easily that 
 $f(T) = f(S)$.  And since
 we are in Case 2, $f(T) = f(S)= 0$.
 For all $x\in[n]$, $|T_x| = |S_x|+1$, except for $x = k$
 and $x = l$. 
 Note that $|S_k|$ and $|S_l|$ are both $< |S_{k}\cup S_l| = |S_{i^*}|$.
 It follows that $g(T) = g(S)$.
 But $h(T) < h(S)$, since $h(S)$ contains $(k,l)$
 but $h(T)$ does not.
 
 From all of these, $i(T) = (f(T),g(T),h(T)) \prec 
 (f(S),g(S),h(S)) = i(S)$.
 
 \bigskip
 
 
 \subcase
 There is some $(k,l)\in\Pairs(n)$
 such that $|S_{k}\cup S_l| = |S_{i^*}|$
 and $(i^*,i^*)< (k,l)$.
 Note again that $i^*$ must be different from $k$ and $l$.
 
 This time we take $T$ to be $\Clamp(S,i^*,i^*,1)$.
 Just as above, we have
  $i(T)\prec i(S)$.
 
 \bigskip
 
 
 \subcase
 There is some $(k,l)\in\Pairs(n)$
 such that $(i^*,i^*)< (k,l)$
 and $|S_k\cup S_l| \leq |S_{i^*}|$.
 
\end{subcases}
\end{mycases}


\end{document}

