\documentclass[12pt]{article}  
\usepackage{amssymb,amsthm,amsmath}
\usepackage{lscape}

% Packages Caleb added: %%%
\usepackage{xcolor}
\usepackage{stmaryrd}
\usepackage{comment}
\usepackage{soul}
%%%%%%%%%%%%%%%%%%%%%%%%%%%

\usepackage{bigstrut}
%\usepackage{MnSymbol}
\usepackage{bbm}
\usepackage{proof}
\usepackage{bussproofs}
\usepackage{tikz}
\usepackage{lingmacros}


\usepackage{hyperref}
\hypersetup{
    colorlinks,
    citecolor=black,
    filecolor=black,
    linkcolor=black,
    urlcolor=black
}

\newcommand{\more}{\mbox{\sf More}}
\newcommand{\existsgeq}{\mbox{\sf AtLeast}}
\newcommand{\Pol}{\mbox{\emph{Pol}}}
  \newcommand{\nonered}{\textcolor{red}{=}}
  \newcommand{\equalsred}{\nonered}
  \newcommand{\redstar}{\textcolor{red}{\star}}
    \newcommand{\dred}{\textcolor{red}{d}}
    \newcommand{\dmark}{\dred}
    \newcommand{\redflip}{\textcolor{red}{flip}}
        \newcommand{\flipdred}{\textcolor{red}{\mbox{\scriptsize \em flip}\ d}}
        \newcommand{\mdred}{\textcolor{blue}{m}\textcolor{red}{d}}
        \newcommand{\ndred}{\textcolor{blue}{n}\textcolor{red}{d}}
\newcommand{\arrowm}{\overset{\textcolor{blue}{m}}{\rightarrow} }
\newcommand{\arrown}{\overset{\textcolor{blue}{n}}{\rightarrow} }
\newcommand{\arrowmn}{\overset{\textcolor{blue}{mn}}{\longrightarrow} }
\newcommand{\arrowmonemtwo}{\overset{\textcolor{blue}{m_1 m_2}}{\longrightarrow} }
\newcommand{\bluen}{\textcolor{blue}{n}}
\newcommand{\bluem}{\textcolor{blue}{m}}
\newcommand{\bluemone}{\textcolor{blue}{m_1}}
\newcommand{\bluemtwo}{\textcolor{blue}{m_2}}
\newcommand{\blueminus}{\textcolor{blue}{-}}

\newcommand{\bluedot}{\textcolor{blue}{\cdot}}
\newcommand{\bluepm}{\textcolor{blue}{\pm}}
\newcommand{\blueplus}{\textcolor{blue}{+ }}
\newcommand{\translate}[1]{{#1}^{tr}}
\newcommand{\Caba}{\mbox{\sf Caba}} 
\newcommand{\Set}{\mbox{\sf Set}} 
\newcommand{\Pre}{\mbox{\sf Pre}} 
\newcommand{\wmarkpolarity}{\scriptsize{\mbox{\sf W}}}
\newcommand{\wmarkmarking}{\scriptsize{\mbox{\sf Mon}}}
\newcommand{\smark}{\scriptsize{\mbox{\sf S}}}
\newcommand{\bmark}{\scriptsize{\mbox{\sf B}}}
\newcommand{\mmark}{\scriptsize{\mbox{\sf M}}}
\newcommand{\jmark}{\scriptsize{\mbox{\sf J}}}
\newcommand{\kmark}{\scriptsize{\mbox{\sf K}}}
\newcommand{\tmark}{\scriptsize{\mbox{\sf T}}}
\newcommand{\greatermark}{\mbox{\tiny $>$}}
\newcommand{\lessermark}{\mbox{\tiny $<$}}
%%{\mbox{\ensuremath{>}}}
\newcommand{\true}{\top}
\newcommand{\false}{\bot}
\newcommand{\upred}{\textcolor{red}{\uparrow}}
\newcommand{\downred}{\textcolor{red}{\downarrow}}
\usepackage[all,cmtip]{xy}
\usepackage{enumitem}
\usepackage{fullpage}
\usepackage[authoryear]{natbib}
\usepackage{multicol}
\theoremstyle{definition}
\newtheorem{definition}{Definition}
\newtheorem{theorem}{Theorem}
\newtheorem{lemma}[theorem]{Lemma}
\newtheorem{conjecture}{Conjecture}
%\theoremstyle{case}


\newcounter{cases}
\newcounter{subcases}
\newenvironment{mycases}
  {%
    \setcounter{cases}{0}%
    \def\case
      {\bigskip
        \par\noindent
        \refstepcounter{cases}%
        \textbf{Case \thecases\ }
      }%
  }
  {%
    \par
  }
\newenvironment{subcases}
  {
    \setcounter{subcases}{0}%
    \def\subcase
      {\bigskip
        \par\noindent
        \refstepcounter{subcases}%
        \textbf{Subcase \thesubcases\ }
      }%
  }
  {%
  }
\renewcommand*\thecases{\arabic{cases}}
\renewcommand*\thesubcases{\roman{subcases}}


\newtheorem{claim}{Claim}
\newtheorem{corollary}{Corollary}
%\newtheorem{theorem}{Theorem}
\newtheorem{proposition}{Proposition}
\newtheorem{example}{Example}
\newtheorem{remark}[theorem]{Remark}
\newcommand{\semantics}[1]{[\![\mbox{\em $ #1 $\/}]\!]}
\newcommand{\abovearrow}[1]{\rightarrow\hspace{-.14in}\raiseonebox{1.0ex}
{$\scriptscriptstyle{#1}$}\hspace{.13in}}
\newcommand{\toplus}{\abovearrow{r}}
\newcommand{\tominus}{\abovearrow{i}} 
\newcommand{\todestroy}{\abovearrow{d}}
\newcommand{\tom}{\abovearrow{m}}
\newcommand{\tomprime}{\abovearrow{m'}}
\newcommand{\A}{\textsf{App}}
\newcommand{\At}{\textsf{At}}
\newcommand{\Emb}{\textsf{Emb}}
\newcommand{\EE}{\mathbb{E}}
\newcommand{\DD}{\mathbb{D}}
\newcommand{\PP}{\mathbb{P}}
\newcommand{\QQ}{\mathbb{Q}}
\newcommand{\LL}{\mathbb{L}}
\newcommand{\MM}{\mathbb{M}}
\usepackage{verbatim}
\newcommand{\TT}{\mathcal{T}}
\newcommand{\Marking}{\mbox{Mar}}
\newcommand{\Markings}{\Marking}
\newcommand{\Mar}{\Marking}
\newcommand{\Model}{\mathcal{M}}
\renewcommand{\SS}{\mathcal{S}}
\newcommand{\TTM}{\TT_{\Markings}}
\newcommand{\CC}{\mathbb{C}}
\newcommand{\erase}{\mbox{\textsf{erase}}}
\newcommand{\set}[1]{\{ #1 \}}
\newcommand{\arrowplus}{\overset{\blueplus}{\rightarrow} }
\newcommand{\arrowminus}{\overset{\blueminus}{\rightarrow} }
\newcommand{\arrowdot}{\overset{\bluedot}{\rightarrow} }
\newcommand{\arrowboth}{\overset{\bluepm}{\rightarrow} }
\newcommand{\arrowpm}{\arrowboth}
\newcommand{\arrowplusminus}{\arrowboth}
\newcommand{\arrowmone}{\overset{m_1}{\rightarrow} }
\newcommand{\arrowmtwo}{\overset{m_2}{\rightarrow} }
\newcommand{\arrowmthree}{\overset{m_3}{\rightarrow} }
\newcommand{\arrowmcomplex}{\overset{m_1 \orr m_2}{\longrightarrow} }
\newcommand{\arrowmproduct}{\overset{m_1 \cdot m_2}{\longrightarrow} }
\newcommand{\proves}{\vdash}
\newcommand{\Dual}{\mbox{\sc dual}}
\newcommand{\orr}{\vee}
\newcommand{\uar}{\uparrow}
\newcommand{\dar}{\downarrow}
\newcommand{\andd}{\wedge}
\newcommand{\bigandd}{\bigwedge}
\newcommand{\arrowmprime}{\overset{m'}{\rightarrow} }
\newcommand{\quadiff}{\quad \mbox{ iff } \quad}
\newcommand{\Con}{\mbox{\sf Con}}
\newcommand{\type}{\mbox{\sf type}}
\newcommand{\lang}{\mathcal{L}}
\newcommand{\necc}{\Box}
\newcommand{\vocab}{\mathcal{V}}
\newcommand{\wocab}{\mathcal{W}}
\newcommand{\Types}{\mathcal{T}_\mathcal{M}}
\newcommand{\mon}{\mbox{\sf mon}}
\newcommand{\anti}{\mbox{\sf anti}}
\newcommand{\FF}{\mathcal{F}}
\newcommand{\rem}[1]{\relax}


\newcommand{\raiseone}{\mbox{raise}^1}
\newcommand{\raisetwo}{\mbox{raise}^2}
\newcommand{\wrapper}[1]{{#1}}
\newcommand{\sfa}{\wrapper{\mbox{\sf a}}}
\newcommand{\sfb}{\wrapper{\mbox{\sf b}}}
\newcommand{\sfv}{\wrapper{\mbox{\sf v}}}
\newcommand{\sfw}{\wrapper{\mbox{\sf w}}}
\newcommand{\sfx}{\wrapper{\mbox{\sf x}}}
\newcommand{\sfy}{\wrapper{\mbox{\sf y}}}
\newcommand{\sfz}{\wrapper{\mbox{\sf z}}}
  \newcommand{\sff}{\wrapper{\mbox{\sf f}}}
    \newcommand{\sft}{\wrapper{\mbox{\sf t}}}
      \newcommand{\sfc}{\wrapper{\mbox{\sf c}}}
      \newcommand{\sfu}{\wrapper{\mbox{\sf u}}}
            \newcommand{\sfs}{\wrapper{\mbox{\sf s}}}
  \newcommand{\sfg}{\wrapper{\mbox{\sf g}}}

\newcommand{\sfvomits}{\wrapper{\mbox{\sf vomits}}}
\newlength{\mathfrwidth}
  \setlength{\mathfrwidth}{\textwidth}
  \addtolength{\mathfrwidth}{-2\fboxrule}
  \addtolength{\mathfrwidth}{-2\fboxsep}
\newsavebox{\mathfrbox}
\newenvironment{mathframe}
    {\begin{lrbox}{\mathfrbox}\begin{minipage}{\mathfrwidth}\begin{center}}
    {\end{center}\end{minipage}\end{lrbox}\noindent\fbox{\usebox{\mathfrbox}}}
    \newenvironment{mathframenocenter}
    {\begin{lrbox}{\mathfrbox}\begin{minipage}{\mathfrwidth}}
    {\end{minipage}\end{lrbox}\noindent\fbox{\usebox{\mathfrbox}}} 
 \renewcommand{\hat}{\widehat}
 \newcommand{\nott}{\neg}
  \newcommand{\preorderO}{\mathbb{O}}
 \newcommand{\PreorderP}{\mathbb{P}}
  \newcommand{\preorderE}{\mathbb{E}}
\newcommand{\preorderP}{\mathbb{P}}
\newcommand{\preorderN}{\mathbb{N}}
\newcommand{\preorderQ}{\mathbb{Q}}
\newcommand{\preorderX}{\mathbb{X}}
\newcommand{\preorderA}{\mathbb{A}}
\newcommand{\preorderR}{\mathbb{R}}
\newcommand{\preorderOm}{\mathbb{O}^{\bluem}}
\newcommand{\preorderPm}{\mathbb{P}^{\bluem}}
\newcommand{\preorderQm}{\mathbb{Q}^{\bluem}}
\newcommand{\preorderOn}{\mathbb{O}^{\bluen}}
\newcommand{\preorderPn}{\mathbb{P}^{\bluen}}
\newcommand{\preorderQn}{\mathbb{Q}^{\bluen}}
 \newcommand{\PreorderPop}{\mathbb{P}^{\blueminus}}
  \newcommand{\preorderEop}{\mathbb{E}^{\blueminus}}
\newcommand{\preorderPop}{\mathbb{P}^{\blueminus}}
\newcommand{\preorderNop}{\mathbb{N}^{\blueminus}}
\newcommand{\preorderQop}{\mathbb{Q}^{\blueminus}}
\newcommand{\preorderXop}{\mathbb{X}^{\blueminus}}
\newcommand{\preorderAop}{\mathbb{A}^{\blueminus}}
\newcommand{\preorderRop}{\mathbb{R}^{\blueminus}}
 \newcommand{\PreorderPflat}{\mathbb{P}^{\flat}}
  \newcommand{\preorderEflat}{\mathbb{E}^{\flat}}
\newcommand{\preorderPflat}{\mathbb{P}^{\flat}}
\newcommand{\preorderNflat}{\mathbb{N}^{\flat}}
\newcommand{\preorderQflat}{\mathbb{Q}^{\flat}}
\newcommand{\preorderXflat}{\mathbb{X}^{\flat}}
\newcommand{\preorderAflat}{\mathbb{A}^{\flat}}
\newcommand{\preorderRflat}{\mathbb{R}^{\flat}}
\newcommand{\pstar}{\preorderBool^{\preorderBool^{E}}}
\newcommand{\pstarplus}{(\pstar)^{\blueplus}}
\newcommand{\pstarminus}{(\pstar)^{\blueminus}}
\newcommand{\pstarm}{(\pstar)^{\bluem}}
\newcommand{\Reals}{\preorderR}
\newcommand{\preorderS}{\mathbb{S}}
\newcommand{\preorderBool}{\mathbbm{2}}
 \renewcommand{\o}{\cdot}
 \newcommand{\NP}{\mbox{\sc np}}
 \newcommand{\NPplus}{\NP^{\blueplus}}
  \newcommand{\NPminus}{\NP^{\blueminus}}
   \newcommand{\NPplain}{\NP}
    \newcommand{\npplus}{np^{\blueplus}}
  \newcommand{\npminus}{np^{\blueminus}}
   \newcommand{\npplain}{np}
   \newcommand{\np}{np}
   \newcommand{\Term}{\mbox{\sc t}}
  \newcommand{\N}{\mbox{\sc n}}
   \newcommand{\X}{\mbox{\sc x}}
      \newcommand{\Y}{\mbox{\sc y}}
            \newcommand{\V}{\mbox{\sc v}}
    \newcommand{\Nbar}{\overline{\mbox{\sc n}}}
    \newcommand{\Pow}{\mathcal{P}}
    \newcommand{\powcontravariant}{\mathcal{Q}}
    \newcommand{\Id}{\mbox{Id}}
    \newcommand{\pow}{\Pow}
   \newcommand{\Sent}{\mbox{\sc s}}
   \newcommand{\lookright}{\slash}
   \newcommand{\lookleft}{\backslash}
   \newcommand{\dettype}{(e \to t)\arrowminus ((e\to t)\arrowplus t)}
\newcommand{\ntype}{e \to t}
\newcommand{\etttype}{(e\to t)\arrowplus t}
\newcommand{\nptype}{(e\to t)\arrowplus t}
\newcommand{\verbtype}{TV}
\newcommand{\who}{\infer{(\nptype)\arrowplus ((\ntype)\arrowplus (\ntype))}{\mbox{who}}}
\newcommand{\iverbtype}{IV}
\newcommand{\Nprop}{\N_{\mbox{prop}}}
\newcommand{\VP}{{\mbox{\sc vp}}}
\newcommand{\CN}{{\mbox{\sc cn}}}
\newcommand{\Vintrans}{\mbox{\sc iv}}
\newcommand{\Vtrans}{\mbox{\sc tv}}
\newcommand{\Num}{\mbox{\sc num}}
%\newcommand{\S}{\mathbb{A}}
\newcommand{\Det}{\mbox{\sc det}}
\newcommand{\preorderB}{\mathbb{B}}
\newcommand{\simA}{\sim_A}
\newcommand{\simB}{\sim_B}
\newcommand{\polarizedtype}{\mbox{\sf poltype}}
\newcommand{\Diag}{\mbox{Diag}}
\newcommand{\OffDiag}{\mbox{Off-diag}}
\newcommand{\Pairs}{\mbox{Pairs}}
\newcommand{\Bad}{\mbox{Bad}}
\newcommand{\argmax}{\mbox{argmax}}
\newcommand{\Clamp}{\mbox{Clamp}}
\newcommand{\sClamp}{\mbox{subset-Clamp}}
\newcommand{\ordercanonical}{<_{\scriptstyle can}}
\newcommand{\lex}{\ordercanonical}
\newcommand{\lexcanonical}{\ordercanonical}
\newcommand{\precsubseteq}{\preceq_{\scriptsize subset}}
\newcommand{\approxsubset}{\approx_{\scriptsize subset}}

%%%%%%%%%%%%%%%%%%%%%%%%%%%%%%%%%%%%%%%%%%%%%%%%%%%%%%%%%%%%%%%%%%%%%%%%%%%%%
% I'm trying on different choices of symbols for our relations.
% 
% This choice is inspired by the "Syllogistic Logic with Cardinality Comparisons" paper.
% Here, 'p' could mean "pairs", but in the paper *should mean* "provable"
\newcommand{\provsub}{\subseteq_{\Gamma}}
\newcommand{\provle}{\le_{\Gamma}}
\newcommand{\provlt}{<_{\Gamma}}

\newcommand{\nprovle}{\nleq_{\Gamma}}
\newcommand{\provextended}{\preceq_{\Gamma}}
\newcommand{\provextendedstrict}{\prec_{\Gamma}}
\newcommand{\nprovextended}{\npreceq_{\Gamma}}

\newcommand{\proverule}{\textsc}


% Something I use for words in a 'code' font.
\definecolor{light-gray}{gray}{0.95}
\newcommand{\code}[1]{\colorbox{light-gray}{\texttt{#1}}}

%%%%%%%%%%%%%%%%%%%%%%%%%%%%%%%%%%%%%%%%%%%%%%%%%%%%%%%%%%%%%%%%%%%%%%%%%%%%%

\begin{document}

\section{Interpretation of the Lemma}

The purpose of these notes is to discuss the relevance of our combinatorial result (\textbf{Lemma 1}) to proving the completeness of the syllogistic logic involving `all', `some', `more than', and `at least'.  We offer a helpful interpretation of the conjecture that makes this connection clear.

\section{Statement of the Lemma}

Our (recently proved) lemma is the following:

\begin{lemma}\label{Combinatorial-Lemma}
    Let $<_0$ be a strict linear order of pairs $(i, j)$ of indices $1, \ldots, n$ satisfying the \textbf{Singleton Condition}, that is: $i \ne j \implies (i, i) <_0 (i, j)$.  Then there exist sets $S_1$, \ldots, $S_n$ such that:
    
    \begin{itemize}
        \item $|S_i \cup S_j| < |S_k \cup S_l|$ iff $(i, j) <_0 (k, l)$
    \end{itemize}
\end{lemma}

\section{The Interpretation}

Let $\Gamma$ be a set of sentences containing sentences $all(x, y)$, $some(x, y)$, $more(x, y)$, and $atleast(x, y)$, where $x$ and $y$ can be base nouns or noun unions.

In our completeness proof, we must show that:
\begin{enumerate}
    \item \label{sound} If $\Gamma \vdash \varphi$ then $\Gamma \vDash \varphi$, and
    \item \label{complete} If $\Gamma \vDash \varphi$ then $\Gamma \vdash \varphi$.
\end{enumerate}

(\ref{sound}) is easy; we just verify that none of our rules can produce sentences that do not follow from the premises, i.e. that all of our rules are sound.  As for (\ref{complete}), we suppose $\Gamma \vDash \varphi$, then \textit{construct a model for} $\Gamma$, and then subsequently make use of this model's structure in order to show that $\Gamma \vdash \varphi$.  Our Lemma \ref{Combinatorial-Lemma} relates specifically to the step of \textit{constructing a model for} $\Gamma$.

Here is the interpretation: Our strict linear order $<_0$ corresponds to the order imposed by $more(A \cup B, C \cup D)$ sentences in $\Gamma$.  The claim that ``there exist $S_1, \ldots, S_n$" corresponds to ``there exists a model with nouns $S_1, \ldots, S_n$."  Finally, the condition $|S_i \cup S_j| < |S_k \cup S_l|$ iff $(i, j) <_0 (k, l)$ corresponds to the interpretation: $|\llbracket S_i \cup S_j \rrbracket| < |\llbracket S_k \cup S_l \rrbracket|$ iff the $more(A \cup B, C \cup D)$ sentences require that $S_i \cup S_j$ comes before $S_k \cup S_l$ (i.e. that this supposed
model is \textit{actually} a model of $\Gamma$).

We will explicitly point out that pairs $(i, i)$ in our ordering $<_0$ are meant to represent singleton nouns $S_i$ in our model (i.e. unions $S_i \cup S_i$).  All the \textbf{Singleton Condition} says is that our ordering $<_0$ never imposes $(i, i) <_0 (i, j)$, i.e. our set $\Gamma$ never imposes $more(S_i \cup S_j, S_i)$ (where, of course, $i \ne j$).

Following this interpretation, our lemma can be restated as:

\begin{lemma}\label{Reinterpreted-Comb-Lemma}
    Suppose $\Gamma$ is a set of sentences of the type $more(x, y)$,
    with $x$ and $y$ possibly involving noun unions.
    Suppose furthermore that for all nouns $a \ne b$, 
    $\Gamma \nvDash more(a, a \cup b)$ (this is part of the assumption
    that $\Gamma$ is \textit{consistent}).  Then there exists a model
    of $\Gamma$ with nouns $S_1, \ldots, S_n$.
\end{lemma}

This is almost exactly the lemma we will need for direction 
\ref{complete} of our completeness proof.  The only issue is that this
lemma only applies to $\Gamma$ with $more(x, y)$ sentences in it, and
says nothing about modeling $all(x, y), some(x, y), atleast(x, y) \in \Gamma$.

\section{How the lemma gets used in a completeness theorem}

We consider the logic of $\more(t,u)$, where $t$ and $u$ are terms (possibly union terms).
The logic that we use is transitivity of $\more(t,u)$, and also $\more(x\cup y, x)$ for $x$ and $y$
basic nouns.  Further, we have the contradiction rule:
from $\Gamma\proves\more(t,t)$, derive anything.
We write $\Gamma\proves\more(t,u)$
for the evident provability relation.  

Recall that an \emph{antichain model} is one where for all basic $x$ and $y$, if $x$ and $y$ are different,
then $\semantics{x} \not\subseteq \semantics{y}$.
We also write $\Gamma\models\more(t,u)$
to mean that for all {antichain models} $\Model$, if $\Model$ satisfies every sentence in $\Gamma$,
then it also satisfies $\more(t,u)$.

\begin{proposition}
$\Gamma\proves \more(t,u)$ iff $\Gamma\models\more(t,u)$.
\end{proposition}

\begin{proof}
The only non-trivial step is to show that if 
$\Gamma\not\proves \more(t,u)$, then there is a model of $\Gamma$
where $\more(t,u)$ is false.

Write $<$ for the order on basic or union nouns determined by provability from
$\Gamma$:  $i < j$ iff $\Gamma\proves \more(j,i)$.
This order is transitive and irreflexive.   (For if $\Gamma\proves\more(i,i)$,
then by the contradiction rule,  we have $\Gamma\proves \more(t,u)$, 
contrary to our assumption.)

Recall Lemma 3.4 of ``Syllogistic Logic with Cardinality Comparisions'' 
(or any other source on this): if $(T, <)$ transitive irreflexive relation
on a finite set,
and if $x \not{<} y$ in $T$, then there is a listing of $T$ as 
\[ t_1, \ldots, t_n \]
Such that 
\begin{enumerate}
    \item For $i < j$ in the listing $t_i < t_j$ in $T$
    \item $y$ is listed before $x$.
\end{enumerate}
In effect, we linearize $T$.

We apply this result to the order $<$ on 
basic and union terms determined from $\Gamma$, 
and incorporating the extra condition that 
$\Gamma\not\proves \more(t,u)$.
We then apply Lemma 1.  (But that lemma also should be strengthened
to say that the sets $S_i$ are an antichain.)
\end{proof}


\section{Future TODO}

We would like our final result to look something like:

\begin{conjecture}
    Let $<_0$, $\le_0$, $\subseteq_0$ be linear orders (strict, nonstrict, nonstrict, respectively) of pairs $(i, j)$ of indices. $1, \ldots, n$ satisfying some \textbf{Consistency Conditions}.  Then there exist sets $S_1$, \ldots, $S_n$ such that:
    
    \begin{itemize}
        \item $|S_i \cup S_j| < |S_k \cup S_l|$ iff $(i, j) <_0 (k, l)$,
        \item $|S_i \cup S_j| \le |S_k \cup S_l|$ iff $(i, j) \le_0 (k, l)$, and
        \item $|S_i \cup S_j| \subseteq |S_k \cup S_l|$ iff $(i, j) \subseteq_0 (k, l)$
    \end{itemize}
\end{conjecture}

where our \textbf{Consistency Conditions} are conditions similar to our \textbf{Singleton Condition} that amount in our interpretation to the claim that $\Gamma$ is consistent.  That is, the conditions will state constraints on $<_0$, $\le_0$, and $\subseteq_0$ (and interactions between these orderings) so that we do not obtain nonsense pair orderings.


\begin{section}{Definition of `Suitable Relations' (Reminder)}
The construction we use to build a model for our completeness theorem for our logic requires a \textit{suitable pair} of relations over $\Pairs(n)$.
For convenience, here is the definition of a suitable pair taken from \code{thoughts.tex}:

\begin{definition}
A \emph{suitable pair of relations on $\Pairs(n)$} is a pair of relations
\[ (\le, \subseteq) \]
such that 
\begin{enumerate}
\item $\le$ and $\subseteq$ are preorders on $\Pairs(n)$.
\item  
 For all $(i,j), (k,l)\in \Pairs(n)$,
 either $(i,j) \le (k,l)$ or $(k,l) < (i,j)$, where
 \[ x < y \quadiff x \le y \mbox{ but not } y \le x\]
 \item If $i < j$, then $(i,i) \subseteq (i,j)$.  If $j < i$, then $(i,i) \subseteq (j,i)$. 
 \item If $(i,i) \subseteq (k,l)$ and $(j,j) \subseteq (k,l)$ and $i < j$,
 then $(i,j) \subseteq (k,l)$.
\item If $(i,j) \subseteq (k,l)$, then $(i,j) \le (k,l)$.
\item If $(i,j) \subseteq (k,l)$ and $(k,l) \le (i,j)$, then $(k,l) \subseteq (i,j)$.
\end{enumerate}
\label{def-suitable-pair}
\end{definition}

% \[ (i, j) \provle (k, l) \quadiff \Gamma \vdash atleast(k \cup l, i \cup j)\]
% \[ (i, j) \provlt (k, l) \quadiff \Gamma \vdash more(k \cup l, i \cup j)\]
% \[ (i, j) \provsub (k, l) \quadiff \Gamma \vdash all(i \cup j, k \cup l)\]
\end{section}

\begin{section}{Verifying that our Interpretation Works}

\hl{TODO:} What do we do about $\varphi$?  Do we need to ensure anything about its falseness in our model before the extension?  This will be dealt with in the actual completeness proof.

The rules of our logic include those rules from ``Syllogistic Logic with Cardinality Comparisons'', excluding rules involving $some(x, y)$ and noun complement.  In addition, we have the rules displayed in Figure \ref{fig-all-unions}

In order to apply the Clamping Construction to build our model (in the completeness proof for this logic), we need to construct orderings $\provextended, \provsub$ that (1) preserve those sentences provable from $\Gamma$, (2) ensure the particular sentence $\varphi$ is false, and (3) form a `suitable' pair.

We first define:

\begin{itemize}
    \item $(i, j) \provle (k, l)$ iff $\Gamma \vdash atleast(k \cup l, i \cup j)$
    
    \item $(i, j) \provsub (k, l)$ iff $\Gamma \vdash all(i \cup j, k \cup l)$
\end{itemize}
% We would like to extend $\provle$ to a linear ordering $\provextended$ over $\Pairs(n)$.

\begin{lemma}
    We can extend $\provle$ to a \textit{linear} ordering $\provextended$ over $\Pairs(n)$ that preserves those relevant sentences provable from $\Gamma$, i.e.
    
    \begin{enumerate}
        \item If $\Gamma \vdash atleast(k \cup l, i \cup j)$, then $(i,j) \provextended (k,l)$, and
        
        \item If $\Gamma \vdash more(k \cup l, i \cup j)$, then $(i,j) \provextendedstrict (k,l)$, where
        \[ x \provextendedstrict y \quadiff x \provextended y \mbox{ but } y \nprovextended x\]
    \end{enumerate}
\end{lemma}

\begin{proof}
To construct $\provextended$, we simply invoke the fact that any partial order can be extended to a linear order (extending $\provle$ to $\provextended$).
We easily check (1): If $\Gamma \vdash atleast(k \cup l, i \cup j)$, then $(i,j) \provle (k,l)$, and by extension $(i,j) \provextended (k,l)$.

Regarding (2): If $\Gamma \vdash more(k \cup l, i \cup j)$, since $\Gamma$ is consistent (and using $\proverule{(x)}$), $\Gamma \nvdash atleast(i \cup j, k \cup l)$.  So $(k,l) \nprovle (i,j)$.  
Since $(i,j)$ and $(k,l)$ were comparable before the extension, we have $(k,l) \nprovextended (i,j)$.  In addition, since $\Gamma \vdash more(k \cup l, i \cup j)$, we have $\Gamma \vdash atleast(k \cup l, i \cup j)$ (by $\proverule{(More At Least)}$) and hence $(i,j) \provextended (k,l)$.  By definition of $\provextendedstrict$, this gives us $(i,j) \provextendedstrict (k,l)$.

\end{proof}

We now need to show that $(\provextended, \provsub)$ is, in fact, a suitable pair. This will allow us to apply our Clamping construction from before.  

\begin{lemma}
    Let $\Gamma$ be a set of sentences involving $atleast(x, y)$, $more(x, y)$, and $all(x, y)$, where $x, y$ are union terms.  Then $(\provextended, \provsub)$ is a suitable pair.
    
\end{lemma}

\begin{proof}


% We show each implication:
% \paragraph{$(\longrightarrow)$} Suppose $(i, j) \provlt (k, l)$.  So $\Gamma \vdash more(k \cup l, i \cup j)$.  By ($\proverule{More At Least}$), $\Gamma \vdash atleast(k \cup l, i \cup j)$.  So $(i, j) \provle (k, l)$.  Additionally, using ($\proverule{x}$), since $\Gamma$ is consistent and $\Gamma vdash more(k \cup l, i \cup j)$, $\Gamma \nvdash atleast(i \cup j, k \cup l)$.  So it is not the case that $(k, l) \provle (i, j)$
% \paragraph{$(\longleftarrow)$} TODO
 

We show that $(\provextended, \provsub)$ satisfy each of the 6 properties in turn:

\begin{enumerate}
    \item We must show that $\provextended$ and $\provsub$ are both reflexive and transitive.  $\provsub$ is reflexive by ($\proverule{Axiom}$), and is transitive by ($\proverule{Barbara}$).  $\provextended$ is reflexive and transitive simply because it was constructed to be a linear ordering.
    
    % OLD: $\provle$ is reflexive by application of ($\proverule{Axiom}$) and ($\proverule{Subset  Size}$).  $\provle$ is transitive by ($\proverule{Card Trans}$)
    
    \item $\provextended$ is a linear ordering, and hence is trichotomous.  So for any $(i, j), (k, l) \in \Pairs(n)$, either $(i,j) \provextended (k,l)$ or $(k,l) \provextendedstrict (i,j)$.
    
    \item Suppose $i < j$.  Well, $\Gamma \vdash all(i, i \cup j)$ by ($\proverule{Union Extend}$).  So $(i, i) \provsub (i, j)$.  Similarly, if $j < i$ we may use ($\proverule{Union Extend}$) again to obtain $(i, i) \provsub (j, i)$.
    
    \item Suppose $(i, i) \provsub (k, l)$, $(j, j) \provsub (k, l)$, and $i < j$.  So $\Gamma \vdash all(i, k \cup l)$ and $\Gamma \vdash all(j, k \cup l)$.  So by ($\proverule{Union All}$), $\Gamma \vdash all(i \cup j, k \cup l)$.  So $(i, j) \provsub (k, l)$.
    
    \item Suppose $(i, j) \provsub (k, l)$.  Then $\Gamma \vdash all(i \cup j, k \cup l)$.  By ($\proverule{Subset Size}$), $\Gamma \vdash atleast(k \cup l, i \cup j)$, i.e. $(i, j) \provextended (k, l)$.
    
    \item Suppose $(i, j) \provsub (k, l)$ and $(k, l) \provextended (i, j)$.  So $\Gamma \vdash all(i \cup j, k \cup l)$.  By ($\proverule{Subset Size}$), $\Gamma \vdash atleast(k \cup l, i \cup j)$.  This means that $(i,j)$ and $(k,l)$ were comparable before the extension!  So $(k,l) \provextended (i,j) \implies (k,l) \provle (i,j)$.  This means that $\Gamma \vdash atleast(i \cup j, k \cup l)$.  So by ($\proverule{Card Trans}$) we may conclude that $\Gamma \vdash all(i \cup j, k \cup l)$.
    
\end{enumerate}



\end{proof}

\end{section}

\begin{figure}[t]
\begin{mathframe}
\[
\begin{array}{l@{\qquad}l@{\qquad}l}
\infer[(\proverule{union-idemp})]{\mbox{\sf All ($x\cup x$) $x$}}{} &
\infer[(\proverule{union-extend})]{\mbox{\sf All $x$ ($x\cup y$) }}{}   \\  \\
\infer[(\proverule{union-symm})]{\mbox{\sf All ($y \cup x$) ($x\cup y$) }}{} &
\infer[(\proverule{union-all})]{\mbox{\sf All ($x\cup y$) $t$}}{\mbox{\sf All $x$ $t$} & \mbox{\sf All $y$ $t$}}
\end{array}
\]
\caption{The additional rules for our logic with unions
\label{fig-all-unions}}
\end{mathframe}
\end{figure}

\end{document}