\def\year{2020}\relax
%File: formatting-instruction.tex
\documentclass[letterpaper]{article} % DO NOT CHANGE THIS
\usepackage{aaai20}  % DO NOT CHANGE THIS
\usepackage{times}  % DO NOT CHANGE THIS
\usepackage{helvet} % DO NOT CHANGE THIS
\usepackage{courier}  % DO NOT CHANGE THIS
\usepackage[hyphens]{url}  % DO NOT CHANGE THIS
\usepackage{graphicx} % DO NOT CHANGE THIS
\urlstyle{rm} % DO NOT CHANGE THIS
\def\UrlFont{\rm}  % DO NOT CHANGE THIS
\usepackage{graphicx}  % DO NOT CHANGE THIS
\frenchspacing  % DO NOT CHANGE THIS
\setlength{\pdfpagewidth}{8.5in}  % DO NOT CHANGE THIS
\setlength{\pdfpageheight}{11in}  % DO NOT CHANGE THIS
% \nocopyright
%PDF Info Is REQUIRED.
% For /Author, add all authors within the parentheses, separated by commas. No accents or commands.
% For /Title, add Title in Mixed Case. No accents or commands. Retain the parentheses.

% REMOVED FOR BLIND REVIEW
%  \pdfinfo{
% /Title (Logics for Sizes with Union or Intersection)
% /Author (Caleb Kisby, Saul A. Blanco, Alex Kruckman, Lawrence S. Moss)
% } %Leave this	

% /Title ()
% Put your actual complete title (no codes, scripts, shortcuts, or LaTeX commands) within the parentheses in mixed case
% Leave the space between \Title and the beginning parenthesis alone
% /Author ()
% Put your actual complete list of authors (no codes, scripts, shortcuts, or LaTeX commands) within the parentheses in mixed case. 
% Each author should be only by a comma. If the name contains accents, remove them. If there are any LaTeX commands, 
% remove them. 

% DISALLOWED PACKAGES
% \usepackage{authblk} -- This package is specifically forbidden
% \usepackage{balance} -- This package is specifically forbidden
% \usepackage{caption} -- This package is specifically forbidden
% \usepackage{color (if used in text)
% \usepackage{CJK} -- This package is specifically forbidden
% \usepackage{float} -- This package is specifically forbidden
% \usepackage{flushend} -- This package is specifically forbidden
% \usepackage{fontenc} -- This package is specifically forbidden
% \usepackage{fullpage} -- This package is specifically forbidden
% \usepackage{geometry} -- This package is specifically forbidden
% \usepackage{grffile} -- This package is specifically forbidden
% \usepackage{hyperref} -- This package is specifically forbidden
% \usepackage{navigator} -- This package is specifically forbidden
% (or any other package that embeds links such as navigator or hyperref)
% \indentfirst} -- This package is specifically forbidden
% \layout} -- This package is specifically forbidden
% \multicol} -- This package is specifically forbidden
% \nameref} -- This package is specifically forbidden
% \natbib} -- This package is specifically forbidden -- use the following workaround:
% \usepackage{savetrees} -- This package is specifically forbidden
% \usepackage{setspace} -- This package is specifically forbidden
% \usepackage{stfloats} -- This package is specifically forbidden
% \usepackage{tabu} -- This package is specifically forbidden
% \usepackage{titlesec} -- This package is specifically forbidden
% \usepackage{tocbibind} -- This package is specifically forbidden
% \usepackage{ulem} -- This package is specifically forbidden
% \usepackage{wrapfig} -- This package is specifically forbidden
% DISALLOWED COMMANDS
% \nocopyright -- Your paper will not be published if you use this command
% \addtolength -- This command may not be used
% \balance -- This command may not be used
% \baselinestretch -- Your paper will not be published if you use this command
% \clearpage -- No page breaks of any kind may be used for the final version of your paper
% \columnsep -- This command may not be used
% \newpage -- No page breaks of any kind may be used for the final version of your paper
% \pagebreak -- No page breaks of any kind may be used for the final version of your paperr
% \pagestyle -- This command may not be used
% \tiny -- This is not an acceptable font size.
% \vspace{- -- No negative value may be used in proximity of a caption, figure, table, section, subsection, subsubsection, or reference
% \vskip{- -- No negative value may be used to alter spacing above or below a caption, figure, table, section, subsection, subsubsection, or reference

\setcounter{secnumdepth}{1} %May be changed to 1 or 2 if section numbers are desired.

% The file aaai20.sty is the style file for AAAI Press 
% proceedings, working notes, and technical reports.
%
\setlength\titlebox{2.5in} % If your paper contains an overfull \vbox too high warning at the beginning of the document, use this
% command to correct it. You may not alter the value below 2.5 in
\title{Supplemental Material:\\Logics for Sizes with Union or Intersection}
%Your title must be in mixed case, not sentence case. 
% That means all verbs (including short verbs like be, is, using,and go), 
% nouns, adverbs, adjectives should be capitalized, including both words in hyphenated terms, while
% articles, conjunctions, and prepositions are lower case unless they
% directly follow a colon or long dash

% REMOVED FOR BLIND REVIEW
% \author{Caleb Kisby\textsuperscript{\rm 1},
%         Sa\'ul A.~Blanco\textsuperscript{\rm 1},
%         Alex Kruckman\textsuperscript{\rm 2}, and
%         Lawrence S.~Moss\textsuperscript{\rm 3}\\ 
%     % All authors must be in the same font size and format. Use \Large and \textbf to achieve this result when breaking a line
% \textsuperscript{1}
%     Department of Computer Science, Indiana University, Bloomington, IN 47408, USA \\
%     cckisby@indiana.edu, sblancor@indiana.edu \\
% \textsuperscript{2}
%     Department of Mathematics and Computer Science, Wesleyan University, Middletown, CT 06459, USA\\
%     akruckman@wesleyan.edu \\
% \textsuperscript{3}
%     Department of Mathematics, Indiana University, Bloomington, IN 47405, USA \\
%     lmoss@indiana.edu \\
% }
    
%If you have multiple authors and multiple affiliations
% use superscripts in text and roman font to identify them. For example, Sunil Issar,\textsuperscript{\rm 2} J. Scott Penberthy\textsuperscript{\rm 3} George Ferguson,\textsuperscript{\rm 4} Hans Guesgen\textsuperscript{\rm 5}. Note that the comma should be placed BEFORE the superscript for optimum readability
% email address must be in roman text type, not monospace or sans serif


% \authorrunning{Example}
% \titlerunning{Logics for Sizes with Union or Intersection}



%%%%%%%%%%%%%%%%%%%%%%%%%%%%%%%%%%%%%%%%%%%%%%%%%%
% Additional Packages (Not required by AAAI)
%%%%%%%%%%%%%%%%%%%%%%%%%%%%%%%%%%%%%%%%%%%%%%%%%%

\usepackage{amssymb,amsthm,amsmath}
\usepackage{mathtools}
\usepackage{lscape}

% Packages Caleb added: %%%
\usepackage{xcolor}
\usepackage{stmaryrd}
\usepackage{comment}
\usepackage{soul}
\usepackage{enumerate} % Used by Alex
\usepackage[shortlabels]{enumitem}
\usepackage[mathscr]{eucal}
\usepackage{marginnote} % I couldn't get a simple marginpar to work for the life of me...
\usepackage{stackengine}
\usepackage{pdfpages}
\usepackage{xr}
%%%%%%%%%%%%%%%%%%%%%%%%%%%

\newcommand{\authnote}[2]{{#1}{#2}}
\newcommand{\snote}[1]{{\authnote{Sa\'ul: }{{\color{blue} #1}}}}

\newcommand{\cnote}[1]{{\authnote{Caleb: }{{\color{purple} #1}}}}
%%%%

\usepackage{bigstrut}
%\usepackage{MnSymbol}
\usepackage{bbm}
\usepackage{proof}
\usepackage{bussproofs}
\usepackage{tikz}
\usepackage{lingmacros}

% NOTE: FORBIDDEN (see AAAI guidelines above)
% \usepackage{hyperref}
% \hypersetup{
%     colorlinks,
%     citecolor=black,
%     filecolor=black,
%     linkcolor=black,
%     urlcolor=black
% }


\newcommand{\existsgeq}{\mbox{\sf AtLeast}}
\newcommand{\Pol}{\mbox{\emph{Pol}}}
  \newcommand{\nonered}{\textcolor{red}{=}}
  \newcommand{\equalsred}{\nonered}
  \newcommand{\redstar}{\textcolor{red}{\star}}
    \newcommand{\dred}{\textcolor{red}{d}}
    \newcommand{\dmark}{\dred}
    \newcommand{\redflip}{\textcolor{red}{flip}}
        \newcommand{\flipdred}{\textcolor{red}{\mbox{\scriptsize \em flip}\ d}}
        \newcommand{\mdred}{\textcolor{blue}{m}\textcolor{red}{d}}
        \newcommand{\ndred}{\textcolor{blue}{n}\textcolor{red}{d}}
\newcommand{\arrowm}{\overset{\textcolor{blue}{m}}{\rightarrow} }
\newcommand{\arrown}{\overset{\textcolor{blue}{n}}{\rightarrow} }
\newcommand{\arrowmn}{\overset{\textcolor{blue}{mn}}{\longrightarrow} }
\newcommand{\arrowmonemtwo}{\overset{\textcolor{blue}{m_1 m_2}}{\longrightarrow} }
\newcommand{\bluen}{\textcolor{blue}{n}}
\newcommand{\bluem}{\textcolor{blue}{m}}
\newcommand{\bluemone}{\textcolor{blue}{m_1}}
\newcommand{\bluemtwo}{\textcolor{blue}{m_2}}
\newcommand{\blueminus}{\textcolor{blue}{-}}

\newcommand{\bluedot}{\textcolor{blue}{\cdot}}
\newcommand{\bluepm}{\textcolor{blue}{\pm}}
\newcommand{\blueplus}{\textcolor{blue}{+ }}
\newcommand{\translate}[1]{{#1}^{tr}}
\newcommand{\Caba}{\mbox{\sf Caba}} 
\newcommand{\Set}{\mbox{\sf Set}} 
\newcommand{\Pre}{\mbox{\sf Pre}} 
\newcommand{\wmarkpolarity}{\scriptsize{\mbox{\sf W}}}
\newcommand{\wmarkmarking}{\scriptsize{\mbox{\sf Mon}}}
\newcommand{\smark}{\scriptsize{\mbox{\sf S}}}
\newcommand{\bmark}{\scriptsize{\mbox{\sf B}}}
\newcommand{\mmark}{\scriptsize{\mbox{\sf M}}}
\newcommand{\jmark}{\scriptsize{\mbox{\sf J}}}
\newcommand{\kmark}{\scriptsize{\mbox{\sf K}}}
\newcommand{\tmark}{\scriptsize{\mbox{\sf T}}}
\newcommand{\greatermark}{\mbox{\tiny $>$}}
\newcommand{\lessermark}{\mbox{\tiny $<$}}
%%{\mbox{\ensuremath{>}}}
\newcommand{\true}{\top}
\newcommand{\false}{\bot}
\newcommand{\upred}{\textcolor{red}{\uparrow}}
\newcommand{\downred}{\textcolor{red}{\downarrow}}
\usepackage[all,cmtip]{xy}
\usepackage{enumitem}
\usepackage{fullpage}
\usepackage[authoryear]{natbib}
\usepackage{multicol}

\newtheorem{theorem}{Theorem}[section]
\newtheorem{lemma}[theorem]{Lemma}
\newtheorem{claim}[theorem]{Claim}
\newtheorem{proposition}[theorem]{Proposition}
\newtheorem{corollary}[theorem]{Corollary}
%\newtheorem{theorem}{Theorem}

\theoremstyle{definition}
\newtheorem{example}[theorem]{Example}
\newtheorem{remark}[theorem]{Remark}
\newtheorem*{relatedwork*}{Related Work}
\newtheorem*{organization*}{Organization}
\newtheorem*{nextsteps*}{Next Steps}
\newtheorem{definition}[theorem]{Definition}

\newcommand{\semantics}[1]{[\![\mbox{\em $ #1 $\/}]\!]}
\newcommand{\abovearrow}[1]{\rightarrow\hspace{-.14in}\raiseonebox{1.0ex}
{$\scriptscriptstyle{#1}$}\hspace{.13in}}
\newcommand{\toplus}{\abovearrow{r}}
\newcommand{\tominus}{\abovearrow{i}} 
\newcommand{\todestroy}{\abovearrow{d}}
\newcommand{\tom}{\abovearrow{m}}
\newcommand{\tomprime}{\abovearrow{m'}}
\newcommand{\A}{\textsf{App}}
\newcommand{\At}{\textsf{At}}
\newcommand{\Emb}{\textsf{Emb}}
\newcommand{\EE}{\mathbb{E}}
\newcommand{\DD}{\mathbb{D}}
\newcommand{\PP}{\mathbb{P}}
\newcommand{\QQ}{\mathbb{Q}}
\newcommand{\LL}{\mathbb{L}}
\newcommand{\MM}{\mathbb{M}}
\usepackage{verbatim}
\newcommand{\TT}{\mathcal{T}}
\newcommand{\Marking}{\mbox{Mar}}
\newcommand{\Markings}{\Marking}
\newcommand{\Mar}{\Marking}
\newcommand{\Model}{\mathcal{M}}
\newcommand{\Nodel}{\mathcal{N}}
\renewcommand{\SS}{\mathcal{S}}
\newcommand{\TTM}{\TT_{\Markings}}
\newcommand{\CC}{\mathbb{C}}
\newcommand{\erase}{\mbox{\textsf{erase}}}
\newcommand{\set}[1]{\{ #1 \}}
\newcommand{\arrowplus}{\overset{\blueplus}{\rightarrow} }
\newcommand{\arrowminus}{\overset{\blueminus}{\rightarrow} }
\newcommand{\arrowdot}{\overset{\bluedot}{\rightarrow} }
\newcommand{\arrowboth}{\overset{\bluepm}{\rightarrow} }
\newcommand{\arrowpm}{\arrowboth}
\newcommand{\arrowplusminus}{\arrowboth}
\newcommand{\arrowmone}{\overset{m_1}{\rightarrow} }
\newcommand{\arrowmtwo}{\overset{m_2}{\rightarrow} }
\newcommand{\arrowmthree}{\overset{m_3}{\rightarrow} }
\newcommand{\arrowmcomplex}{\overset{m_1 \orr m_2}{\longrightarrow} }
\newcommand{\arrowmproduct}{\overset{m_1 \cdot m_2}{\longrightarrow} }
\newcommand{\proves}{\vdash}
\newcommand{\Dual}{\mbox{\sc dual}}
\newcommand{\orr}{\vee}
\newcommand{\uar}{\uparrow}
\newcommand{\dar}{\downarrow}
\newcommand{\andd}{\wedge}
\newcommand{\bigandd}{\bigwedge}
\newcommand{\arrowmprime}{\overset{m'}{\rightarrow} }
\newcommand{\quadiff}{\quad \mbox{ iff } \quad}
\newcommand{\Con}{\mbox{\sf Con}}
\newcommand{\type}{\mbox{\sf type}}
\newcommand{\lang}{\mathcal{L}}
\newcommand{\necc}{\Box}
\newcommand{\vocab}{\mathcal{V}}
\newcommand{\wocab}{\mathcal{W}}
\newcommand{\Types}{\mathcal{T}_\mathcal{M}}
\newcommand{\mon}{\mbox{\sf mon}}
\newcommand{\anti}{\mbox{\sf anti}}
\newcommand{\FF}{\mathcal{F}}
\newcommand{\rem}[1]{\relax}
\renewcommand{\phi}{\varphi}

\newcommand{\raiseone}{\mbox{raise}^1}
\newcommand{\raisetwo}{\mbox{raise}^2}
\newcommand{\wrapper}[1]{{#1}}
\newcommand{\sfa}{\wrapper{\mbox{\sf a}}}
\newcommand{\sfb}{\wrapper{\mbox{\sf b}}}
\newcommand{\sfv}{\wrapper{\mbox{\sf v}}}
\newcommand{\sfw}{\wrapper{\mbox{\sf w}}}
\newcommand{\sfx}{\wrapper{\mbox{\sf x}}}
\newcommand{\sfy}{\wrapper{\mbox{\sf y}}}
\newcommand{\sfz}{\wrapper{\mbox{\sf z}}}
  \newcommand{\sff}{\wrapper{\mbox{\sf f}}}
    \newcommand{\sft}{\wrapper{\mbox{\sf t}}}
      \newcommand{\sfc}{\wrapper{\mbox{\sf c}}}
      \newcommand{\sfu}{\wrapper{\mbox{\sf u}}}
            \newcommand{\sfs}{\wrapper{\mbox{\sf s}}}
  \newcommand{\sfg}{\wrapper{\mbox{\sf g}}}

\newcommand{\sfvomits}{\wrapper{\mbox{\sf vomits}}}
\newlength{\mathfrwidth}
  \setlength{\mathfrwidth}{\textwidth}
  \addtolength{\mathfrwidth}{-2\fboxrule}
  \addtolength{\mathfrwidth}{-2\fboxsep}
\newsavebox{\mathfrbox}
\newenvironment{mathframe}
    {\begin{lrbox}{\mathfrbox}\begin{minipage}{\mathfrwidth}\begin{center}}
    {\end{center}\end{minipage}\end{lrbox}\noindent\fbox{\usebox{\mathfrbox}}}
    \newenvironment{mathframenocenter}
    {\begin{lrbox}{\mathfrbox}\begin{minipage}{\mathfrwidth}}
    {\end{minipage}\end{lrbox}\noindent\fbox{\usebox{\mathfrbox}}} 
 \renewcommand{\hat}{\widehat}
 \newcommand{\nott}{\neg}
  \newcommand{\preorderO}{\mathbb{O}}
 \newcommand{\PreorderP}{\mathbb{P}}
  \newcommand{\preorderE}{\mathbb{E}}
\newcommand{\preorderP}{\mathbb{P}}
\newcommand{\preorderN}{\mathbb{N}}
\newcommand{\preorderQ}{\mathbb{Q}}
\newcommand{\preorderX}{\mathbb{X}}
\newcommand{\preorderA}{\mathbb{A}}
\newcommand{\preorderR}{\mathbb{R}}
\newcommand{\preorderOm}{\mathbb{O}^{\bluem}}
\newcommand{\preorderPm}{\mathbb{P}^{\bluem}}
\newcommand{\preorderQm}{\mathbb{Q}^{\bluem}}
\newcommand{\preorderOn}{\mathbb{O}^{\bluen}}
\newcommand{\preorderPn}{\mathbb{P}^{\bluen}}
\newcommand{\preorderQn}{\mathbb{Q}^{\bluen}}
 \newcommand{\PreorderPop}{\mathbb{P}^{\blueminus}}
  \newcommand{\preorderEop}{\mathbb{E}^{\blueminus}}
\newcommand{\preorderPop}{\mathbb{P}^{\blueminus}}
\newcommand{\preorderNop}{\mathbb{N}^{\blueminus}}
\newcommand{\preorderQop}{\mathbb{Q}^{\blueminus}}
\newcommand{\preorderXop}{\mathbb{X}^{\blueminus}}
\newcommand{\preorderAop}{\mathbb{A}^{\blueminus}}
\newcommand{\preorderRop}{\mathbb{R}^{\blueminus}}
 \newcommand{\PreorderPflat}{\mathbb{P}^{\flat}}
  \newcommand{\preorderEflat}{\mathbb{E}^{\flat}}
\newcommand{\preorderPflat}{\mathbb{P}^{\flat}}
\newcommand{\preorderNflat}{\mathbb{N}^{\flat}}
\newcommand{\preorderQflat}{\mathbb{Q}^{\flat}}
\newcommand{\preorderXflat}{\mathbb{X}^{\flat}}
\newcommand{\preorderAflat}{\mathbb{A}^{\flat}}
\newcommand{\preorderRflat}{\mathbb{R}^{\flat}}
\newcommand{\pstar}{\preorderBool^{\preorderBool^{E}}}
\newcommand{\pstarplus}{(\pstar)^{\blueplus}}
\newcommand{\pstarminus}{(\pstar)^{\blueminus}}
\newcommand{\pstarm}{(\pstar)^{\bluem}}
\newcommand{\Reals}{\preorderR}
\newcommand{\preorderS}{\mathbb{S}}
\newcommand{\preorderBool}{\mathbbm{2}}
 \renewcommand{\o}{\cdot}
% \newcommand{\NP}{\mbox{\sc np}}
 \newcommand{\NPplus}{\NP^{\blueplus}}
  \newcommand{\NPminus}{\NP^{\blueminus}}
   \newcommand{\NPplain}{\NP}
    \newcommand{\npplus}{np^{\blueplus}}
  \newcommand{\npminus}{np^{\blueminus}}
   \newcommand{\npplain}{np}
   \newcommand{\np}{np}
   \newcommand{\Term}{\mbox{\sc t}}
  \newcommand{\N}{\mbox{\sc n}}
   \newcommand{\X}{\mbox{\sc x}}
      \newcommand{\Y}{\mbox{\sc y}}
            \newcommand{\V}{\mbox{\sc v}}
    \newcommand{\Nbar}{\overline{\mbox{\sc n}}}
    \newcommand{\Pow}{\mathcal{P}}
    \newcommand{\powcontravariant}{\mathcal{Q}}
    \newcommand{\Id}{\mbox{Id}}
    \newcommand{\pow}{\Pow}
   \newcommand{\Sent}{\mbox{\sc s}}
   \newcommand{\lookright}{\slash}
   \newcommand{\lookleft}{\backslash}
   \newcommand{\dettype}{(e \to t)\arrowminus ((e\to t)\arrowplus t)}
\newcommand{\ntype}{e \to t}
\newcommand{\etttype}{(e\to t)\arrowplus t}
\newcommand{\nptype}{(e\to t)\arrowplus t}
\newcommand{\verbtype}{TV}
\newcommand{\who}{\infer{(\nptype)\arrowplus ((\ntype)\arrowplus (\ntype))}{\mbox{who}}}
\newcommand{\iverbtype}{IV}
\newcommand{\Nprop}{\N_{\mbox{prop}}}
\newcommand{\VP}{{\mbox{\sc vp}}}
\newcommand{\CN}{{\mbox{\sc cn}}}
\newcommand{\Vintrans}{\mbox{\sc iv}}
\newcommand{\Vtrans}{\mbox{\sc tv}}
\newcommand{\Num}{\mbox{\sc num}}
%\newcommand{\S}{\mathbb{A}}
\newcommand{\Det}{\mbox{\sc det}}
\newcommand{\preorderB}{\mathbb{B}}
\newcommand{\simA}{\sim_A}
\newcommand{\simB}{\sim_B}
\newcommand{\polarizedtype}{\mbox{\sf poltype}}

% Fonts for the logics we are talking about
\newcommand{\Aunion}{\mathscr{A}^{\cup}}
\newcommand{\Munion}{\mathscr{M}^{\cup}}
\newcommand{\Sunion}{\mathscr{S}^{\cup}}
\newcommand{\Ainter}{\mathscr{A}^{\cap}}
\newcommand{\Minter}{\mathscr{M}^{\cap}}
\newcommand{\Sinter}{\mathscr{S}^{\cap}}

\newcommand{\BAPA}{\sf{BAPA}}
\newcommand{\QFBAPA}{\sf{QFBAPA}}
\newcommand{\CardCompLogic}{\sf{CardCompLogic}}

\newcommand{\proverule}{\textsc}

\newcommand{\axiom}{\proverule{axiom}}
\newcommand{\barbara}{\proverule{barbara}}
\newcommand{\unionl}{\proverule{union-l}}
\newcommand{\unionr}{\proverule{union-r}}
\newcommand{\unionall}{\proverule{union-all}}
\newcommand{\interl}{\proverule{inter-l}}
\newcommand{\interr}{\proverule{inter-r}}
\newcommand{\interall}{\proverule{inter-all}}
\newcommand{\some}{\proverule{some}}
\newcommand{\conversion}{\proverule{conversion}}
\newcommand{\darii}{\proverule{darii}}
\newcommand{\mix}{\proverule{mix}}
\newcommand{\size}{\proverule{size}}
\newcommand{\trans}{\proverule{trans}}
\newcommand{\morel}{\proverule{more-l}}
\newcommand{\morer}{\proverule{more-r}}
\newcommand{\moreatleast}{\proverule{more-atleast}}
\newcommand{\x}{\proverule{x}}
\newcommand{\raa}{\proverule{raa}}

% Complexity Macros
\newcommand{\Ptime}{\textsc{PTime}}
\newcommand{\NP}{\textsc{NP}}

%Macros for logic
\newcommand{\All}[2]{\mathsf{All}\,\,#1\,\,#2}
\newcommand{\Some}[2]{\mathsf{Some}\,\,#1\,\,#2}
\newcommand{\Atleast}[2]{\mathsf{AtLeast}\,\,#1\,\,#2}
\newcommand{\More}[2]{\mathsf{More}\,\,#1\,\,#2}
\newcommand{\Most}[2]{\mathsf{Most}\,\,#1\,\,#2}
\newcommand{\R}[2]{\mathsf{R}\,\,#1\,\,#2}
\newcommand{\AllNoArgs}{\mathsf{All}}
\newcommand{\SomeNoArgs}{\mathsf{Some}}
\newcommand{\AtleastNoArgs}{\mathsf{AtLeast}}
\newcommand{\MoreNoArgs}{\mathsf{More}}
\newcommand{\MostNoArgs}{\mathsf{Most}}

\newcommand{\card}{\mathrm{card}}


% Miscellaneous
\newcommand{\provesarbitrary}{\proves_{\mbox{\small{arb}}}}
\newcommand{\Ruleset}{\mathcal{R}}

\newcommand{\Diag}{\mbox{Diag}}
\newcommand{\OffDiag}{\mbox{Off-diag}}
\newcommand{\Pairs}{\mbox{Pairs}}
\newcommand{\Bad}{\mbox{Bad}}
\newcommand{\argmax}{\mbox{argmax}}
\newcommand{\Clamp}{\protect{\mbox{\textit{Clamp}}}}
%\newcommand{\sClamp}{\mbox{subset-Clamp}}
\newcommand{\ordercanonical}{<_{\scriptstyle can}}
\newcommand{\lex}{\ordercanonical}
\newcommand{\lexcanonical}{\ordercanonical}

\newcommand{\precsubseteq}{\Subset}
\newcommand{\approxsubset}{\Subset}

\newcommand{\suitable}{suitable}%%Saul removed the $\Aunion(\card)$ part

%%%%%%%%%%%%%%%%%%%%%%%%%%%%%%%%%%%%%%%%%%%%
% Some additional symbols for the proof of completeness of Aunion(\card)
\newcommand{\provsub}{\subseteq_{\Gamma}}
\newcommand{\provle}{\le_{\Gamma}}
\newcommand{\provlt}{<_{\Gamma}}
\newcommand{\provsubDelta}{\subseteq_{\Delta}}
\newcommand{\provleDelta}{\le_{\Delta}}
\newcommand{\provltDelta}{<_{\Delta}}
\newcommand{\provlestrict}{\provlt}

\newcommand{\nprovle}{\nleq_{\Gamma}}
\newcommand{\provextended}{\preceq_{\Gamma}}
\newcommand{\provextendedstrict}{\prec_{\Gamma}}
\newcommand{\nprovextended}{\npreceq_{\Gamma}}
\newcommand{\provprecsubseteq}{\precsubseteq_{\Gamma}}
\newcommand{\nprovleDelta}{\nleq_{\Delta}}
\newcommand{\provextendedDelta}{\preceq_{\Delta}}
\newcommand{\provextendedstrictDelta}{\prec_{\Delta}}
\newcommand{\nprovextendedDelta}{\npreceq_{\Delta}}
\newcommand{\provprecsubseteqDelta}{\precsubseteq_{\Delta}}

\newcommand{\provsubstar}{\subseteq_{\Gamma^\star}}
\newcommand{\provlestar}{\le_{\Gamma^\star}}
\newcommand{\provltstar}{<_{\Gamma^\star}}
\newcommand{\provlestrictstar}{\provltstar}
\newcommand{\provextendedstar}{\preceq_{\Gamma^\star}}
\newcommand{\provextendedstrictstar}{\prec_{\Gamma^\star}}
%%%%%%%%%%%%%%%%%%%%%%%%%%%%%%%%%%%%%%%%%%%%

%%%%%%%%%%%%%%%%%%%%%%%%%%%%%%%%%%%%%%%%%%%%%%%%%%%%%%%%%%%%%%
% New commands needed just for notes at the end
\newcommand{\toAtLeast}{\rightarrow}
\newcommand{\toMore}{\xrightarrow[]{<}}
\newcommand{\toAll}{\xrightarrow[]{\subseteq}}

\newcommand{\chainAtLeast}{\toAtLeast \ldots \toAtLeast}
\newcommand{\chainMore}{\toAtLeast \ldots \toMore \ldots \toAtLeast}
\newcommand{\chainAll}{\toAll \ldots \toAll}

%%%%%%%%%%%%%%%%%%%%%%%%%%%%%%%%%%%%%%%%%%%%%%%%%%%%%%%%%%%%%%
\renewcommand{\thesection}{S\arabic{section}}
\renewcommand{\thefigure}{S\arabic{figure}}

%%%%%%%%%%%%%%%%%%%%%%%%%%%%%%%%%%%%%%%%%%%%%%%%%%%%%%%%%%%%%%%
\newcommand{\noproof}{\rem}


%%%%%%%%%%%%%%%%%%%%%%%%%%%%%%%%%%%%%%%%%%%%%%%%%%%%%%%%%%%%%%%
% For cross-referencing between main file and supplementary file
\makeatletter
\newcommand*{\addFileDependency}[1]{
  \typeout{(#1)}
  \@addtofilelist{#1}
  \IfFileExists{#1}{}{\typeout{No file #1.}}
}
\makeatother

\newcommand*{\myexternaldocument}[1]{
    \externaldocument{#1}
    \addFileDependency{#1.tex}
    \addFileDependency{#1.aux}
}

\myexternaldocument{AAAI_draft}
%%%%%%%%%%%%%%%%%%%%%%%%%%%%%%%%%%%%%%%%%%%%%%%%%%%%%%%%%%%%%%%

\begin{document}

\maketitle

%%%%%%%%%%%%%%%%%%%%%%%%%%%%%%%%%%%%%%%%%%%%%%%%%%%%%%%%%%%%%%%
\section{Completeness of $\Aunion$ and $\Sunion$}
\label{s:supp:completeness_Aunion_Sunion}
%%%%%%%%%%%%%%%%%%%%%%%%%%%%%%%%%%%%%%%%%%%%%%%%%%%%%%%%%%%%%%%%%%%%%%

% Rules for \Sunion
\begin{figure*}[t]
\begin{equation*}
\boxed{
\begin{array}{c}
\begin{array}{ccc}
\\ 
\infer[(\some)]
    {\Some{x}{x}}
    {\Some{x}{y}} 
&
\infer[(\conversion)]
    {\Some{y}{x}}
    {\Some{x}{y}}
&
\infer[(\darii)]
    {\Some{x}{z}}
    {\Some{x}{y} & \All{y}{z}}
\end{array}
\\
\end{array}
}
\end{equation*}
\caption{The additional rules for the logics $\Sunion$ and $\Sinter$.
\label{fig-rules-some}}
\end{figure*}

In this section, we show the completeness of the logics $\Aunion$ and $\Sunion$ related to our main logics.  For reference, $\Aunion$ and $\Sunion$ are discussed in Remark~\ref{remark-related-logics}.  Both proofs of completeness rely on the following notions of up-closure and primality:

\begin{definition} 
A set $S$ of terms is an \emph{up-set (for $\Gamma$)} if whenever $t\in S$ and $\Gamma \nvdash \All{t}{u}$, then also $u\in S$.  We denote the \emph{upper closure} of a term $t$ by $\uparrow t = \set{u \mid \Gamma \proves \All{t}{u}}$.
$S$ is \emph{prime} if whenever $t\cup u \in S$, then either $t\in S$ or $u\in S$.
\end{definition}

Note that the notion of an up-set is relative to a set $\Gamma$, but the notion of a prime set does not refer to $\Gamma$ at all.
When $\Gamma$ is clear from the context, we just speak of a set $S$ being an up-set (without referencing $\Gamma$).

Lemma \ref{lemma-zorn} and Lemma \ref{lemma-some-prime-upset} relate prime up-sets to our logics $\Aunion$ and $\Sunion$, respectively.

\begin{lemma}  Fix a set $\Gamma$.
Let $x$ be any term, and assume that $\Gamma \not\proves \All{x}{(a \cup b)}$.
Then there is a prime up-set containing $x$ but not containing either $a$ or $b$.
\label{lemma-zorn}
\end{lemma}

\begin{proof}
Let $\SS$ be the family of sets $T$ which contains $x$, is closed upwards, and contains neither  $a$ nor $b$.
One such set in $\SS$ is $\uparrow x$.  Note first that $\uparrow x$ does not contain either $a$ or $b$ (for if $\Gamma \proves \All{x}{a}$, then since $\Gamma \proves \All{a}{(a \cup b)}$, we would contradict our hypothesis that $\Gamma \not \proves \All{x}{(a \cup b)}$).

By Zorn's Lemma, let $S$ be a maximal element of $\SS$ with respect to inclusion.
We claim that 
$S$ is  prime.   To see this, suppose that $c \cup d\in S$.  Suppose towards a contradiction that neither $c$ nor $d$ were in $S$.
By maximality, $S\cup\uparrow c$ and $S\cup\uparrow d$  would not belong to $\SS$. 
So they each contain $a$ or $b$.   Without loss of generality, $\Gamma \proves \All{c}{a}$ and $\Gamma \proves \All{d}{b}$.  
By (\unionall), $\Gamma \proves \All{(c \cup d)}{(a \cup b)}$.  Since $S$ is an up-set, $a \cup b$ belongs to $S$.  And this is a contradiction.
\end{proof}

\begin{lemma}
 Suppose that $\Gamma\not\proves \Some{t}{u}$.
 Suppose also that $\Gamma$ contains the sentence $\Some{x}{y}$.
 Then there is a prime up-set $S$ containing both $x$ and $y$ such that
 $S$
 does not contain both $t$ and $u$.
 \label{lemma-some-prime-upset}
 \end{lemma}
 
\begin{proof}
Let $\SS$ be the family of sets $T$ such that (1) $T$ contains both $x$ and $y$,
(2) $T$ is closed upwards, and
(3) $T$  does not contain both $t$ and $u$.
One such set in $\SS$ is $(\uparrow x)\cup (\uparrow y)$.  
  This set obviously has (1) and (2).
Here is the argument for (3):
 If $\Gamma \proves \All{x}{t}$
and $\Gamma \proves \All{y}{u}$, using (\darii)
and 
 the fact that $\Gamma$ contains the sentence $\Some{x}{y}$,
 we have
 $\Gamma \proves \Some{t}{u}$, a contradiction.)
The same would happen in other cases such as $\Gamma \proves \All{x}{t}$ and $\Gamma \proves \All{y}{u}$.
The other two rules of the logic are needed in the other cases of this lemma.

By Zorn's Lemma, let $S$ be a maximal element of $\SS$ with respect to inclusion.
We claim that 
$S$ is  prime.   To see this, suppose that $a \cup b\in S$,
where $a$ and $b$ are basic terms.
Suppose towards a contradiction that neither $a$ nor $b$ were in $S$.
By maximality, $S\cup(\uparrow a)$ and $S\cup(\uparrow b)$  do not belong to $\SS$. 
The only problems could come from condition (3).
Then $\Gamma \proves \All{a}{t}, \All{a}{u}, \All{b}{t}$, and $\All{b}{u}$.
But then $\Gamma \proves \All{(a \cup b)}{t}$ and $\Gamma \proves \All{(a \cup b)}{u}$.
So $S$, being closed upwards, contains both $t$ and $u$, and this is a contradiction.
\end{proof}

We may now show the completeness of both $\Aunion$ and $\Sunion$.

\begin{theorem}
    The logic $\Aunion$ is complete.
    \label{theorem-completeness-Aunion}
\end{theorem}

\begin{proof}
We need to show that if $\Gamma\models \All{t}{u}$,
$\Gamma\proves \All{t}{u}$.
We may assume that $u$ is a union term.  (If $u$ were a basic term $a$, replace $a$ with $a\cup a$.)
We also may assume that $t$ is a basic term, since if $t$ were $a \cup b$ then both $\Gamma\models \All{a}{u}$
and $\Gamma\models \All{b}{u}$ follow from our assumption.   If we were to prove that $\Gamma \vdash \All{a}{u}$ and $\Gamma \vdash \All{b}{u}$, then by $(\proverule{union-all})$
we would have our desired conclusion:
$\Gamma \vdash \All{(a \cup b)}{u}$.

Thus, we reduce to showing that if  $\Gamma\models \All{a}{(b \cup c)}$, then also  $\Gamma \proves \All{a}{(b \cup c)}$, for basic terms $a, b, c$.
We show the contrapositive.   Suppose
 that $\Gamma\not\proves \All{a}{(b \cup c)}$.
By Lemma~\ref{lemma-zorn}, let $S$ be a prime up-set containing $a$ but not containing either $b$ or $c$.
We use $S$ to make a model $\Model$ with one point, say $*$.   We put $*\in \semantics{u}$ iff $u\in S$.

First we check that $\Model\models \Gamma$.  
Suppose that $\Gamma$ contains the sentence $\All{d}{(e \cup f)}$. We may assume that $\semantics{d} = \set{*}$, 
since otherwise $\semantics{d} = \emptyset$, and trivially $\semantics{d}\subseteq \semantics{e}\cup\semantics{f}$.
So $d \in S$.  As $S$ is closed upwards and $\Gamma \proves \All{d}{(e \cup f)}$, $e \cup f\in S$ also.   Since $S$ is prime, either $e\in S$ or $f\in S$.
So either $*\in\semantics{e}$ or $*\in \semantics{f}$.  Either way, $\semantics{e} \cup \semantics{f} = \set{*}$.  And again we have 
$\semantics{d}\subseteq \semantics{e}\cup\semantics{f}$.
Thus, $\Model\models \Gamma$.  

By the defining property of $S$, $*\in \semantics{a}$ but $*\not\in \semantics{b}\cup\semantics{c}$.   So 
$ \semantics{a} \not\subseteq \semantics{b}\cup\semantics{c}$.  Thus $\Model \not\models \All{a}{(b \cup c)}$.  We conclude that $\Gamma \not\models \varphi$.
This concludes the proof of Theorem~\ref{theorem-completeness-Aunion}.
\end{proof}


\begin{theorem}
    The logic $\Sunion$ is complete.
    \label{theorem-completeness-Sunion}
\end{theorem}
 
\begin{proof}

We would like to show the contrapositive:  
If $\Gamma \nvdash \varphi$ then $\Gamma \nvDash \varphi$. 
We have two cases.  Either $\varphi$ is $\All{t}{u}$, or $\varphi$ is $\Some{t}{u}$.

Consider the first case, and suppose $\Gamma \nvdash \All{t}{u}$.
Let $\Gamma_{\scriptsize all}$ be the $\AllNoArgs$-sentences in $\Gamma$.
Note that $\Gamma_{\scriptsize all}\not\proves \All{t}{u}$.
By Theorem~\ref{theorem-completeness-Aunion}, let $\Model$
be a model of $\Gamma_{\scriptsize all}$ where $\All{t}{u}$ is false.
Then add the same fresh point $*$ to $\semantics{a}$ for all basic terms $a$, and call the resulting model $\Nodel$.
Since for all $a$, $\semantics{a}_{\Nodel} = \semantics{a}_{\Model}\cup\set{*}$, $\Nodel$ satisfies every sentence $\Some{x}{y}$,
no matter whether this sentence is in $\Gamma$ or not.
And the addition of the fresh point to the interpretation of every term
has no effect on the $\AllNoArgs$-sentences, as a moment's thought shows.
So $\Nodel\models\Gamma$, and $\Nodel\not\models\All{t}{u}$.  Thus $\Gamma \nvDash \varphi$.

Now consider the second case.  Suppose $\Gamma \nvdash \Some{t}{u}$.  By Lemma \ref{lemma-some-prime-upset}, 
 for each sentence $\Some{x}{y}$ in $\Gamma$ we may
 choose a prime upset $S_{x,y}$ containing both $x$ and $y$
 but not containing both $t$ and $u$.
 Let 
 \[ M = \set{S_{x,y}: \Gamma \mbox{ contains }\protect{\Some{x}{y}}}.\]
 For a basic term $a$, let 
 \[\semantics{a} = \set{S_{x,y}
\in M: \Gamma \vdash \All{x}{a} \mbox{ or } \Gamma \vdash \All{y}{a}}.\]
 This equips $M$ with the structure of a model which we call $\Model$.
 Of course, for a binary union term $a \cup b$, 
 we automatically have $\semantics{a \cup b} = \semantics{a} \cup\semantics{b}$.
 Then the fact that each $S_{x,y}$ is closed upwards implies that $\Model$
satisfies the $\AllNoArgs$-sentences in $\Gamma$.
For a $\SomeNoArgs$-sentence in $\Gamma$, say $\Some{x}{y}$,
note that $S_{x,y}\in \semantics{x}\cap\semantics{y}$.  
Thus, $\Model\models\Gamma$.   

We claim that $\semantics{t}\cap\semantics{u} = \emptyset$
in $\Model$.   To see this, suppose towards a contradiction that 
$S_{x,y} \in\semantics{t}\cap\semantics{u} $.
Now $S_{x,y}\in M$, so $\Gamma$ contains
the sentence $\Some{x}{y}$.
We have a number of cases; one representative case is when
$S_{x,y} \in\semantics{t}$ due to $\Gamma \vdash \All{x}{t}$,
and $S_{x,y} \in\semantics{u}$ due to $\Gamma \vdash \All{y}{u}$.  But then we may use the rules (\darii) and (\conversion) to deduce $\Gamma \proves \Some{t}{u}$, which results in a contradiction.  So $\Model \vDash \Gamma$ and $\Model \nvDash \Some{t}{u}$.  Thus, again, $\Gamma \nvDash \varphi$.
 \end{proof}



%%%%%%%%%%%%%%%%%%%%%%%%%%%%%%%%%%%%%%%%%%%%%%%%%%%%%%%%%%%%%%%
\section{A Note on Arbitrary Terms}
\label{s:supp:arbitrary_terms}

In this section, we show that completeness of $\Aunion(\card)$ and $\Ainter(\card)$ with arbitrarily large finite terms follows from their completeness with only binary terms.  We illustrate this for union terms, although the same argument can be given mutatis mutandis for intersection terms.

Formally, we define an expanded logic $\Aunion_\mathrm{arb}(\card)$ as follows. We allow nested terms by changing our definition to an inductive one: a term is either a basic term or $(x\cup y)$, where $x$ and $y$ are terms. The semantics for terms is extended to nested terms in the obvious way. The sentences and rules of $\Aunion_\mathrm{arb}(\card)$ are the same as for $\Aunion(\card)$, except that they may now contain arbitrary nested terms.  We write $\provesarbitrary$ for the provability relation for $\Aunion_\mathrm{arb}(\card)$, reserving $\proves$ for the provability relation in $\Aunion(\card)$.

Let $\Gamma$ be a set of sentences in $\Aunion_\mathrm{arb}(\card)$, and let $\varphi$ be another such sentence.  We show that if $\Gamma \models \varphi$ then $\Gamma \provesarbitrary \varphi$. For any given sentence $\psi$ of $\Aunion_\mathrm{arb}(\card)$, we may obtain a new sentence $\psi^\star$ involving only binary union terms by recursively replacing binary unions in $\psi$ by fresh basic terms $t_i$ until there is only one union per argument remaining in $\psi$.  
Let $\Gamma^\star$ and $\varphi^\star$ be initially defined accordingly, modifying $\Gamma^\star$ as follows.  For every fresh term $t_i$ replacing binary union term, say $s_m \cup s_n$ in either $\Gamma$ \emph{or} $\varphi$, we include in $\Gamma^\star$ the sentences $\All{t_i}{(s_m \cup s_n)}$ and $\All{(s_m \cup s_n)}{t_i}$.  Note that $\Gamma^\star$ and $\varphi^\star$ involve only binary union terms.

It follows from $\Gamma \models \varphi$ that $\Gamma^\star \models \varphi^\star$, since a model $\Model$ of $\Gamma^\star$ is a model of both $\Gamma$ (and hence $\varphi$) as well as a model of those sentences added to $\Gamma^\star$ that ensure the intended semantics of the fresh terms $t_i$.  Assuming completeness of $\Aunion(\card)$ (shown in this paper), we have $\Gamma^\star \proves \varphi^\star$.  Let $\mathcal{T}^\star$ be a proof tree witnessing $\Gamma^\star \proves \varphi^\star$.  We construct a proof tree $\mathcal{T}$ for $\Gamma \provesarbitrary \varphi$ from $\mathcal{T}^\star$ by substituting back every previously introduced term $t_i$ in each sentence $\psi^\star$ in $\mathcal{T}^\star$ with the union it represents.  It remains to show that the premises of $\mathcal{T}$ are in $\Gamma$ (or are axioms), its conclusion is $\varphi$, and that each of the deductions in $\mathcal{T}$ follow by $\provesarbitrary$.  
Regarding the former facts, any premise of $\mathcal{T}^\star$ is either a sentence in $\Gamma$ with terms substituted, or is a new sentence that we added to $\Gamma^\star$.  A premise that simply has terms substituted will have the respective unions substituted back in for each $t_i$, and hence the corresponding premise of $\mathcal{T}$ is in $\Gamma$.  If a premise conclusion of $\mathcal{T}^\star$ is a sentence we added to $\Gamma^\star$, it is either $\All{t_i}{(s_m \cup s_n)}$ or $\All{(s_m \cup s_n)}{t_i}$.  Either way, after substituting back $s_m \cup s_n$ for $t_i$, we obtain $\All{(s_m \cup s_n)}{(s_m \cup s_n)}$, which is an instance of $(\axiom)$.  Similarly, after substitution the conclusion of $\mathcal{T}$ is $\varphi$.  As for the latter fact, each deduction still follows in $\mathcal{T}$ via the same rule that was used in that position of $\mathcal{T}^\star$.


%%%%%%%%%%%%%%%%%%%%%%%%%%%%%%%%%%%%%%%%%%%%%%%%%%%%%%%%%%%%%%%
\section{Proofs for Completeness of $\Aunion(\card)$}
\label{s:supp:completeness-Aunioncard}

In this section we prove facts stated in Section~\ref{section3} and Section~\ref{s:representation} in the paper.

%%%%%%%%%%%%%%%%%%%%%%%%%%%%%%%%%%%%%%%%%%%%%%%%%%%%%%%%%%%%%%%%%%%%%%%%%%%%%%%%%%%%%
\subsection*{Proof of Proposition~\ref{proposition-linearization}}

We begin with a very general result.   Let $(P,\leq_0)$ be any preorder whatsoever.
We show that $(P,\leq_0)$ has a 
 \emph{linearization}.  This is a relation $\leq_1$ which is a linear preorder on the same set $P$,
and such that for all $x,y\in P$:
(a) if $x \leq_0 y$, then also $x\leq_1 y$; and 
(b) if $x <_0 y$, then $x <_1 y$.
Here is how we obtain $\leq_1$:
Let $\equiv$ be the equivalence
relation on $P$ defined by $x \equiv y$ iff $x \leq_0 y$ and
$y\leq_0 x$. Let $Q$ be the quotient set $P/\!\equiv$.
(This is the set of equivalence classes $[x]$ for $x\in P$.)
Let $\leq_Q$ be the induced order: $[x]\leq_Q [y]$ iff $x \leq_0 y$ in $P$.  Then $(Q, \leq_Q)$ is anti-symmetric, and indeed it is a 
\emph{partial order}.  It is a standard fact that every partial order
(finite or not) may be enlarged to a linear order.
In the finite case, this is a topological sort;
in the general case, it requires a weak form of the Axiom of Choice.

Let $\leq_L$ be such an extension of $\leq_Q$.
Define $\leq_1$ on $P$ by $x \leq_1 y$ iff $[x] \leq_L [y]$.
Then $\leq_1$  is 
a linear preorder.  For (a), let  $x \leq_0 y$.
Then $[x] \leq_Q [y]$, so also $[x] \leq_L [y]$,
and thus $x\leq_1 y$.  For (b), suppose that $x <_0 y$.
Then $[x] <_Q [y]$.  So $[x] \leq_L [y]$.
We claim that $[x] <_L [y]$.  The reason:
if $[y] \leq_L [x]$, then we would have $[x] = [y]$
by anti-symmetry of $\leq_L$. 
This means that $x \leq_0 y\leq_0 x$.
But this contradicts $x <_0 y$.

The paragraphs above are quite general.   We now return to the
setting of Proposition~\ref{proposition-linearization}.
We take $P$ to be $\Pairs$
and $\leq_0$ to be $\provle$.
We use what we have just seen  to obtain a linearization $\leq_1$
of $(P,\leq_0)$
as we have defined it above.
Let \[ A = \set{z :  z \leq_1 x^*  \mbox{ but } z \nleq_0 x^*}.\]
We construct a new linear preorder $\leq_2$ by taking $\leq_1$ 
and moving all points in $A$ (in order) to just after $x^*$. 
Formally, $u \leq_2 v$ if any of the following four conditions holds:
\begin{enumerate}
    \item  $u,v\notin A$, and $u\leq_1 v$
    \item $u, v \in A$, and $u\leq_1 v$
        \item $u\in A$, $v\notin A$, and $x^* <_1 v$
    \item $u\notin A$, $v\in A$, and $u\leq_1 x^*$

\end{enumerate}
This relation $\leq_2$ is reflexive due to (1) and (2).
For the linearity, let $u, v\in A$.   If $u,v\in A$, then by (1), either $u \leq_2 v$ or $v\leq_2 u$.
The same thing happens when $u,v\notin A$.   For the remaining two cases, let us assume that $u\in A$ 
but $v\notin A$.     Since $\leq_1$ is linear, either $v \leq_1 x^*$ or $x^* <_1 v$.
In the first case, $v \leq_2 u$ by (4).   In the second case, $u\leq_2 v$ by (3).

\medskip

The transitivity of $\leq_2$ takes sixteen cases.  We are going to list them by pairs $(i,j)$.
So $(i,j)$ means that $u \leq_2 v$ by (i) and $v \leq_2 w$ by (i).


$(1,1)$: we have $u,v,w\notin A$,  and 
$u \leq_1 v \leq_1 w$.   So $u \leq_1 w$.    Thus  $u \leq_2 w$ by (1).

$(1,2)$, $(1,3)$, $(3,2)$, and $(3,3)$: these are impossible because we would have $v\notin A$ and $v\in A$.

 

$(1,4)$:  we have $u,v\notin A$, $w\in A$, $u\leq_1 v$ and $v \leq_1 x^*$.
So $u \leq_1 x^*$ as well. 
Thus, $u\leq_2 w$ by (4).

$(2,1)$, $(2,4)$,
$(4,1)$, and  $(4,4)$ : these are impossible  because we would have  $v\in A$ and $v\notin A$.


$(2,2)$: we have $u,v,w\in A$,  and 
$u \leq_1 v \leq_1 w$.   So $u \leq_1 w$.    Thus  $u \leq_2 w$ by (2).

$(2,3)$: we have $u,v\in A$, $u\leq_1 v$,  and $w <_1 x^*$. 
Then  $u \leq_2 w$ by (3).

$(3,1)$: we have $u\in A$, $v,w\notin A$, $x^*<_1 v$, and $v \leq_1 w$.
So $x^* <_1 w$.   We have $u \leq_2 w$ by (3).


$(3,4)$: we have $u, w\in A$, $v\notin A$, $x^* <_1 v$, and $v\leq_1 x^*$.
So we have $x^* <_1 v \leq_1 x^*$.  This gives $x^* <_1 x^*$.  Of course, this is a contradiction.

 

$(4,2)$: we have $u\notin A$, $v,w\in A$, $u\leq_1 x^*$, and $v \leq_1 w$.   
Then $u \leq_2 w$ by (4)

$(4,3)$: we have  $u,w\notin A$, $v\in A$, $u\leq_1 x^*$, and $x^* <_1 w$.
We have $u \leq_1 w$.   So  $u \leq_2 w$ by (1).
 
 \medskip


Before we move on,
notice that  $x^*\notin A$, and our definition above arranges that
$x^* <_2 z$ for all $z\in A$. 

 \medskip
 
Let us check linearization condition  (a) for   $\leq_2$, using the fact that $\leq_1$ is linear.
  Suppose that  $u\leq_0 v$.
We break into four cases depending on membership in $A$.
Suppose that $u, v\in A$.  Then since $u\leq_1 v$, we also have $u \leq_2 v$.
The same happens when $u, v\notin A$.

We next consider the case when $u\in A$ and $v\notin A$.
We want to show that $x^* <_1 v$.  If not, then by linearity, $v\leq_1 x^*$.
Since $v\notin A$, we must have $v \leq_0 x^*$.  But then $u \leq_0 v \leq_0 x^*$,
and we contradict $u\in A$.  Thus, $u\leq_2 v$ by (3) in the definition of $\leq_2$.

For our last case, suppose that $u\notin A$ and $v\in A$.
Since $v\in A$, $v\leq_1 x^*$.  Since $u\leq_0 v$, we also have 
$u\leq_1 v$.  By transitivity, $u\leq_1 x^*$.
Then $u \leq_2 v$ by condition (4) in the  definition of 
$\leq_2$.
 
 \medskip
 

Turning to linearization condition (b) for $\leq_2$, suppose that $u <_0 v$.
We showed above that (a) holds for $\leq_2$, and so $u \leq_2 v$; we must show that $v \nleq_2 u$.
Since $\leq_1$ is a linearization of $\leq_0$, we know that $u <_1 v$.

We shall assume that $v \leq_2 u$ and derive a contradiction.
We have four cases, (1)--(4) in the definition of $\leq_2$ above, but with $u$ and $v$ switched.  
In the first two cases, we have $v \leq_1 u$; this contradicts  $u <_1 v$.
 
In the third case, $v\in A$, $u\notin A$, and $x^* <_1 u$.
Then $v\leq_1 x^*$, since $v\in A$.
So $v\leq_1 x^* <_1 u$, and thus $v <_1 u$.  
This contradicts  $u <_1 v$.

In the last case, $v\notin A$, $u\in A$,  and $v \leq_1 x^*$.
Since   $u <_0 v$ and since $\leq_1$ is a linearization of $\leq_0$, we have $u <_1 v$.
And now we have $u <_1 v \leq_1 x^*$.  So $u <_1 x^*$.  
By the linearization property again, $u <_0 x^*$.  This contradicts $u\in A$.

\medskip

Now we have a linearization $\leq_2$.
We check the additional property in our result
for  $\leq_2$.  Suppose that $y\nleq_0 x^*$.
If $x^*\leq_1 y$, then also $x^*\leq_2 y$.   So we may assume that  $x^*\nleq_1 y$.  By linearity,
$y <_1 x^*$.
But then $y\in A$.  By condition (4) above, $x^* \leq_2 y$.
But we don't have $y \leq_2 x^*$.
(Since $y\in A$ and $x\notin A$, the only way that we could have  $y \leq_2 x^*$ is if $x^* <_1 x^*$.
Of course this is impossible.)
So $x^* <_2 y$, as desired.

%%%%%%%%%%%%%%%%%%%%%%%%%%%%%%%%%%%%%%%%%%%%%%%%%%%%%%%%%%%%%%%%%%%%%%%%%%%%%%%%%%%%%
\subsection*{Proof of Proposition~\ref{proposition-suitablepair}}

We will show that $(\provextended, \provsub)$ is a suitable pair.  Except for parts (2) and (6) in the definition of a \suitable{} pair, everything follows from the fact that our logical system has rules that directly ensure the relevant closure properties.  Part (2) follows by construction, since $\provextended$ is a linearization of $\provle$.  So we turn to (6).  Write $p$ as $(a,b)$ and $q$ as $(c,d)$.
Suppose that $p \provsub q$ and $q \provextended p$.    We must show that $q \provsub p$.
    
We have $\Gamma \vdash \All{(a \cup b)}{(c \cup d)}$.  
    By ($\proverule{subset-size}$), $\Gamma \vdash \Atleast{(c \cup d)}{(a \cup b)}$.  
    That is,  $p\provle q$.
   Now if $p\provlestrict q$, we would also have $p\provextendedstrict q$,
    since $\provextended$ is a linearization of $\provle$.
    This would contradict  $q \provextended p$.  
     Thus  $p\provle q$ but $p \not \provlestrict q$; i.e.,
     $q\provle p$.  
     This means that 
     $\Gamma \vdash \Atleast{(a \cup b)}{(c \cup d)}$.  
     Using (\mix),  $\Gamma \vdash \All{(c \cup d)}{(a \cup b)}$.
     Therefore $q \provsub p$, as desired.

%%%%%%%%%%%%%%%%%%%%%%%%%%%%%%%%%%%%%%%%%%%%%%%%%%%%%%%%%%%%%%%%%%%%%%%%%%%%%%%%%%%%%
\subsection*{Proof of Proposition~\ref{proposition-sClamp}}

We wish to show that properties (1) - (3) hold after a single application of the $\Clamp$ construction (i.e. that they hold for $T$).  We first show (1).  Suppose that $(a, b) \Subset (c, d)$.  Then both $(a,a) \Subset (c,d)$ and $(b,b) \Subset (c,d)$.  And so $T_a = S_a$ and $T_b = S_b$.  Thus, $T_{a,b} = T_a \cup T_b = S_a \cup S_b = S_{a,b}$.
 
Turning to part (2), let  $(c,d) \prec (a,b)$.
 We claim that either $(a,a) \not \precsubseteq (c,d)$ or $(b,b) \not \precsubseteq (c,d)$.
To see this, suppose towards a contradiction that both $(a,a) \precsubseteq(c,d)$ and $(b,b) \precsubseteq (c,d)$.
 Then $(a,b) \precsubseteq (c,d)$, and so $(a,b) \preceq (c,d)$.  And this contradicts 
$(c,d) \prec (a,b)$.

Without loss of generality, say that $(a,a) \precsubseteq (c,d)$.
Then $T_a = S_a \cup \set{*_1,\ldots, *_r}$.
And so,
\begin{align*} T_{a,b} &= T_a \cup T_b\\ &= S_a\cup\set{*_1,\ldots, *_r} \cup S_b \\&= S_{a,b}\cup\set{*_1,\ldots, *_r}.
\end{align*}
This completes the proof of part (2).

Finally, we show part (3): that if $S$ preserves and reflects $\Subset$, then so does $T$.  We first show that $T$ preserves $\Subset$, i.e. for $p, q \in \Pairs$ if $p \Subset q$ then $T_p \subseteq T_q$.
We have two cases.  First, if $p \precsubseteq (c,d)$ then we have $T_p = S_p$.
By our assumption that $S$ preserves $\precsubseteq$, $S_p \subseteq S_q$.
And clearly $S_q \subseteq T_q$.
So in this case we easily get $T_p \subseteq T_q$.

It remains to argue the case when
$p \not \precsubseteq (c,d)$.  In this case,  
$T_p =  S_p \cup \set{*_1,\ldots, *_r}$.
We claim that in this case,
$T_q =  S_q \cup \set{*_1,\ldots, *_r}$.
This again would imply $T_p \subseteq T_q$.

Suppose towards a contradiction that 
$T_q \neq  S_q \cup \set{*_1,\ldots, *_r}$.
Then $T_q =  S_q $, by construction.  Write $q$ as $(e, f)$.  So we have $T_{e,e} = S_{e,e}$ and $T_{f,f} = S_{f,f}$.
By construction of $T$, we must have $(e,e)\precsubseteq (c, d)$
and   $(f,f)\precsubseteq (c, d)$.
By property (4) of Definition~\ref{def-suitable-pair-first} of a suitable pair of relations, we have $(e,f)\precsubseteq (c, d)$.
Recall that 
$p \precsubseteq (e,f)$. 
And so we have $p \precsubseteq (c, d)$. 
This is a contradiction to the assumption in this case that 
$p \not \precsubseteq (c, d)$.

Concerning the reflection of $\precsubseteq$:
 if we take any family which reflects $\precsubseteq$ and
add fresh points to the base sets
in any way whatsoever, the resulting family will reflect $\precsubseteq$.  Since $T$ is the result of single application of $\Clamp$, T reflects $\precsubseteq$.\hfill$\square$


%%%%%%%%%%%%%%%%%%%%%%%%%%%%%%%%%%%%%%%%%%%%%%%%%%%%%%%%%%%%%%%%%%%%%%%%%%%%%%%%%%%%%
\subsection*{Proof of Lemma~\ref{lemma-equalizing}}

We now verify that the construction of $T$ in the proof of Lemma~\ref{lemma-equalizing} satisfies properties (1) - (3) listed in Lemma~\ref{lemma-equalizing}.
Recall that in order to construct $T$, we first selected one pair in each $\approx$-class of $C$ and listed the pairs $(a_1, b_1), \ldots, (a_k, b_k)$ such that $s_{a_1, b_1} \leq s_{a_2, b_2} \leq \cdots \leq s_{a_k, b_k}$.  We then took $T = T^{k-1}$, where:

\[ \begin{array}{lcl}
 T^1  & = &  \Clamp(S,a_2,b_2,s_{a_2, b_2} - s_{a_1, b_1})\\
T^2 & = & \Clamp(T^1,a_3, b_3, s_{a_3, b_3} - s_{a_2, b_2} )\\
  & \vdots   & \\
T^{k-1} & = & \Clamp(T^{k-2},a_k,b_k,
s_{a_k, b_k} - s_{a_{k-1}, b_{k-1}})\\
\end{array}
\]

To save on a lot of notation, we write $s_i$ for $s_{a_i, b_i}$,
and similarly for $t^j_i$.

We would first like to show (1): for $1 \leq r, s \leq k$, $t_{r} = t_{s}$ (actually, since $T = T^{k-1}$, we want to show that $t^{k-1}_{r} = t^{k-1}_{s}$).  In fact, we will show by induction on $1 \leq i \leq k - 1$ something stronger:
\begin{equation}
\label{equalization}
s_{i+1}  = t^i_{1} = t^i_{i} = \cdots = t^i_{i} =  t^i_{i+1}.
\end{equation}
We may then take $i = k - 1$ in order to prove our result.    

Well, for $i = 1$, we must show that $s_2 = t^1_1 = t^1_2$.  Recall our observation made at the beginning of the proof of Lemma~\ref{lemma-equalizing}:  $(a_1,b_1) \not \precsubseteq (a_2,b_2)$.  So by the first $\Clamp$ application, $t^1_1 = s_1 + (s_2 - s_1) = s_2$.
So $t^1_1 = s_1 + (s_2 - s_1) = s_2$.
Moreover, $t^1_2= s_2$, since the definition of $T^1$ uses $\Clamp$ at $(a_2,b_2)$.

Assume (\ref{equalization}) for  $i$.
Let   $1\leq j \leq i+1$.  Again, we observed before that $(a_{i+1},b_{i+1}) \not \precsubseteq (a_{i+2},b_{i+2})$.  Then
\[ t^{i+1}_j = s_{i+1} + (s_{i+2} - s_{i+1}) 
= s_{i+2}\]
Also, $t^{i+1}_{i+2} = s_{i+2} $, since $T^{i+1}$ uses $\Clamp$
at $(a_{i+2},b_{i+2})$.

We would now like to show (2) and (3).  Proposition~\ref{proposition-sClamp}, part (2) states that a single application of $\Clamp$ equally increases the sizes of all $(a, b)$ above a fixed $(c, d)$.  We may use this fact to show (2) by straightforward induction on the $i^{\textrm{th}}$ application of $\Clamp$.  By Proposition~\ref{proposition-sClamp}, part (3), the result of a single application of $\Clamp$ preserves and reflects $\Subset$.  So (3) can be shown by straightforward induction on the $i^{\textrm{th}}$ application of $\Clamp$ as well.  \hfill$\square$

%%%%%%%%%%%%%%%%%%%%%%%%%%%%%%%%%%%%%%%%%%%%%%%%%%%%%%%%%%%%%%%%%%%%%%%%%%%%%%%%%%%%%
\subsection*{Proof of Lemma~\ref{lemma-sizeadjustment}}

We verify that the construction of $T$ in the proof of Lemma~\ref{lemma-sizeadjustment} satisfies properties (1) - (3) listed in Lemma~\ref{lemma-sizeadjustment}.
Again, recall that in order to construct $T$, we listed all the size competitors $p_1, \ldots, p_k$ and then took $T = T^k$ where:

\[ \begin{array}{lcl}
 T^1  & = &  \Clamp(S,p_1,  s_{p_1} -m + 1)\\

T^2 & = & \Clamp(T^1,p_2,  s_{p_2}-m + 1 )\\
  & \vdots   & \\
T^{k} & = & \Clamp(T^{k-1},p_k, s_{p_k}-m + 1 )\\
\end{array}
\]

We first handle (2):  For all pairs $p \prec q_j$ (for all $j$), we also have $t_p < t_{q_j}$ (for all $j$).  Let $p \prec q_j$ for all $j$.  If $p$ is not a size competitor, then $s_p < s_{q_j}$ for all $j$.  So $t_{q_j} - t_p \ge s_{q_j} - s_p > 0$, and we are done.

Now suppose instead that $p$ is a size competitor; say $p$ is some $p_i$.  Then $p_i$ gains elements from the $\Clamp$ construction as we move from $T^{i-1}$ to $T^i$.  That is,

\begin{equation}
\label{nearendsubset}
\begin{array}{lccc}
t_{p_i} & = & & s_{p_i}\\
& & + & ( s_{p_1}-m  + 1)\\
& & + & \vdots\\
& & + & (s_{p_{i-1}} -m  + 1)\\
& & + & (s_{p_{i+1}} -m  + 1)\\
& & + & \vdots\\
& & + & (s_{p_k}-m + 1)\\
%& > &  \\
\end{array}
\end{equation}
The upshot is that
\begin{equation}\label{upshot}
\begin{array}{lcl}
t_{q_j} - t_{p_i} & \geq & s_{q_j} + (s_{p_i}-m + 1) - s_{p_i}\\
& = & s_{q_j} - m + 1\\
& \geq & s_{q_j} -s_{q_i} +1\\
& = & 1\\
\end{array}
\end{equation}
(At the end of (\ref{upshot}), we used the fact that $m \leq s_{q_i}$.)
By (\ref{upshot}) we see that $t_{q_j} > t_{p_i}$.

\rem{The reason that the first $\ge$ is not an equals sign $=$ is that it may be the case that $p_i \Subset p_{i^'}$}

Parts (1) and (3) follow from straightforward induction arguments on applications of $\Clamp$ (as in Lemma~\ref{lemma-equalizing}).
This completes the proof.\hfill$\square$



%%%%%%%%%%%%%%%%%%%%%%%%%%%%%%%%%%%%%%%%%%%%%%%%%%%%%%%%%%%%%%%
\section{Proofs for Completeness of $\Ainter(\card)$}
\label{s:supp:completeness-Aintercard}

\subsection*{Proof of Lemma~\ref{lemma-proof-translation}}

Our mapping $\phi\mapsto\phi^\cup$ extends to instances of inference rules; instances of rules in $\Ainter(\card)$ are mapped to instances of the rules in $\Aunion(\card)$. 
In particular, instances of (\interl), (\interr), and (\interall) are sent to instances of (\unionl), (\unionr), and (\unionall), respectively.  For every other rule, an instance of the rule is sent to an instance of the same rule.  So any proof tree in $\Aunion(\card)$ witnessing $\Gamma \proves \varphi$ is sent to a proof tree in $\Aunion(\card)$ witnessing $\Gamma^\cup \proves \varphi^\cup$.  Showing that $\Gamma^\cup \proves \varphi^\cup \implies \Gamma \proves \varphi$ is similar.\hfill$\square$


\subsection*{Proof of Lemma~\ref{proposition-union-inter-conversion}}

Note that for all basic terms $a$ and $b$, 
$\semantics{a\cap b}_{\Model^\cap} =\semantics{a}_{\Model^\cap}
\cap \semantics{b}_{\Model^\cap}
=  \overline{\semantics{a}_{\Model}
\cup \semantics{b}_{\Model}  }
=  \overline{\semantics{(a\cap b)^{\cup}}_{\Model}  }
$.
Thus, for all $\lang^\cap$-terms $x$, 
$\semantics{x}_{\Model^\cap} = \overline{\semantics{x^\cup}}_{\Model}$.
Let $\psi$ be the $\lang^\cap$-sentence $\All{x}{y}$.
Then 
$\Model^\cap \models \psi$
iff
$\semantics{x}_{\Model^\cap} \subseteq \semantics{y}_{\Model^\cap}$
iff
$\overline{\semantics{y}_{\Model^\cap}} \subseteq \overline{\semantics{x}_{\Model^\cap}}$
iff
$\semantics{y^\cup}_\Model \subseteq \semantics{x^\cup}_\Model$
iff
$\Model\models\All{y^\cup}{x^\cup}$
iff
$\Model\models\psi^\cup$.
The same argument works for $\AtleastNoArgs$-sentences.\hfill$\square$



\subsection*{Proof of Theorem~\ref{theorem-completeness-intersection}}

Suppose that $\Gamma\cup\set{\phi}$ are $\lang^\cap$
sentences, and $\Gamma \not \proves \varphi$ in $\Ainter(\card)$.
By Lemma~\ref{lemma-proof-translation},  $\Gamma^\cup\not\proves\phi^\cup$ in $\Aunion(\card)$.
By Theorem \ref{theorem-completeness-Aunioncard},
we have a model $\Model$ for $\lang^\cup$
such that $\Model \models \Gamma^\cup$ and $\Model \not \models \varphi^\cup$. 
Consider $\Model^\cap$ as defined  above.
By Proposition~\ref{proposition-union-inter-conversion},
$\Model^\cap\models\Gamma$
and $\Model^\cap\not\models\phi$.\hfill$\square$


%%%%%%%%%%%%%%%%%%%%%%%%%%%%%%%%%%%%%%%%%%%%%%%%%%%%%%%%%%%%%%%
\section{Proofs for Completeness of $\Munion(\card)$}
\label{s:supp:completeness-Munioncard}

In this section, we provide the full proof of completeness of $\Munion(\card)$.  We make frequent references to the proof of Theorem~\ref{theorem-completeness-Aunioncard}. 

\subsection*{Proof of Theorem~\ref{theorem-completeness-Munioncard}}

Suppose that b$\Gamma$ is a finite, consistent set of sentences in $\Munion(\card)$, and suppose that $\Gamma \not \proves \varphi$.  Our plan is again to build a model of $\Gamma$ where $\varphi$ is false.
When $\varphi$ is an $\AllNoArgs$- or $\AtleastNoArgs$-sentence, we build our model as in the proof of Theorem \ref{theorem-completeness-Aunioncard}.  We deal here with the case where $\varphi$ is $\More{x}{y}$

Since $\Gamma \not \proves \More{x}{y}$, we cannot have a proof of $\More{x}{y}$ via (\raa) in particular.  That is, $\Gamma \cup \set{\Atleast{y}{x}} \not \proves \More{z}{z}$.  This means that $\Gamma \cup \set{\Atleast{y}{x}}$ is consistent.  We now only need to construct a model $\Model$ of $\Gamma \cup \set{\Atleast{y}{x}}$; such a model is a model of $\Gamma$, and in addition satisfies $\Model \not \models \More{x}{y}$, since $|\semantics{y}| \ge |\semantics{x}|$.

For what follows, let $\Gamma^\star = \Gamma \cup \set{\Atleast{y}{x}}$.  In order to construct the model $\Model$ of $\Gamma^\star$, we first obtain the suitable pair $(\provextendedstar, \provsubstar)$ as before.  We obtain a $BT$-family of sets $(S_{a})_{a \in BT}$ such that, in addition to the implications in (\ref{arrows}) (with $\Gamma^\star$ in place of $\Gamma$), for all $(a, b), (c, d) \in \Pairs$ we have:

\begin{equation}
\label{arrows-more}
\begin{array}{c}
\Gamma^\star \proves \More{(a \cup b)}{(c \cup d)}\\
\Downarrow\\
(c,d) \provextendedstrictstar (a,b)\\
\Updownarrow\\ 
S_c \cup S_d < S_a \cup S_b\\
\end{array}
\end{equation}

The $\Updownarrow$-arrow follows from the Representation Lemma (Lemma~\ref{lemma-representation}).  For the $\Downarrow$, suppose that $\Gamma^\star \proves \More{(a \cup b)}{(c \cup d)}$.  We have $\Gamma^\star \proves \Atleast{(a \cup b)}{(c \cup d)}$ by (\moreatleast).  Write $p$ for the pair $(a, b)$ and $q$ for $(c, d)$.  So we have $q \provlestar p$.  We cannot have $p \provlestar q$, since that would mean $\Gamma^\star \proves \Atleast{(c \cup d)}{(a \cup b)}$, and $\Gamma$ would be inconsistent.  So $q \provlestrictstar p$.  By the definition of ``linearization'', we have $q \provextendedstrictstar p$.

We build $\Model$ from our family $S$ exactly as in the proof of Theorem \ref{theorem-completeness-Aunioncard}:  For every basic term $a$, let $\semantics{a} = S_a$.  By the implications in (\ref{arrows}), with $\Gamma$ replaced with $\Gamma^\star$, $\Model$ satisfies the $\AllNoArgs$- and $\AtleastNoArgs$-sentences in $\Gamma^\star$.  Additionally, by (\ref{arrows-more}), $\Model$ staisfies the $\MoreNoArgs$-sentences in $\Gamma^\star$.  So $\Model \models \Gamma^\star$, and we are done.\hfill$\square$


%%%%%%%%%%%%%%%%%%%%%%%%%%%%%%%%%%%%%%%%%%%%%%%%%%%%%%%%%%%%%%%
\section{Proofs for Complexity of Our Logics}
\label{s:supp:complexity-proofs}

\subsection*{Proof of Theorem~\ref{theorem-ptime}}

For this proof, we need to be a bit more careful about which basic terms are used in proofs. So we fix a background language, built from a set of basic terms. Suppose $\mathcal{L}$ is a sublanguage, built from a subset of the basic terms, $\Gamma$ is a set of $\mathcal{L}$-sentences, and $\varphi$ is an $\mathcal{L}$-sentence. Then we write $\Gamma\vdash_{\mathcal{L}} \varphi$ if $\Gamma\vdash \varphi$ and this is witnessed by a proof tree using only $\mathcal{L}$-sentences. Similarly, we write $\Gamma\models_{\mathcal{L}} \varphi$ if every $\mathcal{L}$-model of $\Gamma$ satisfies $\varphi$. 

\begin{lemma}\label{lemma-language}
Let $\Gamma$ be a set of sentences, and let $\varphi$ be a sentence. Let $\mathcal{L}$ be the language containing only the basic terms appearing in $\Gamma \cup \{\varphi\}$. Then $\Gamma\vdash \varphi$ if and only if $\Gamma\vdash_{\mathcal{L}} \varphi$. 
\end{lemma}

\begin{proof}
One direction is clear. For the other, we assume $\Gamma\vdash \varphi$ and we want to show $\Gamma\vdash_{\mathcal{L}}\varphi$. By soundness in the full language and completeness in the restricted language $\mathcal{L}$, it suffices to show that $\Gamma\models \varphi$ implies  $\Gamma\models_{\mathcal{L}} \varphi$.

So assume $\Gamma\models \varphi$, and let $\Model$ be an $\mathcal{L}$-model of $\Gamma$. We extend $\Model$ to a structure $\Model'$ in the full language by assigning the basic terms which are not in $\mathcal{L}$ arbitrary interpretations. Then $\Model'\models \Gamma$, so $\Model'\models \varphi$, and $\Model\models \varphi$, since satisfaction of $\mathcal{L}$-sentences does not depend on the basic terms which are not in $\mathcal{L}$.  
\end{proof}

With this Lemma at hand, we may now prove that $\vdash$ is decidable in $\Ptime$.
Let $\Gamma$ be a set of sentences, and let $\varphi$ be a sentence. Let $n$ be the combined length of $\Gamma$ and $\varphi$. Furthermore, we let $\mathcal{L}$ be the language with the set of basic terms restricted to those appearing in $\Gamma$ and $\varphi$. The number of terms and the number of sentences in $\mathcal{L}$ are each bounded by a polynomial in $n$, and by Lemma~\ref{lemma-language}, $\Gamma\vdash \varphi$ if and only if $\Gamma\vdash_{\mathcal{L}} \varphi$.

Now we have a finite set of rules, and a substitution instance of a rule is obtained by substituting at most three terms for term variables in the rule. So there is a polynomial $p(x)$ and a set $R$ of substitution instances of rules of size at most $p(n)$ such that if $\Gamma\vdash \varphi$, then there is a proof tree such that each leaf and node is labeled by an element of $R\cup \Gamma$. Further, we may assume that no element of $R$ appears twice along any path through the proof tree from the root to a leaf. Otherwise, we could shorten the path by replacing the subtrees above the premises of the lower instance of the rule by the subtrees above the premises of the higher instance of the rule. It follows from the pigeonhole principle that if $\Gamma\vdash \varphi$, then this is witnessed by a proof tree of height at most $p(n)$. 

We can now decide if $\Gamma\vdash \varphi$ as follows: Let $\Gamma_0 = \Gamma$. Given $\Gamma_i$, let $\Gamma_{i+1}$ be $\Gamma_i$ together with all sentences which can be deduced from premises in $\Gamma_i$ by a proof rule in $R$. Each set $\Gamma_i$ has size bounded by a polynomial in $n$ (since every element of $\Gamma_i$ is either in $\Gamma$ or is the conclusion of an element of $R$), and $\Gamma_{i+1}$ can be computed from $\Gamma_i$ in polynomial time. It follows that $\Gamma_{p(n)}$ can be computed from $\Gamma$ in polynomial time. By induction, $\Gamma_i$ is the set of all sentences $\psi$ in $\mathcal{L}$ such that $\Gamma\vdash_{\mathcal{L}} \psi$ by a proof tree of height at most $i$. Then $\Gamma\vdash \varphi$ if and only if $\varphi\in \Gamma_{p(n)}$.\hfill$\square$


\subsection*{Proof of Theorem~\ref{theorem-ptime-model-building}}

Let $n$ be the combined length of $\Gamma$ and $\varphi$. For $\Ainter(\card)$, one may use the translation from intersection terms to union terms in order to build a countermodel of $\varphi$ from one for $\varphi$ in $\Ainter(\card)$, as is done in the proof of Theorem~\ref{theorem-completeness-intersection}.  It is easily seen that this translation can be done in polynomial time.

For $\Aunion(\card)$, we wish to show that the model $\Model$ used in the proof of Theorem~\ref{theorem-completeness-Aunioncard} can be constructed in polynomial time.  First, we may construct $\provle$ and $\provsub$ over $\Pairs$ in polynomial time, since deciding whether $\Gamma \proves \Atleast{(a \cup b)}{(c \cup d)}$ and whether $\Gamma \proves \All{(a \cup b)}{c \cup d)}$ is in $\Ptime$ by Theorem~\ref{theorem-ptime}.  One may also check that extending $\provle$ to a linear ordering $\provextended$ can be done in polynomial time.

It remains to ensure that the application of Lemma~\ref{lemma-representation} can be done in polynomial time, since no more work is needed to build the countermodel $\Model$.  Let $K$ be the number of size classes listed in the proof of Lemma~\ref{lemma-representation}.  Note that $K$ is bounded by a polynomial in $n$.  Our procedure for constructing $S^K$ involves $K$ steps; in each step, we apply Lemmas \ref{lemma-equalizing} and \ref{lemma-sizeadjustment} in sequence.  The former lemma involves selecting pairs from $\provsub$-equivalence classes, which can be done in polynomial time.  Both lemmas otherwise involve fewer than $K$ applications of the $\Clamp$ construction.  

For each application of $\Clamp$, we must first check whether $(a, a) \not \provsub (i, j)$, i.e. whether $\Gamma \not \proves \All{(a \cup a)}{(i \cup j)}.$   Again, this check is in $\Ptime$ by Theorem~\ref{theorem-ptime}.
Finally, in each application of $\Clamp$, we must verify that the number of points added is bounded by $K$.  We proceed by induction on $i$, the index denoting the current stage $S^i$ of our family of sets.  We consider only those applications of $\Clamp$ in Lemma \ref{lemma-equalizing}, although the argument follows similarly for those in Lemma \ref{lemma-sizeadjustment}.

Consider the number of points added in a given instance of $\Clamp$ just prior to extending family $S^i$.  By inductive hypothesis, each instance of $\Clamp$ applied to obtain $S^i$ added a number of points bounded by a polynomial in $K$ to each set $S^i_a$.  Hence the \textit{total} number of points in each set $S^i_a$ is polynomial in $K$.  When extending $S^i$ to $T = S^{i+1}$, the number of points added in a given $\Clamp$ instance is $|S^i_{a_k} \cup S^i_{b_k}| - |S^i_{a_{k-1}} \cup S^i_{b_{k-1}}|$.  This is consequently bounded by a polynomial in $K$.
With this, we are done.



%%%%%%%%%%%%%%%%%%%%%%%%%%%%%%%%%%%%%%%%%%%%%%%%%%%%%%%%%%%%%%%
\section{Failure of Compactness for $\Aunion(\card)$}
\label{s:supp:non-compact}

We prove that the logic $\Aunion(\card)$ is  not compact.
We exhibit a set $\Gamma$ and a sentence $\phi$
 such that
 $\Gamma\models\phi$, but for all finite $\Gamma_0\subseteq\Gamma$,
 $\Gamma\not\models\phi$. 
 
 Our set of basic terms is $\set{x,y,a_0,a_1, \ldots}$.
We take $\varphi$ to be the sentence $\All{x}{y}$. For each $n\in \omega$, let $\Gamma_n$ be the following set of sentences:
\begin{align*}
\All{(x\cup y)}{(a_i\cup a_j)}&\quad \text{for }0\leq  i\neq j\leq n\\
\All{a_i}{(x\cup y)} &\quad \text{for  } 0\leq i\leq n\\
\Atleast{y}{a_i}&\quad \text{for } 0\leq i\leq n
\end{align*}
Let $\Gamma = \bigcup_{n\in \omega} \Gamma_n$. 

\begin{claim} $\Gamma\models \varphi$. 
\end{claim}

\begin{proof}
Suppose that $\Model\models \Gamma$. We have $\semantics{a_i}\subseteq \semantics{x\cup y}$ for all $i\in \omega$.
Recall that $\Model$ is a finite model.
Thus,  there are only finitely many subsets of $\semantics{x\cup y}$.
So we must have  $i\neq j$ such that  $\semantics{a_i} = \semantics{a_j}$. But then $\semantics{y}\subseteq \semantics{x\cup y}\subseteq \semantics{a_i\cup a_j}=\semantics{a_i}$, and $|\semantics{y}| \geq |\semantics{a_i}|$, so $\semantics{y} = \semantics{a_i}$. Also $\semantics{x}\subseteq \semantics{x\cup y}\subseteq \semantics{a_i\cup a_j}=\semantics{a_i} = \semantics{y}$, so $\Model\models \varphi$. 
\end{proof}

\begin{claim} For any finite $\Delta\subseteq \Gamma$, $\Delta\not\models \varphi$. 
\end{claim}
\begin{proof}
Since $\Delta$ is finite, there is some $n\in \omega$ such that $\Delta\subseteq \Gamma_n$. So it suffices to exhibit a model $\Model\models \Gamma_n$ with $\Model\not\models \varphi$. 

The domain of $\Model$ will be $M = \{0,\dots,n\}$. Let $\semantics{x} = \semantics{a_n} = M\setminus \{n\}$, let $\semantics{y} = \semantics{a_0} = M\setminus \{0\}$, and let $\semantics{a_i} = M\setminus \{i\}$ for all $0\leq i\leq n$. Then $\semantics{x\cup y} = M$. 
Also, $\semantics{a_i\cup a_j} = M$ for all $0\leq i \ne j\leq n$, and $|\semantics{y}| = |\semantics{a_i}| = n$ for all $i\in \omega$.
So $\mathcal{M}\models \Gamma_n$. 
In addition, $\semantics{x}\not\subseteq \semantics{y}$, so $\mathcal{M}\not\models \varphi$. 
\end{proof}



%%%%%%%%%%%%%%%%%%%%%%%%%%%%%%%%%%%%%%%%%%%%%%%%%%%%%%%%%%%%%%%
\section{Illustration of the Representation Lemma Algorithm}
\label{s:supp:illustration}

% |S[0]| = 120 \\
% |S[1]| = 120 \\
% |S[2]| = 120 \\
% |S[3]| = 120 \\
% |S[4]| = 89 \\
% |S[5]| = 94 \\
% |S[6]| = 111 \\
% |S[7]| = 101 \\
% |S[8]| = 120 \\

In this section, we provide an example to better illustrate the algorithm for our Representation Lemma (described in Section~\ref{s:representation}).  In particular, we illustrate a step in the process of repeatedly applying Lemma~\ref{lemma-equalizing} and Lemma~\ref{lemma-sizeadjustment} for each size class $C_i$.  Let $n = 9$, and let $\prec$ have size classes as shown in lists below:
\[
 \begin{array}{l}
\ [(5,5),(6,6)], \\
 \  [(5,6),(4,4),(7,7)], \\
  \  [(4,7),(4,5),(2, 2),(1,1),(0,0),(8,8), (3,3)],\\
  \ [(2,3),  (1,2),(1,3),(0,7),(0,3), (0,2)],\\
\	    [(0,1), (0,4),(1,7),(2,7), (2,8),(1,8), (3,8), (7,8)],\\
 \        [(3,7), (5,7),(6,7),(1,4),(2,4), (3,4), (6,8)], \\
\	    [(0,6),(1,6),(2,6),(3,6),(1,5), (5,8), (4,8)],\\
 \        [(4,6),(0,5),(2,5),(3,5)]
         \end{array}
\]
To make this procedure easier to follow, we define $\Subset$ for this example such that the only pairs $\Subset$-below a given pair $(a, b)$ are $(a, a)$, $(b, b)$, and $(a, b)$.
 
We illustrate the procedure with step $6$.  We begin with a family $S$ with cardinalities as shown:
\[
\arraycolsep=3.0pt
%\begin{array}[t]{l@{\qquad\qquad}l@{\qquad\qquad}l@{\qquad\qquad}l@{\qquad\qquad}l}
\begin{array}[t]{lllll}
 \begin{array}{l}
|S_0| = 65 \\
|S_1| = 63 \\
|S_2| = 63 \\
|S_3| = 64 \\
|S_4| = 64 \\
|S_5| = 70 \\
|S_6| = 71 \\
|S_7| = 60 \\
|S_8| = 68 \\
\\
s_{5, 5} = 70 \\
s_{6, 6} = 71 \\
\end{array}
 &
\begin{array}{l}
s_{5, 6} = 79 \\
s_{4, 4} = 64 \\
s_{7, 7} = 60 \\
\\
s_{4, 7} = 79 \\
s_{4, 5} = 79 \\
s_{2, 2} = 63 \\
s_{1, 1} = 63 \\
s_{0, 0} = 65 \\
s_{8, 8} = 68 \\
s_{3, 3} = 64 \\
\\
s_{2, 3} = 80 \\
s_{1, 2} = 80 \\
s_{1, 3} = 80 \\
s_{0, 7} = 80 \\
s_{0, 3} = 80 \\
s_{0, 2} = 80 \\
 \end{array}
&
  \begin{array}{l}
s_{0, 1} = 81 \\
s_{0, 4} = 81 \\
s_{1, 7} = 81 \\
s_{2, 7} = 81 \\
s_{2, 8} = 81 \\
s_{1, 8} = 81 \\
s_{3, 8} = 81 \\
s_{7, 8} = 81 \\
\\
s_{3, 7} = 82 \\
s_{5, 7} = 82 \\
s_{6, 7} = 82 \\
s_{1, 4} = 82 \\
s_{2, 4} = 82 \\
s_{3, 4} = 82 \\
s_{6, 8} = 82 \\
   \end{array}
 &
  \begin{array}{l}
s_{0, 6} = 83 \\
s_{1, 6} = 83 \\
s_{2, 6} = 83 \\
s_{3, 6} = 83 \\
s_{1, 5} = 83 \\
s_{5, 8} = 83 \\
s_{4, 8} = 83 \\
\\
s_{4, 6} = 84 \\
s_{0, 5} = 84 \\
s_{2, 5} = 84 \\
s_{3, 5} = 84 \\
\end{array}
\end{array}
\]

Step $6$ concerns the sixth size class, starting from the highest one.
So  we are working on the size class $[(7,4),(4,5),(2, 2),(1,1),(0,0),(8,8), (3,3)]$.  The first step is to equalize the sizes of unions of the pairs in this class, using Lemma~\ref{lemma-equalizing}.  We first reorder our size classes in order of the size of the union corresponding to each pair, obtaining $[(1,1), (2, 2), (3,3),  (0,0),(8,8), (7,4),(4,5)]$.  We then calculate:
\[ \begin{array}{lcl}
 T^1  & = &  \Clamp(S,(2,2), 63-63)\\

T^2 & = & \Clamp(T^1,(3,3),  64-63)\\

T^{3} & = & \Clamp(T^{2}, (0,0), 65-64 )\\
T^{4} & = & \Clamp(T^{3}, (8,8), 68-65 )\\
T^{5} & = & \Clamp(T^4, (7,4),79-69 )\\
T^{6} & = & \Clamp(T^{5}, (4,5), 79-79)\\
\end{array}
\]
We take $T = T^6$.  Note that the process of equalization preserves the relative size of pairs within the same size class, for all size classes above the current (sixth) size class.

After equalizing, we get 
\[
\arraycolsep=3.0pt
\begin{array}[t]{lllll}
 \begin{array}{l}
|S_0| = 103 \\
|S_1| = 103 \\
|S_2| = 103 \\
|S_3| = 103 \\
|S_4| = 72 \\
|S_5| = 94 \\
|S_6| = 111 \\
|S_7| = 84 \\
|S_8| = 103 \\
\\
s_{5, 5} = 94 \\
s_{6, 6} = 111 \\
\end{array}
 &
\begin{array}{l}
s_{5, 6} = 119 \\
s_{4, 4} = 72 \\
s_{7, 7} = 84 \\
\\
s_{4, 7} = 103 \\
s_{4, 5} = 103 \\
s_{2, 2} = 103 \\
s_{1, 1} = 103 \\
s_{0, 0} = 103 \\
s_{8, 8} = 103 \\
s_{3, 3} = 103 \\
\\
s_{2, 3} = 120 \\
s_{1, 2} = 120 \\
s_{1, 3} = 120 \\
s_{0, 7} = 120 \\
s_{0, 3} = 120 \\
s_{0, 2} = 120 \\
 \end{array}
&
  \begin{array}{l}
s_{0, 1} = 121 \\
s_{0, 4} = 121 \\
s_{1, 7} = 121 \\
s_{2, 7} = 121 \\
s_{2, 8} = 121 \\
s_{1, 8} = 121 \\
s_{3, 8} = 121 \\
s_{7, 8} = 121 \\
\\
s_{3, 7} = 122 \\
s_{5, 7} = 122 \\
s_{6, 7} = 122 \\
s_{1, 4} = 122 \\
s_{2, 4} = 122 \\
s_{3, 4} = 122 \\
s_{6, 8} = 122 \\
   \end{array}
 &
  \begin{array}{l}
s_{0, 6} = 123 \\
s_{1, 6} = 123 \\
s_{2, 6} = 123 \\
s_{3, 6} = 123 \\
s_{1, 5} = 123 \\
s_{5, 8} = 123 \\
s_{4, 8} = 123 \\
\\
s_{4, 6} = 124 \\
s_{0, 5} = 124 \\
s_{2, 5} = 124 \\
s_{3, 5} = 124 \\
\end{array}
\end{array}
\]

We must now ensure that the pairs in the sixth size class have greater size than all pairs in preceding size classes.  At this point, the size competitors are $(5,6)$ and $(6,6)$.  So we want to make the sizes of the pairs in our current size class larger than the sizes of $(5,6)$ and $(6,6)$.  We use Lemma~\ref{lemma-sizeadjustment} to do this:  We clamp $(5,6)$ and $(6,6)$, increasing all sets by
one more than the difference of the sizes of those sets with $103$.
We get
\[
\arraycolsep=3.0pt
\begin{array}[t]{lllll}
 \begin{array}{l}
|S_0| = 120 \\
|S_1| = 120 \\
|S_2| = 120 \\
|S_3| = 120 \\
|S_4| = 89 \\
|S_5| = 94 \\
|S_6| = 111 \\
|S_7| = 101 \\
|S_8| = 120 \\
\\
s_{5, 5} = 94 \\
s_{6, 6} = 111 \\
\end{array}
 &
\begin{array}{l}
s_{5, 6} = 119 \\
s_{4, 4} = 89 \\
s_{7, 7} = 101 \\
\\
s_{4, 7} = 120 \\
s_{4, 5} = 120 \\
s_{2, 2} = 120 \\
s_{1, 1} = 120 \\
s_{0, 0} = 120 \\
s_{8, 8} = 120 \\
s_{3, 3} = 120 \\
\\
s_{2, 3} = 137 \\
s_{1, 2} = 137 \\
s_{1, 3} = 137 \\
s_{0, 7} = 137 \\
s_{0, 3} = 137 \\
s_{0, 2} = 137 \\
 \end{array}
&
  \begin{array}{l}
s_{0, 1} = 138 \\
s_{0, 4} = 138 \\
s_{1, 7} = 138 \\
s_{2, 7} = 138 \\
s_{2, 8} = 138 \\
s_{1, 8} = 138 \\
s_{3, 8} = 138 \\
s_{7, 8} = 138 \\
\\
s_{3, 7} = 139 \\
s_{5, 7} = 139 \\
s_{6, 7} = 139 \\
s_{1, 4} = 139 \\
s_{2, 4} = 139 \\
s_{3, 4} = 139 \\
s_{6, 8} = 139 \\
   \end{array}
 &
  \begin{array}{l}
s_{0, 6} = 140 \\
s_{1, 6} = 140 \\
s_{2, 6} = 140 \\
s_{3, 6} = 140 \\
s_{1, 5} = 140 \\
s_{5, 8} = 140 \\
s_{4, 8} = 140 \\
\\
s_{4, 6} = 141 \\
s_{0, 5} = 141 \\
s_{2, 5} = 141 \\
s_{3, 5} = 141 \\
\end{array}
\end{array}
\]
Note that it wasn't really necessary to clamp $(6,6)$ after we clamped $(5,6)$.
So our algorithm does a bit of work that is not necessary.   
It could be elaborated to produce slightly smaller sets in the end.  But it is correct.



\end{document}