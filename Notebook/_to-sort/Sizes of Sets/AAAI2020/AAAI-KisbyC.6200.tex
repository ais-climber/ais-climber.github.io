\def\year{2020}\relax
%File: formatting-instruction.tex
\documentclass[letterpaper]{article} % DO NOT CHANGE THIS
\usepackage{aaai20}  % DO NOT CHANGE THIS
\usepackage{times}  % DO NOT CHANGE THIS
\usepackage{helvet} % DO NOT CHANGE THIS
\usepackage{courier}  % DO NOT CHANGE THIS
\usepackage[hyphens]{url}  % DO NOT CHANGE THIS
\usepackage{graphicx} % DO NOT CHANGE THIS
\urlstyle{rm} % DO NOT CHANGE THIS
\def\UrlFont{\rm}  % DO NOT CHANGE THIS
\usepackage{graphicx}  % DO NOT CHANGE THIS
\frenchspacing  % DO NOT CHANGE THIS
\setlength{\pdfpagewidth}{8.5in}  % DO NOT CHANGE THIS
\setlength{\pdfpageheight}{11in}  % DO NOT CHANGE THIS
% \nocopyright

%PDF Info Is REQUIRED.
% For /Author, add all authors within the parentheses, separated by commas. No accents or commands.
% For /Title, add Title in Mixed Case. No accents or commands. Retain the parentheses.
\pdfinfo{
/Title (Logics for Sizes with Union or Intersection)
/Author (Caleb Kisby, Saul A. Blanco, Alex Kruckman, Lawrence S. Moss)
} %Leave this	

% /Title ()
% Put your actual complete title (no codes, scripts, shortcuts, or LaTeX commands) within the parentheses in mixed case
% Leave the space between \Title and the beginning parenthesis alone
% /Author ()
% Put your actual complete list of authors (no codes, scripts, shortcuts, or LaTeX commands) within the parentheses in mixed case. 
% Each author should be only by a comma. If the name contains accents, remove them. If there are any LaTeX commands, 
% remove them. 

% DISALLOWED PACKAGES
% \usepackage{authblk} -- This package is specifically forbidden
% \usepackage{balance} -- This package is specifically forbidden
% \usepackage{caption} -- This package is specifically forbidden
% \usepackage{color (if used in text)
% \usepackage{CJK} -- This package is specifically forbidden
% \usepackage{float} -- This package is specifically forbidden
% \usepackage{flushend} -- This package is specifically forbidden
% \usepackage{fontenc} -- This package is specifically forbidden
% \usepackage{fullpage} -- This package is specifically forbidden
% \usepackage{geometry} -- This package is specifically forbidden
% \usepackage{grffile} -- This package is specifically forbidden
% \usepackage{hyperref} -- This package is specifically forbidden
% \usepackage{navigator} -- This package is specifically forbidden
% (or any other package that embeds links such as navigator or hyperref)
% \indentfirst} -- This package is specifically forbidden
% \layout} -- This package is specifically forbidden
% \multicol} -- This package is specifically forbidden
% \nameref} -- This package is specifically forbidden
% \natbib} -- This package is specifically forbidden -- use the following workaround:
% \usepackage{savetrees} -- This package is specifically forbidden
% \usepackage{setspace} -- This package is specifically forbidden
% \usepackage{stfloats} -- This package is specifically forbidden
% \usepackage{tabu} -- This package is specifically forbidden
% \usepackage{titlesec} -- This package is specifically forbidden
% \usepackage{tocbibind} -- This package is specifically forbidden
% \usepackage{ulem} -- This package is specifically forbidden
% \usepackage{wrapfig} -- This package is specifically forbidden
% DISALLOWED COMMANDS
% \nocopyright -- Your paper will not be published if you use this command
% \addtolength -- This command may not be used
% \balance -- This command may not be used
% \baselinestretch -- Your paper will not be published if you use this command
% \clearpage -- No page breaks of any kind may be used for the final version of your paper
% \columnsep -- This command may not be used
% \newpage -- No page breaks of any kind may be used for the final version of your paper
% \pagebreak -- No page breaks of any kind may be used for the final version of your paperr
% \pagestyle -- This command may not be used
% \tiny -- This is not an acceptable font size.
% \vspace{- -- No negative value may be used in proximity of a caption, figure, table, section, subsection, subsubsection, or reference
% \vskip{- -- No negative value may be used to alter spacing above or below a caption, figure, table, section, subsection, subsubsection, or reference

\setcounter{secnumdepth}{1} %May be changed to 1 or 2 if section numbers are desired.

% The file aaai20.sty is the style file for AAAI Press 
% proceedings, working notes, and technical reports.
%
\setlength\titlebox{2.5in} % If your paper contains an overfull \vbox too high warning at the beginning of the document, use this
% command to correct it. You may not alter the value below 2.5 in
%Your title must be in mixed case, not sentence case. 
% That means all verbs (including short verbs like be, is, using,and go), 
% nouns, adverbs, adjectives should be capitalized, including both words in hyphenated terms, while
% articles, conjunctions, and prepositions are lower case unless they
% directly follow a colon or long dash



\title{Logics for Sizes with Union or Intersection}

\author{Caleb Kisby,\textsuperscript{\rm 1} 
Sa\'ul A.~Blanco,\textsuperscript{\rm 1} 
Alex Kruckman,\textsuperscript{\rm 2} and 
Lawrence S.~Moss\textsuperscript{\rm 3}\\ 
% All authors must be in the same font size and format. Use \Large and \textbf to achieve this result when breaking a line
\textsuperscript{1}
Department of Computer Science, Indiana University, Bloomington, IN 47408, USA \\
cckisby@indiana.edu, sblancor@indiana.edu \\
\textsuperscript{2}
Department of Mathematics and Computer Science, Wesleyan University, Middletown, CT 06459,  USA\\
akruckman@wesleyan.edu \\
\textsuperscript{3}
Department of Mathematics, Indiana University, Bloomington, IN 47405, USA \\
lmoss@indiana.edu \\
}


%If you have multiple authors and multiple affiliations
% use superscripts in text and roman font to identify them. For example, Sunil Issar,\textsuperscript{\rm 2} J. Scott Penberthy\textsuperscript{\rm 3} George Ferguson,\textsuperscript{\rm 4} Hans Guesgen\textsuperscript{\rm 5}. Note that the comma should be placed BEFORE the superscript for optimum readability
% email address must be in roman text type, not monospace or sans serif



%%%%%%%%%%%%%%%%%%%%%%%%%%%%%%%%%%%%%%%%%%%%%%%%%%
% Additional Packages (Not required by AAAI)
%%%%%%%%%%%%%%%%%%%%%%%%%%%%%%%%%%%%%%%%%%%%%%%%%%

\usepackage{amssymb,amsthm,amsmath}
\usepackage{mathtools}
\usepackage{lscape}

% Packages Caleb added: %%%
%\usepackage{xcolor}
%\usepackage{stmaryrd}
%\usepackage{comment}
%\usepackage{soul}
\usepackage{enumerate} % Used by Alex
\usepackage[shortlabels]{enumitem}
\usepackage[mathscr]{eucal}
%\usepackage{marginnote} % I couldn't get a simple marginpar to work for the life of me...
\usepackage{stackengine}
\usepackage{pdfpages}
\usepackage{xr}
\usepackage{microtype}
%%%%%%%%%%%%%%%%%%%%%%%%%%%

\newcommand{\authnote}[2]{{#1}{#2}}
\newcommand{\snote}[1]{{\authnote{Sa\'ul: }{{\color{blue} #1}}}}

\newcommand{\cnote}[1]{{\authnote{Caleb: }{{\color{purple} #1}}}}
%%%%

\usepackage{bigstrut}
%\usepackage{MnSymbol}
%\usepackage{bbm}
\usepackage{proof}
\usepackage{bussproofs}
%\usepackage{tikz}
%\usepackage{lingmacros}


\newcommand{\existsgeq}{\mbox{\sf AtLeast}}
\newcommand{\Pol}{\mbox{\emph{Pol}}}
  \newcommand{\nonered}{\textcolor{red}{=}}
  \newcommand{\equalsred}{\nonered}
  \newcommand{\redstar}{\textcolor{red}{\star}}
    \newcommand{\dred}{\textcolor{red}{d}}
    \newcommand{\dmark}{\dred}
    \newcommand{\redflip}{\textcolor{red}{flip}}
        \newcommand{\flipdred}{\textcolor{red}{\mbox{\scriptsize \em flip}\ d}}
        \newcommand{\mdred}{\textcolor{blue}{m}\textcolor{red}{d}}
        \newcommand{\ndred}{\textcolor{blue}{n}\textcolor{red}{d}}
\newcommand{\arrowm}{\overset{\textcolor{blue}{m}}{\rightarrow} }
\newcommand{\arrown}{\overset{\textcolor{blue}{n}}{\rightarrow} }
\newcommand{\arrowmn}{\overset{\textcolor{blue}{mn}}{\longrightarrow} }
\newcommand{\arrowmonemtwo}{\overset{\textcolor{blue}{m_1 m_2}}{\longrightarrow} }
\newcommand{\bluen}{\textcolor{blue}{n}}
\newcommand{\bluem}{\textcolor{blue}{m}}
\newcommand{\bluemone}{\textcolor{blue}{m_1}}
\newcommand{\bluemtwo}{\textcolor{blue}{m_2}}
\newcommand{\blueminus}{\textcolor{blue}{-}}

\newcommand{\bluedot}{\textcolor{blue}{\cdot}}
\newcommand{\bluepm}{\textcolor{blue}{\pm}}
\newcommand{\blueplus}{\textcolor{blue}{+ }}
\newcommand{\translate}[1]{{#1}^{tr}}
\newcommand{\Caba}{\mbox{\sf Caba}} 
\newcommand{\Set}{\mbox{\sf Set}} 
\newcommand{\Pre}{\mbox{\sf Pre}} 
\newcommand{\wmarkpolarity}{\scriptsize{\mbox{\sf W}}}
\newcommand{\wmarkmarking}{\scriptsize{\mbox{\sf Mon}}}
\newcommand{\smark}{\scriptsize{\mbox{\sf S}}}
\newcommand{\bmark}{\scriptsize{\mbox{\sf B}}}
\newcommand{\mmark}{\scriptsize{\mbox{\sf M}}}
\newcommand{\jmark}{\scriptsize{\mbox{\sf J}}}
\newcommand{\kmark}{\scriptsize{\mbox{\sf K}}}
\newcommand{\tmark}{\scriptsize{\mbox{\sf T}}}
%%{\mbox{\ensuremath{>}}}
\newcommand{\true}{\top}
\newcommand{\false}{\bot}
\newcommand{\upred}{\textcolor{red}{\uparrow}}
\newcommand{\downred}{\textcolor{red}{\downarrow}}
\usepackage[all,cmtip]{xy}
\usepackage{enumitem}
%\usepackage{fullpage} % Forbidden by AAAI



%\usepackage[authoryear]{natbib}
%\newcommand{\citet}[1]{\citeauthor{#1}~\shortcite{#1}}
%\newcommand{\citep}{\cite}
%\newcommand{\citealp}[1]{\citeauthor{#1}~\citeyear{#1}



%\usepackage{multicol}

\newtheorem{theorem}{Theorem}[section]
\newtheorem{lemma}[theorem]{Lemma}
\newtheorem{claim}[theorem]{Claim}
\newtheorem{proposition}[theorem]{Proposition}
\newtheorem{corollary}[theorem]{Corollary}
%\newtheorem{theorem}{Theorem}

\theoremstyle{definition}
\newtheorem{example}[theorem]{Example}
\newtheorem{remark}[theorem]{Remark}
\newtheorem*{relatedwork*}{Related Work}
\newtheorem*{contribution*}{Our Contribution}
\newtheorem*{organization*}{Organization}
\newtheorem*{nextsteps*}{Next Steps}
\newtheorem{definition}[theorem]{Definition}

\newcommand{\semantics}[1]{[\![\mbox{\em $ #1 $\/}]\!]}
\newcommand{\abovearrow}[1]{\rightarrow\hspace{-.14in}\raiseonebox{1.0ex}
{$\scriptscriptstyle{#1}$}\hspace{.13in}}
\newcommand{\toplus}{\abovearrow{r}}
\newcommand{\tominus}{\abovearrow{i}} 
\newcommand{\todestroy}{\abovearrow{d}}
\newcommand{\tom}{\abovearrow{m}}
\newcommand{\tomprime}{\abovearrow{m'}}
\newcommand{\A}{\textsf{App}}
\newcommand{\At}{\textsf{At}}
\newcommand{\Emb}{\textsf{Emb}}
\newcommand{\EE}{\mathbb{E}}
\newcommand{\DD}{\mathbb{D}}
\newcommand{\PP}{\mathbb{P}}
\newcommand{\QQ}{\mathbb{Q}}
\newcommand{\LL}{\mathbb{L}}
\newcommand{\MM}{\mathbb{M}}
\usepackage{verbatim}
\newcommand{\TT}{\mathcal{T}}
\newcommand{\Marking}{\mbox{Mar}}
\newcommand{\Markings}{\Marking}
\newcommand{\Mar}{\Marking}
\newcommand{\Model}{\mathcal{M}}
\newcommand{\Nodel}{\mathcal{N}}
\newcommand{\TTM}{\TT_{\Markings}}
\newcommand{\CC}{\mathbb{C}}
\newcommand{\erase}{\mbox{\textsf{erase}}}
\newcommand{\set}[1]{\{ #1 \}}
\newcommand{\arrowplus}{\overset{\blueplus}{\rightarrow} }
\newcommand{\arrowminus}{\overset{\blueminus}{\rightarrow} }
\newcommand{\arrowdot}{\overset{\bluedot}{\rightarrow} }
\newcommand{\arrowboth}{\overset{\bluepm}{\rightarrow} }
\newcommand{\arrowpm}{\arrowboth}
\newcommand{\arrowplusminus}{\arrowboth}
\newcommand{\arrowmone}{\overset{m_1}{\rightarrow} }
\newcommand{\arrowmtwo}{\overset{m_2}{\rightarrow} }
\newcommand{\arrowmthree}{\overset{m_3}{\rightarrow} }
\newcommand{\arrowmcomplex}{\overset{m_1 \orr m_2}{\longrightarrow} }
\newcommand{\arrowmproduct}{\overset{m_1 \cdot m_2}{\longrightarrow} }
\newcommand{\proves}{\vdash}
\newcommand{\Dual}{\mbox{\sc dual}}
\newcommand{\orr}{\vee}
\newcommand{\uar}{\uparrow}
\newcommand{\dar}{\downarrow}
\newcommand{\andd}{\wedge}
\newcommand{\bigandd}{\bigwedge}
\newcommand{\arrowmprime}{\overset{m'}{\rightarrow} }
\newcommand{\quadiff}{\quad \mbox{ iff } \quad}
\newcommand{\Con}{\mbox{\sf Con}}
\newcommand{\type}{\mbox{\sf type}}
\newcommand{\lang}{\mathcal{L}}
\newcommand{\necc}{\Box}
\newcommand{\vocab}{\mathcal{V}}
\newcommand{\wocab}{\mathcal{W}}
\newcommand{\Types}{\mathcal{T}_\mathcal{M}}
\newcommand{\mon}{\mbox{\sf mon}}
\newcommand{\anti}{\mbox{\sf anti}}
\newcommand{\FF}{\mathcal{F}}
\newcommand{\rem}[1]{\relax}

\newcommand{\raiseone}{\mbox{raise}^1}
\newcommand{\raisetwo}{\mbox{raise}^2}
\newcommand{\wrapper}[1]{{#1}}
\newcommand{\sfa}{\wrapper{\mbox{\sf a}}}
\newcommand{\sfb}{\wrapper{\mbox{\sf b}}}
\newcommand{\sfv}{\wrapper{\mbox{\sf v}}}
\newcommand{\sfw}{\wrapper{\mbox{\sf w}}}
\newcommand{\sfx}{\wrapper{\mbox{\sf x}}}
\newcommand{\sfy}{\wrapper{\mbox{\sf y}}}
\newcommand{\sfz}{\wrapper{\mbox{\sf z}}}
  \newcommand{\sff}{\wrapper{\mbox{\sf f}}}
    \newcommand{\sft}{\wrapper{\mbox{\sf t}}}
      \newcommand{\sfc}{\wrapper{\mbox{\sf c}}}
      \newcommand{\sfu}{\wrapper{\mbox{\sf u}}}
            \newcommand{\sfs}{\wrapper{\mbox{\sf s}}}
  \newcommand{\sfg}{\wrapper{\mbox{\sf g}}}

\newcommand{\sfvomits}{\wrapper{\mbox{\sf vomits}}}
\newsavebox{\mathfrbox}
\newenvironment{mathframe}
    {\begin{lrbox}{\mathfrbox}\begin{minipage}{\mathfrwidth}\begin{center}}
    {\end{center}\end{minipage}\end{lrbox}\noindent\fbox{\usebox{\mathfrbox}}}
    \newenvironment{mathframenocenter}
    {\begin{lrbox}{\mathfrbox}\begin{minipage}{\mathfrwidth}}
    {\end{minipage}\end{lrbox}\noindent\fbox{\usebox{\mathfrbox}}} 
 \newcommand{\nott}{\neg}
  \newcommand{\preorderO}{\mathbb{O}}
 \newcommand{\PreorderP}{\mathbb{P}}
  \newcommand{\preorderE}{\mathbb{E}}
\newcommand{\preorderP}{\mathbb{P}}
\newcommand{\preorderN}{\mathbb{N}}
\newcommand{\preorderQ}{\mathbb{Q}}
\newcommand{\preorderX}{\mathbb{X}}
\newcommand{\preorderA}{\mathbb{A}}
\newcommand{\preorderR}{\mathbb{R}}
\newcommand{\preorderOm}{\mathbb{O}^{\bluem}}
\newcommand{\preorderPm}{\mathbb{P}^{\bluem}}
\newcommand{\preorderQm}{\mathbb{Q}^{\bluem}}
\newcommand{\preorderOn}{\mathbb{O}^{\bluen}}
\newcommand{\preorderPn}{\mathbb{P}^{\bluen}}
\newcommand{\preorderQn}{\mathbb{Q}^{\bluen}}
 \newcommand{\PreorderPop}{\mathbb{P}^{\blueminus}}
  \newcommand{\preorderEop}{\mathbb{E}^{\blueminus}}
\newcommand{\preorderPop}{\mathbb{P}^{\blueminus}}
\newcommand{\preorderNop}{\mathbb{N}^{\blueminus}}
\newcommand{\preorderQop}{\mathbb{Q}^{\blueminus}}
\newcommand{\preorderXop}{\mathbb{X}^{\blueminus}}
\newcommand{\preorderAop}{\mathbb{A}^{\blueminus}}
\newcommand{\preorderRop}{\mathbb{R}^{\blueminus}}
 \newcommand{\PreorderPflat}{\mathbb{P}^{\flat}}
  \newcommand{\preorderEflat}{\mathbb{E}^{\flat}}
\newcommand{\preorderPflat}{\mathbb{P}^{\flat}}
\newcommand{\preorderNflat}{\mathbb{N}^{\flat}}
\newcommand{\preorderQflat}{\mathbb{Q}^{\flat}}
\newcommand{\preorderXflat}{\mathbb{X}^{\flat}}
\newcommand{\preorderAflat}{\mathbb{A}^{\flat}}
\newcommand{\preorderRflat}{\mathbb{R}^{\flat}}
\newcommand{\pstar}{\preorderBool^{\preorderBool^{E}}}
\newcommand{\pstarplus}{(\pstar)^{\blueplus}}
\newcommand{\pstarminus}{(\pstar)^{\blueminus}}
\newcommand{\pstarm}{(\pstar)^{\bluem}}
\newcommand{\Reals}{\preorderR}
\newcommand{\preorderS}{\mathbb{S}}
\newcommand{\preorderBool}{\mathbbm{2}}
 \newcommand{\NPplus}{\NP^{\blueplus}}
  \newcommand{\NPminus}{\NP^{\blueminus}}
   \newcommand{\NPplain}{\NP}
    \newcommand{\npplus}{np^{\blueplus}}
  \newcommand{\npminus}{np^{\blueminus}}
   \newcommand{\npplain}{np}
   \newcommand{\np}{np}
   \newcommand{\Term}{\mbox{\sc t}}
  \newcommand{\N}{\mbox{\sc n}}
   \newcommand{\X}{\mbox{\sc x}}
      \newcommand{\Y}{\mbox{\sc y}}
            \newcommand{\V}{\mbox{\sc v}}
    \newcommand{\Nbar}{\overline{\mbox{\sc n}}}
    \newcommand{\Pow}{\mathcal{P}}
    \newcommand{\powcontravariant}{\mathcal{Q}}
    \newcommand{\Id}{\mbox{Id}}
    \newcommand{\pow}{\Pow}
   \newcommand{\Sent}{\mbox{\sc s}}
   \newcommand{\lookright}{\slash}
   \newcommand{\lookleft}{\backslash}
   \newcommand{\dettype}{(e \to t)\arrowminus ((e\to t)\arrowplus t)}
\newcommand{\ntype}{e \to t}
\newcommand{\etttype}{(e\to t)\arrowplus t}
\newcommand{\nptype}{(e\to t)\arrowplus t}
\newcommand{\verbtype}{TV}
\newcommand{\who}{\infer{(\nptype)\arrowplus ((\ntype)\arrowplus (\ntype))}{\mbox{who}}}
\newcommand{\iverbtype}{IV}
\newcommand{\Nprop}{\N_{\mbox{prop}}}
\newcommand{\VP}{{\mbox{\sc vp}}}
\newcommand{\CN}{{\mbox{\sc cn}}}
\newcommand{\Vintrans}{\mbox{\sc iv}}
\newcommand{\Vtrans}{\mbox{\sc tv}}
\newcommand{\Num}{\mbox{\sc num}}
%\newcommand{\S}{\mathbb{A}}
\newcommand{\Det}{\mbox{\sc det}}
\newcommand{\preorderB}{\mathbb{B}}
\newcommand{\simA}{\sim_A}
\newcommand{\simB}{\sim_B}
\newcommand{\polarizedtype}{\mbox{\sf poltype}}

% Fonts for the logics we are talking about
\newcommand{\Aunion}{\mathscr{A}^{\cup}}
\newcommand{\Munion}{\mathscr{M}^{\cup}}
\newcommand{\Sunion}{\mathscr{S}^{\cup}}
\newcommand{\Ainter}{\mathscr{A}^{\cap}}
\newcommand{\Minter}{\mathscr{M}^{\cap}}
\newcommand{\Sinter}{\mathscr{S}^{\cap}}

\newcommand{\BAPA}{\sf{BAPA}}
\newcommand{\QFBAPA}{\sf{QFBAPA}}
\newcommand{\CardCompLogic}{\sf{CardCompLogic}}

\newcommand{\proverule}{\textsc}

\newcommand{\axiom}{\proverule{axiom}}
\newcommand{\barbara}{\proverule{barbara}}
\newcommand{\unionl}{\proverule{union-l}}
\newcommand{\unionr}{\proverule{union-r}}
\newcommand{\unionall}{\proverule{union-all}}
\newcommand{\interl}{\proverule{inter-l}}
\newcommand{\interr}{\proverule{inter-r}}
\newcommand{\interall}{\proverule{inter-all}}
\newcommand{\some}{\proverule{some}}
\newcommand{\conversion}{\proverule{conversion}}
\newcommand{\darii}{\proverule{darii}}
\newcommand{\mix}{\proverule{mix}}
\newcommand{\size}{\proverule{size}}
\newcommand{\trans}{\proverule{trans}}
\newcommand{\morel}{\proverule{more-l}}
\newcommand{\morer}{\proverule{more-r}}
\newcommand{\moreatleast}{\proverule{more-atleast}}
\newcommand{\x}{\proverule{x}}
\newcommand{\raa}{\proverule{raa}}

% Complexity Macros
\newcommand{\Ptime}{\textsc{PTime}}
\newcommand{\NP}{\textsc{NP}}

%Macros for logic
\newcommand{\All}[2]{\mathsf{All}\,\,#1\,\,#2}
\newcommand{\Some}[2]{\mathsf{Some}\,\,#1\,\,#2}
\newcommand{\Atleast}[2]{\mathsf{AtLeast}\,\,#1\,\,#2}
\newcommand{\More}[2]{\mathsf{More}\,\,#1\,\,#2}
\newcommand{\Most}[2]{\mathsf{Most}\,\,#1\,\,#2}
\newcommand{\R}[2]{\mathsf{R}\,\,#1\,\,#2}
\newcommand{\AllNoArgs}{\mathsf{All}}
\newcommand{\SomeNoArgs}{\mathsf{Some}}
\newcommand{\AtleastNoArgs}{\mathsf{AtLeast}}
\newcommand{\MoreNoArgs}{\mathsf{More}}
\newcommand{\MostNoArgs}{\mathsf{Most}}

\newcommand{\card}{\mathrm{card}}


% Miscellaneous
\newcommand{\provesarbitrary}{\proves_{\mbox{\small{arb}}}}
\newcommand{\Ruleset}{\mathcal{R}}

\newcommand{\Diag}{\mbox{Diag}}
\newcommand{\OffDiag}{\mbox{Off-diag}}
\newcommand{\Pairs}{\mbox{Pairs}}
\newcommand{\Bad}{\mbox{Bad}}
\newcommand{\argmax}{\mbox{argmax}}
\newcommand{\Clamp}{\protect{\mbox{\textit{Clamp}}}}
%\newcommand{\sClamp}{\mbox{subset-Clamp}}
\newcommand{\ordercanonical}{<_{\scriptstyle can}}
\newcommand{\lex}{\ordercanonical}
\newcommand{\lexcanonical}{\ordercanonical}

\newcommand{\precsubseteq}{\Subset}
\newcommand{\approxsubset}{\Subset}

\newcommand{\suitable}{suitable}%%Saul removed the $\Aunion(\card)$ part

%%%%%%%%%%%%%%%%%%%%%%%%%%%%%%%%%%%%%%%%%%%%
% Some additional symbols for the proof of completeness of Aunion(\card)
\newcommand{\provsub}{\subseteq_{\Gamma}}
\newcommand{\provle}{\le_{\Gamma}}
\newcommand{\provlt}{<_{\Gamma}}
\newcommand{\provsubDelta}{\subseteq_{\Delta}}
\newcommand{\provleDelta}{\le_{\Delta}}
\newcommand{\provltDelta}{<_{\Delta}}
\newcommand{\provlestrict}{\provlt}

\newcommand{\nprovle}{\nleq_{\Gamma}}
\newcommand{\provextended}{\preceq_{\Gamma}}
\newcommand{\provextendedstrict}{\prec_{\Gamma}}
\newcommand{\nprovextended}{\npreceq_{\Gamma}}
\newcommand{\provprecsubseteq}{\precsubseteq_{\Gamma}}
\newcommand{\nprovleDelta}{\nleq_{\Delta}}
\newcommand{\provextendedDelta}{\preceq_{\Delta}}
\newcommand{\provextendedstrictDelta}{\prec_{\Delta}}
\newcommand{\nprovextendedDelta}{\npreceq_{\Delta}}
\newcommand{\provprecsubseteqDelta}{\precsubseteq_{\Delta}}

\newcommand{\provsubstar}{\subseteq_{\Gamma^\star}}
\newcommand{\provlestar}{\le_{\Gamma^\star}}
\newcommand{\provltstar}{<_{\Gamma^\star}}
\newcommand{\provlestrictstar}{\provltstar}
\newcommand{\provextendedstar}{\preceq_{\Gamma^\star}}
\newcommand{\provextendedstrictstar}{\prec_{\Gamma^\star}}
%%%%%%%%%%%%%%%%%%%%%%%%%%%%%%%%%%%%%%%%%%%%

%%%%%%%%%%%%%%%%%%%%%%%%%%%%%%%%%%%%%%%%%%%%%%%%%%%%%%%%%%%%%%
% New commands needed just for notes at the end
\newcommand{\toAtLeast}{\rightarrow}
\newcommand{\toMore}{\xrightarrow[]{<}}
\newcommand{\toAll}{\xrightarrow[]{\subseteq}}

\newcommand{\chainAtLeast}{\toAtLeast \ldots \toAtLeast}
\newcommand{\chainMore}{\toAtLeast \ldots \toMore \ldots \toAtLeast}
\newcommand{\chainAll}{\toAll \ldots \toAll}

%%%%%%%%%%%%%%%%%%%%%%%%%%%%%%%%%%%%%%%%%%%%%%%%%%%%%%%%%%%%%%%
% For cross-referencing between main file and supplementary file
\makeatletter
\newcommand*{\addFileDependency}[1]{
  \typeout{(#1)}
  \@addtofilelist{#1}
  \IfFileExists{#1}{}{\typeout{No file #1.}}
}
\makeatother

\newcommand*{\myexternaldocument}[1]{
    \externaldocument{#1}
    \addFileDependency{#1.tex}
    \addFileDependency{#1.aux}
}

\myexternaldocument{AAAI_supplementary_material}
%%%%%%%%%%%%%%%%%%%%%%%%%%%%%%%%%%%%%%%%%%%%%%%%%%%%%%%%%%%%%%%

%%%%%%%%%%%%%%%%%%%%%%%%%%%%%%%%%%%%%%%%%%%%%%%%%%%%%%%%%%%%%%%
\newcommand{\noproof}{\rem}

\begin{document}

\maketitle

%%%%%%%%%%%%%%%%%%%%%%%%%%%%%%%%%%%%%%%%%%%%%%%%%%%%%%%%%%%%%%%%%%%%%%
% Abstract
%%%%%%%%%%%%%%%%%%%%%%%%%%%%%%%%%%%%%%%%%%%%%%%%%%%%%%%%%%%%%%%%%%%%%%

\begin{abstract}
    This paper presents the most basic logics for reasoning about the sizes of sets that admit either the union of terms or the intersection of terms.  That is, our logics handle assertions $\All{x}{y}$ and $\Atleast{x}{y}$, where $x$ and $y$ are built up from basic terms by either unions or intersections.  We present a sound, complete, and polynomial-time decidable proof system for these logics. 
    An immediate consequence of our work is the completeness of the logic additionally permitting $\More{x}{y}$.  The logics considered here may be viewed as efficient fragments of two logics which appear in the literature:  Boolean Algebra with Presburger Arithmetic and the Logic of Comparative Cardinality.
\end{abstract}

%%%%%%%%%%%%%%%%%%%%%%%%%%%%%%%%%%%%%%%%%%%%%%%%%%%%%%%%%%%%%%%%%%%%%%
\section{Introduction}
%%%%%%%%%%%%%%%%%%%%%%%%%%%%%%%%%%%%%%%%%%%%%%%%%%%%%%%%%%%%%%%%%%%%%%

Reasoning about the sizes of sets is common in both human and artificial reasoning.  It is also common, both in ``real-world'' human settings and in artificial systems, to reason in this way about unions and intersections of sets. In human settings, this use of union or intersection is often reflected by natural language phrases such as ``animals \emph{or} plants'' (for union), or ``mammals \emph{with} paws'' (for intersection).  For example, one might reason that if \emph{all cats are mammals that purr} and \emph{there are at least as many cats as purring things}, then it follows that \emph{all purring things are cats}.
% Removed per reviewer's comment :'(
%\emph{all plants are vegetarians with roots} and \emph{there are at least as many plants as vegetarians}, then it follows that \emph{all vegetarians are plants}.  
\rem{\hl{REVISED}
For example, one might reason that if \emph{there are at least as many birds as mammals} and \emph{all cats are mammals}, then it follows that \emph{there are at least as many birds as cats}.  
}
In this paper, we examine logics which capture the most basic fragments of reasoning about the sizes of finite sets alongside union or intersection.  Our two main logics handle assertions $\All{x}{y}$ (\emph{all $x$ are $y$}) and $\Atleast{x}{y}$ (\emph{there are at least as many $x$ as $y$}).  In the first logic, terms may be formed using union, whereas in the second logic, terms may be formed using intersection.

We emphasize that these fragments are the most basic because we wish to reflect one of the primary lessons of cognitive science:  That computationally light systems are the most cognitively plausible ones \cite{reasoning_about_sizes_of_sets}.
Accordingly, the two main logics we present are decidable in polynomial time.
%The two main logics we present are both minimally expressive and decidable in polynomial time.

The logics considered here are part of a broader enterprise of \emph{natural logic} \cite{Moss2015,reasoning_about_sizes_of_sets,vanBenthemHistory08}.  One of the main goals of this program is to demonstrate that components of natural language inference that can be modeled at all can be modeled by decidable logical systems.
Also, our work aims at obtaining \emph{complete} logical systems for
its fragments, with an eye toward efficient computer implementations.
Union and intersection are common points of interest and inspiration in natural language semantics~\cite{KeenanFaltz}, and reasoning about the sizes of sets has been a common subject of investigation in natural logic \cite{reasoning_about_sizes_of_sets,pratt_hartmann_2008}.
This paper contributes to this goal of natural logic: we provide complete axiomatizations of fragments of reasoning about sizes with either union or intersection, and we show that these logics are decidable in polynomial time.

\begin{relatedwork*}

There has been recent work at the confluence of logic and artificial intelligence on systems for reasoning about sizes, often including reasoning about union and intersection.
The closest system to ours in the literature is the polynomial-time decidable logic of $\AllNoArgs$, $\SomeNoArgs$,
$\AtleastNoArgs$, $\MoreNoArgs$, with set complement as a term forming
operation (but not including union or intersection), which was investigated in~\cite{syllogistic_cardinality_comparisons}.

% If one were to add set complement and propositional connectives $\andd$, $\orr$, and $\nott$ on top of our work, and also require all variables $x$ to have a finite interpretation (by including the predicate $Fin(x)$), then the resulting logic would be the Logic of Comparative Cardinality $\CardCompLogic$ in~\cite{DHH}.
If one were to add \emph{strict} cardinality comparison, set complement, and propositional connectives $\andd$, $\orr$, and $\nott$ on top of our work, the resulting logic would look like the Logic of Comparative Cardinality $\CardCompLogic$ in~\cite{DHH}.  Our logics are restricted further by requiring that all variables $x$ have a finite interpretation (in $\CardCompLogic$ one may express this using the predicate $Fin(x)$).
% since in our logics all variables $x$ are required to have a finite interpretation, we must consider but with all variables $x$ required to have a finite interpretation by prefixing with $Fin(x)$.
As shown in \cite{DHH}, $\CardCompLogic$ is $\NP$-complete, so what our logics lose in expressive power they recoup in efficiency.
Additionally, the method of proof in~\cite{DHH} is rather different from ours.

The logics presented in this paper may also be viewed as fragments of Boolean Algebra with Presburger Arithmetic ($\BAPA$) \cite{boolean_algebra_presburger_arithmetic}.
$\BAPA$ is a two-sorted logic, allowing both \emph{set relations} in the language of Boolean Algebra and \emph{numerical relations} in the language of Presburger Arithmetic. These two sorts are connected by a set cardinality function $|s|$ mapping sets to numbers. $\BAPA$ builds formulas from set relations and numerical relations via propositional connectives $\andd, \orr, \neg$, and quantifiers over sets and numbers.
The two main logics we present capture the sublogic of $\BAPA$ involving subsets, equality, \emph{nonstrict} cardinality comparison, and union or intersection terms, without propositional connectives or quantifiers.  Decidability in $\BAPA$ is also $\NP$-complete \cite{towards_efficient_bapa}, so again our logics present a more efficient fragment of the more expressive logic.

Reasoning about subsets and sizes alongside union and intersection is also relevant to the description logic community.  In \cite{extending_description_logic_ALC}, the authors incorporate the language of $\BAPA$ into the DL $\mathcal{ALC}$ in order to allow constraints on the cardinalities of concepts.
% reasoning about \emph{concepts} and their inclusions, sizes, with unions (or intersections).
\end{relatedwork*}

\begin{contribution*}
The main contribution of this paper is our axiomatization of reasoning about sizes of sets alongside either union or intersection, and the resulting polynomial decidability of these logics.  This efficiency result is in contrast to other logics involving reasoning of this kind (e.g. the $\NP$-complete logics $\BAPA$ and $\CardCompLogic$).
\end{contribution*}

\noproof{
\begin{organization*}
The rest of the paper is organized as follows. In Section~\ref{s:logics}, we describe formally our logics and their rules.  Our discussion focuses on logics allowing only binary (unnested) union and intersection terms, but we show in Section \ref{subsection-arbitraryterms} that we can interpret arbitrarily large terms in this restricted system.  In Section~\ref{s:comp_Aunion(card)} we establish the completeness of our logic $\Aunion(\card)$, making use of a representation lemma (Lemma~\ref{lemma-representation}), which in turn is proved in Section~\ref{s:representation}. In Section~\ref{s:completenes_intersections}, we show that the completeness of our logic $\Ainter(\card)$ is a straightforward consequence of the completeness of $\Aunion(\card)$. We furthermore show that our logics are decidable in polynomial time, and the details are given in Section~\ref{s:complexity}. Finally, in Section~\ref{s:future_work} we discuss several
avenues to extend our work, particularly towards the logics $\BAPA$ and $\CardCompLogic$.

In Appendix A, we provide proofs of completeness for some basic related logics involving union or intersection terms.
Proofs of particularly technical lemmas and propositions throughout the paper are included in Appendix B.  Finally, Appendix C contains an illustrative example of the primary construction used to prove the representation lemma.
\end{organization*}
}


%\hl{LM: in most CS papers, the end of the first section is a guide to the rest of the paper.  I personally don't always do this, and I also don't think that we need to do something just because others do it. But it is what your readers will expect.}


%%%%%%%%%%%%%%%%%%%%%%%%%%%%%%%%%%%%%%%%%%%%%%%%%%%%%%%%%%%%%%%%%%%%%%
\section{Our Logics}\label{s:logics}
%%%%%%%%%%%%%%%%%%%%%%%%%%%%%%%%%%%%%%%%%%%%%%%%%%%%%%%%%%%%%%%%%%%%%%

We focus our discussion on the two logics we call $\Aunion(\card)$ and $\Ainter(\card)$. We begin by defining the syntax and semantics of these systems.

\emph{Terms} in our syntax may either be \emph{basic terms} or \emph{binary terms}.  We use symbols $a, b, c, \ldots$ to denote basic terms and symbols $x, y, z, \ldots$ to denote terms that may be either basic or binary. If $a$ and $b$ are basic terms, then $a \cup b$ and $a \cap b$ are binary terms. Note that we do not allow nested terms like $(a\cup b)\cup c$ (we discuss this choice in the following subsection). 

The \emph{sentences} which we consider are $\All{x}{y}$ and $\Atleast{x}{y}$, where $x$ and $y$ are terms.
Note that in our logics we do not build up more complex sentences using propositional connectives or quantifiers; every sentence is one of these two operators applied to a pair of terms.

We now provide our terms and sentences with their semantics.  A \emph{model} $\mathcal{M}$ consists of a set $M$ (the \emph{universe} of $\mathcal{M}$), together with an interpretation function which assigns to each basic term $a$ a subset $\semantics{a} \subseteq M$.  We extend the interpretation function to binary terms $a \cup b$ and $a \cap b$ by $\semantics{a \cup b} = \semantics{a} \cup \semantics{b}$ and $\semantics{a \cap b} = \semantics{a} \cap \semantics{b}$.

Our sentences are given the expected semantics:
\[
\begin{array}{lclcl}
    \Model \models \All{x}{y} & \textrm{ iff } & 
        \semantics{x} \subseteq \semantics{y}\\
    \Model \models \Atleast{x}{y} & \textrm{ iff } & 
        |\semantics{x}| \ge |\semantics{y}|
\end{array}
\] 
For a set $\Gamma$ of sentences and another sentence $\varphi$, we have that $\Gamma \vDash \varphi$ if every \textit{finite} model $\mathcal{M}$ that satisfies the sentences in $\Gamma$ satisfies $\varphi$.  Note the finiteness assumption about $\mathcal{M}$; this is weaker than the usual logical consequence notion in logic. 

$\Aunion(\card)$ and $\Ainter(\card)$ differ in the terms and proof systems they use. Both logics employ natural-deduction style rules.  The full table of rules is shown in Figure~\ref{fig-rules}.  In particular, $\Aunion(\card)$ is the logic of $\All{x}{y}$ and $\Atleast{x}{y}$, with binary union terms, but no binary intersection terms, using rules (\axiom), (\barbara), (\mix), (\size), (\trans) in addition to (\unionl), (\unionr), and (\unionall).
$\Ainter(\card)$ is the analogous logic, but with binary intersection terms instead of binary union terms, and using rules $(\interl)$, $(\interr)$, and $(\interall)$ in place of $(\unionl)$, $(\unionr)$, and $(\unionall)$. Note that because we do not allow nested terms, the rules for union and intersection involve variables $a$, $b$, and $c$ representing basic terms, while the other rules involve variables $x$, $y$, and $z$ representing arbitrary terms. 

For a set $\Gamma$ of sentences in one of these logics and another such sentence $\varphi$, we say that $\varphi$ is \emph{provable} from $\Gamma$, written  $\Gamma \proves \varphi$, whenever $\varphi$ may be obtained from the sentences in $\Gamma$ from natural deduction via the rules for that logic.  
When we speak of the \emph{decidability} of a logic, we refer to the problem of determining whether or not $\Gamma \proves \varphi$ as a function of $\Gamma$ and $\phi$, when $\Gamma$ is finite. 

A logic is \emph{sound} if whenever $\Gamma \proves \varphi$ it follows that $\Gamma \models \varphi$.  We say a logic is \emph{complete} if the converse holds for finite $\Gamma$:  If 
$\Gamma \models \varphi$ then $\Gamma \proves \varphi$. We only consider finite $\Gamma$ in the definition of completeness because our logics $\Aunion(\card)$ and $\Ainter(\card)$ are not compact. %Arxiv: (see Section~\ref{s:supp:non-compact} in the Supplemental Material).
One may verify that each of the rules in Figure~\ref{fig-rules} is individually sound for our semantics.  Hence, the selected set of rules for each of our logics is sound.

\begin{remark}
The expected facts about set union and intersection are provable from the rules of $\Aunion(\card)$ and $\Ainter(\card)$.  For example, symmetry of $\cup$ follows:
$$
\infer[(\unionall)]{\All{(b\cup a)}{(a\cup b)}}{
\infer[(\unionr)]{\All{b}{(a\cup b)}}{} & 
\infer[(\unionl)]{\All{a}{(a\cup b)}}{}}
$$
%$\proves \All{(a \cup b)}{(b \cup a)}$.

The assumption that our models are finite is reflected in (\mix); this rule is not sound for infinite models.
\end{remark}

\begin{remark}
It is worth mentioning the related logics $\Aunion$ and $\Sunion$.  The logic $\Aunion$ is simply the $\AllNoArgs$-fragment of $\Aunion(\card)$.  $\Sunion$ extends this $\AllNoArgs$-fragment by admitting the sentence former $\Some{x}{y}$ (\emph{some $x$ is $y$}), with the semantics that $\Model \models \Some{x}{y}$ whenever $\semantics{x} \cap \semantics{y} \ne \emptyset$.  $\Sunion$ additionally %Arxiv:employs
borrows the $(\some)$, $(\conversion)$, and $(\darii)$ rules
from \cite{syllogistic_cardinality_comparisons}.  %Arxiv:in Figure~\ref{fig-rules-some}.  
$\Aunion$ and $\Sunion$ are both complete. %Arxiv:, as is shown in Section~\ref{s:supp:completeness_Aunion_Sunion}.

\noproof{
$\Aunion$ and $\Sunion$ are complete, as is shown in Section~\ref{s:supp:completeness_Aunion_Sunion}.  We encourage the reader to read these proofs of completeness as a warm-up for the completeness of $\Aunion(\card)$.  The completeness of $\Ainter$ follows from the completeness of $\Aunion$, using the trick described in Section~\ref{s:completenes_intersections}.  However, the completeness of $\Sinter$ does not follow straightforwardly.
}
\label{remark-related-logics}
\end{remark}

% For what follows, we will restrict our attention to those logics with union terms: $\Aunion$, $\Sunion$, and $\Aunion(\card)$.  We will discuss later the completeness of those logics with intersection terms and show, in particular, that the completeness of $\Ainter$ and $\Ainter(\card)$ follows from the completeness of their union counterparts.

\begin{figure*}[t!]
\begin{equation*}
\boxed{
\small
\begin{array}{c}
\begin{array}{ccc}
\\ 
\infer[(\axiom)]
    {\All{x}{x}}
    {} &
\infer[(\barbara)]
    {\All{x}{z}}
    {\All{x}{y} & \All{y}{z}}
\end{array}
\\ \\ 
\begin{array}{ccc}
\infer[(\mix)]
    {\All{y}{x}}
    {\All{x}{y} & \Atleast{x}{y}} &
\infer[(\size)]
    {\Atleast{y}{x}}
    {\All{x}{y}} &

\infer[(\trans)]
    {\Atleast{x}{z}}
    {\Atleast{x}{y} & \Atleast{y}{z}} \\ \\
    \hline \\
    
\infer[(\unionl)]
    {\All{a}{(a\cup b)}}
    {} &
    \infer[(\unionr)]
    {\All{b}{(a\cup b)}}
    {} &
\infer[(\unionall)]
    {\All{(a\cup b)}{c}}
    {\All{a}{c} & \All{b}{c}} \\ \\

\infer[(\interl)]
    {\All{(a\cap b)}{a}}
    {} &
\infer[(\interr)]
    {\All{(a\cap b)}{b}}
    {} &
\infer[(\interall)]
    {\All{a}{(b\cap c)}}
    {\All{a}{b} & \All{a}{c}} \\ \\
    
\end{array}
\end{array}
}
\end{equation*}
\caption{The rules for the logics $\Aunion(\card)$ and $\Ainter(\card)$.  
In addition to the rules above the line, $\Aunion(\card)$ uses $(\unionl)$, $(\unionr)$, and $(\unionall)$, whereas $\Ainter(\card)$ uses $(\interl)$, $(\interr)$, and $(\interall)$.
\rem{
Both $\Aunion(\card)$ and $\Ainter(\card)$ share (\axiom), (\barbara), (\mix), (\size), and (\trans).  $\Aunion(\card)$ additionally uses $(\unionl)$, $(\unionr)$, and $(\unionall)$, whereas $\Ainter(\card)$ uses $(\interl)$, $(\interr)$, and $(\interall)$.}
\label{fig-rules}}
\end{figure*}

\begin{subsection}{Logics with Arbitrarily Large Terms}
The reader might object that while in the Introduction we claim to capture basic reasoning about unions and intersections, the logics we address only allow \emph{binary} (unnested) unions and intersections. This restriction has the advantage of simplifying our proof of polynomial decidability in Section~\ref{s:complexity}. Although it may not initially be obvious, the completeness (and polynomial decidability, see Remark~\ref{remark-complexity} below) of $\Aunion(\card)$ and $\Ainter(\card)$ with arbitrarily large finite terms follows from their completeness with only binary terms.

To see this, we reduce arbitrary terms to binary terms in the natural way, iteratively replacing binary subterms of a complex term by fresh basic terms. For example, $a\cup ((b\cup c) \cup d)$ becomes $a\cup (t_1\cup d)$, which becomes $a\cup t_2$. 
Given $\Gamma$ and $\varphi$ with arbitrary terms, we define $\Gamma^\star$ and $\varphi^\star$ by reducing all terms appearing in $\Gamma$ and $\varphi$ in this way, and then adding additional sentences to $\Gamma^\star$: For every fresh term $t$ replacing a binary term, say $a \cup b$, in either $\Gamma$ \emph{or} $\varphi$, we include in $\Gamma^\star$ the sentences $\All{t}{(a \cup b)}$ and $\All{(a \cup b)}{t}$.  

One can check that this transformation does work, i.e. %Arxiv:We show in Section~\ref{s:supp:arbitrary_terms} in the Supplemental Material that this transformation does work, i.e.
\[\begin{array}{lclr}
     \Gamma \models \varphi & 
     \implies &
     \Gamma^\star \models \varphi^\star &\\
     
     & 
     \iff &
     \Gamma^\star \proves \varphi^\star &
     \textrm{(by Section~\ref{s:comp_Aunion(card)})}\\
     
     &
     \implies &
     \Gamma \proves \varphi & \textrm{(allowing arbitrary terms)}\\
     
\end{array}
\]

\noproof{
We illustrate this fact for union terms, although the same argument can be given mutatis mutandis for intersection terms.


Formally, we define an expanded logic $\Aunion_\mathrm{arb}(\card)$ as follows. We allow nested terms by changing our definition to an inductive one: a term is either a basic term or $(x\cup y)$, where $x$ and $y$ are terms. The semantics for terms is extended to nested terms in the obvious way. The sentences and rules of $\Aunion_\mathrm{arb}(\card)$ are the same as for $\Aunion(\card)$, except that they may now contain arbitrary nested terms.  We write $\provesarbitrary$ for the provability relation for $\Aunion_\mathrm{arb}(\card)$, reserving $\proves$ for the provability relation in $\Aunion(\card)$.

Let $\Gamma$ be a set of sentences in $\Aunion_\mathrm{arb}(\card)$, and let $\varphi$ be another such sentence.  We show that if $\Gamma \models \varphi$ then $\Gamma \provesarbitrary \varphi$. For any given sentence $\psi$ of $\Aunion_\mathrm{arb}(\card)$, we may obtain a new sentence $\psi^\star$ involving only binary union terms by recursively replacing binary unions in $\psi$ by fresh basic terms $t_i$ until there is only one union per argument remaining in $\psi$.  
Let $\Gamma^\star$ and $\varphi^\star$ be initially defined accordingly, modifying $\Gamma^\star$ as follows.  For every fresh term $t_i$ replacing binary union term, say $s_m \cup s_n$ in either $\Gamma$ \emph{or} $\varphi$, we include in $\Gamma^\star$ the sentences $\All{t_i}{(s_m \cup s_n)}$ and $\All{(s_m \cup s_n)}{t_i}$.  Note that $\Gamma^\star$ and $\varphi^\star$ involve only binary union terms.

It follows from $\Gamma \models \varphi$ that $\Gamma^\star \models \varphi^\star$, since a model $\Model$ of $\Gamma^\star$ is a model of both $\Gamma$ (and hence $\varphi$) as well as a model of those sentences added to $\Gamma^\star$ that ensure the intended semantics of the fresh terms $t_i$.  Assuming completeness of $\Aunion(\card)$ (shown in this paper), we have $\Gamma^\star \proves \varphi^\star$.  Let $\mathcal{T}^\star$ be a proof tree witnessing $\Gamma^\star \proves \varphi^\star$.  We construct a proof tree $\mathcal{T}$ for $\Gamma \provesarbitrary \varphi$ from $\mathcal{T}^\star$ by substituting back every previously introduced term $t_i$ in each sentence $\psi^\star$ in $\mathcal{T}^\star$ with the union it represents.  It remains to show that the premises of $\mathcal{T}$ are in $\Gamma$ (or are axioms), its conclusion is $\varphi$, and that each of the deductions in $\mathcal{T}$ follow by $\provesarbitrary$.  
Regarding the former facts, any premise of $\mathcal{T}^\star$ is either a sentence in $\Gamma$ with terms substituted, or is a new sentence that we added to $\Gamma^\star$.  A premise that simply has terms substituted will have the respective unions substituted back in for each $t_i$, and hence the corresponding premise of $\mathcal{T}$ is in $\Gamma$.  If a premise conclusion of $\mathcal{T}^\star$ is a sentence we added to $\Gamma^\star$, it is either $\All{t_i}{(s_m \cup s_n)}$ or $\All{(s_m \cup s_n)}{t_i}$.  Either way, after substituting back $s_m \cup s_n$ for $t_i$, we obtain $\All{(s_m \cup s_n)}{(s_m \cup s_n)}$, which is an instance of $(\axiom)$.  Similarly, after substitution the conclusion of $\mathcal{T}$ is $\varphi$.  As for the latter fact, each deduction still follows in $\mathcal{T}$ via the same rule that was used in that position of $\mathcal{T}^\star$.
}

\label{subsection-arbitraryterms}
\end{subsection}





\section{Completeness of $\Aunion(\card)$}\label{s:comp_Aunion(card)}
\label{section3}

In this section, we prove the completeness of the logic $\Aunion(\card)$.  We will return to $\Ainter(\card)$ in Section~\ref{s:completenes_intersections}. First, we present a representation lemma that will later be used to build a model of any finite set $\Gamma$ of sentences in $\Aunion(\card)$.
In logics with sentential negation $\nott$ and a proof rule of \emph{reductio ad absurdum}, such a representation lemma is tantamount to completeness.  But $\Aunion(\card)$ has neither of these, and so more work will be needed.  This extra work will be presented subsequently.


\subsection{Representation Lemma}
\label{subsection-representation}

Since we restrict our attention to binary terms, we may model the $\AllNoArgs$- and $\AtleastNoArgs$-relationships provable from $\Gamma$ by corresponding relations on pairs of basic terms; this is the content of our representation lemma.  We represent the problem in this way in order to argue that model-building in $\Aunion(\card)$ can be done in polynomial time.

In order to state this lemma, we must first define the appropriate relations on pairs.
Let $BT$ be a finite set of basic terms.  We fix a linear order $<$ on $BT$.
We define the set of pairs under discussion as $\Pairs = \set{(a, b) : a, b \in BT \textrm{ and } a \leq b}$.

\begin{definition}
A \emph{\suitable}
pair of relations on $\Pairs$ 
is a pair of relations $(\preceq, \precsubseteq)$ such that for all pairs $p$ and $q$, and all basic terms $a$, $b$, $c$, and $d$,

\begin{enumerate}
\item $\preceq$ and $\precsubseteq$ are preorders on $\Pairs$.
(That is, they are reflexive and transitive.)
\item   $\preceq$ is linear:
 either $p \preceq q$ or $q \preceq p$ (and possibly both).
 \item If $a < b$ in the fixed ordering on $BT$, then $(a,a) \precsubseteq (a,b)$.  If $b < a$, then $(a,a) \precsubseteq (b,a)$. 
 \item If $(a,a) \precsubseteq (c,d)$ and $(b,b) \precsubseteq (c,d)$ and $a < b$,
 then $(a,b) \precsubseteq (c,d)$.
\item If $p \precsubseteq q$, then $p\preceq q$.
\item If $p \precsubseteq q$ and $q\preceq p$, then $q \precsubseteq p$.
\end{enumerate}

%we let $<$ be the \emph{strict part} of $\leq$:
%  \[ x < y \quadiff x \leq y \mbox{ but } y \nleq x\]

For $p,q \in \Pairs$, we often write $\prec$ to denote the \emph{strict part} of $\preceq$, i.e. $p \prec q$ whenever $p \preceq q$ but $q \not \preceq p$.
\label{def-suitable-pair-first}
\end{definition}

Here is an example:
Let $\Nodel$ be any model, and define $(a,b) \preceq (c,d)$ iff 
$|\semantics{a} \cup \semantics{b}| \leq
|\semantics{c} \cup \semantics{d}|$, and 
 $(a,b) \precsubseteq (c,d)$ iff 
$\semantics{a} \cup \semantics{b} \subseteq
\semantics{c} \cup \semantics{d}$.
This gives a {\suitable} pair of relations.
Our representation lemma shows that every {\suitable} pair of relations arises in
this way.  In order to build such models, we use families of sets corresponding to basic terms, defined as follows:

\begin{definition}
 A \emph{$BT$-family} is a family of finite sets
$S = (S_a)_{a\in BT}$.
For a $BT$-family $S$, we write $S_{a,b}$ for $S_a \cup S_b$.  We also write $s_{a,b}$
for the number  $|S_a\cup S_b|$.  
We also write $s_a$ for $s_{a,a}$ (i.e., for $|S_a|$).

For $p\in\Pairs$, say with $p= (a,b)$, we often write $S_p$ instead of $S_{a,b}$.
\end{definition}

\begin{definition}\label{def-family-model}
A $BT$-family $S$ is \emph{$\preceq$-preserving} if, for all $p,q \in \Pairs$, $p \preceq q$ implies that $s_{p} \le s_{q}$.  $S$ is \emph{$\preceq$-reflecting} if $s_{p} \le s_{q}$ implies that $p \preceq q$.

Similarly, $S$ is \emph{$\precsubseteq$-preserving}
if $p \precsubseteq q$ implies that $S_{p} \subseteq S_{q}$, and 
$S$ is \emph{$\precsubseteq$-reflecting} if 
$S_{p} \subseteq S_{q}$ implies that $p \precsubseteq  q$. 

Every $BT$-family $S$ determines
a  model $\Model$ as follows.  Its universe $M$ is  $\bigcup_{a} S_a$.
This set is finite.
For a basic term  $a$, let $\semantics{a} = S_{a}$. 
So  for union terms $a\cup b$, we
automatically
have $\semantics{a\cup b} = S_{a} \cup S_{b} = S_{a,b}$.
\end{definition}

% Our hope is that $S$ preserves and reflects $\preceq$ and $\precsubseteq$, i.e. that we have a model whose terms preserve and relect the behavior of $\mbox{\sf All}$-, $\mbox{\sf More}$-, and $\mbox{\sf Atleast}$-sentences with term union.  
%We now state our representation lemma and use it to prove the completeness of  $\Aunion(\card)$.

We may now state our representation lemma.  In the next section, we use it to prove the completeness of $\Aunion(\card)$.

\begin{lemma} [Representation Lemma]
Let $(\preceq, \precsubseteq)$ be a suitable pair of relations on $\Pairs$.
Then there is a $BT$-family of sets $S$
such that for all $p,q\in\Pairs$,
\begin{align}
\label{goal-main1-first}
p \preceq q \quadiff 
 s_{p}\leq s_q  \\
\label{goal-main2-first} 
 p \precsubseteq  q \quadiff 
S_{p}\subseteq S_{q}
 \end{align}
 That is, $S$ preserves and reflects $\preceq$ and $\precsubseteq$.
 \label{lemma-representation}
 \end{lemma}

\subsection{Completeness}
\label{section-completeness}

\begin{theorem}
The logic $\Aunion(\card)$ is complete.
\label{theorem-completeness-Aunioncard}
\end{theorem}

The rest of this section is devoted to the proof.  We want to show that if $\Gamma$ is a finite set of sentences and $\Gamma \not \proves \varphi$, then $\Gamma \not \models \varphi$.  Our plan is to use Lemma \ref{lemma-representation} to build a model of $\Gamma$ where $\varphi$ is false. Note that we may assume $\varphi$ has the form $\Atleast{(a\cup b)}{(c\cup d)}$ or $\All{(a\cup b)}{(c\cup d)}$ for some basic terms $a$, $b$, $c$, and $d$; for example, $\Atleast{a}{c}$ is provably equivalent to $\Atleast{(a\cup a)}{(c\cup c)}$.

We first define relations $\provle$ and $\provsub$ on $\Pairs$ by

\begin{itemize}%[parsep=0pt,partopsep=0pt]
    \item $(a,b) \provle (c,d)$ iff $\Gamma \vdash  \Atleast{(c \cup d){(a \cup b)}}$
    
    \item $(a,b) \provsub (c,d)$ iff $\Gamma \vdash \All{(a \cup b)}{(c \cup d)}$
\end{itemize}
for all $(a,b)$ and  $(c,d)$ in $\Pairs$.  

%\footnote{LM: Here's an explanation of \hl{my main notational choice, in case we want to discuss and/or change it}.  I resisted the temptation to import pairs into our original syntax.   So when it comes to sentences, I wrote $ \All{(a \cup b)}{(c \cup d)}$ and never ``$\All{x}{y}$ where $x$ and $y$ correspond to $p = (a,b)$ and $q=(c,d)$.''  I figure that this second usage would be too much for people.  Do you agree? }

Note that $(\provle, \provsub)$ has all the properties of a suitable pair, except for linearity of $\provle$.  We use the following fact to extend $\provle$ to a linear preorder.  The statement is a bit stronger than a straightforward linearization, for purposes that will become clear at the end of this section.  %Arxiv:We prove this fact in Section~\ref{s:supp:completeness-Aunioncard}.

\begin{proposition}
\label{proposition-linearization}
Let $x^\star$ be a fixed pair.  We can extend $\provle$ to a linear preorder $\provextended$ over $\Pairs$ such that for all $y \in \Pairs$, if $y \not \provle x^\star$, then $x^\star \provextendedstrict y$.
\end{proposition}

If $\varphi$ is $\Atleast{(a \cup b)}{(c \cup d)}$, let $x^\star$ be the pair $(a, b)$. Otherwise, choose $x^\star$ arbitrarily. Let $\provextended$ be the linear preorder obtained from Proposition~\ref{proposition-linearization} using this $x^\star$.  One can verify the following proposition. %Arxiv:We verify the following proposition in Section~\ref{s:supp:completeness-Aunioncard}.

\begin{proposition}
\label{proposition-suitablepair}
$(\provextended, \provsub)$ is a suitable pair of relations on $\Pairs$.
\end{proposition}

We may now apply Lemma~\ref{lemma-representation}. Take $BT$ to be the set of basic terms appearing in $\Gamma$ and $\varphi$.  There is a $BT$-family of sets $S = (S_a)_{a\in BT}$ such that for all 
$(a,b),(c,d) \in \Pairs$,
\begin{equation}
\label{arrows}
\begin{array}{cc}
\Gamma \proves \All{(a \cup b)}{(c \cup d)} & 
\Gamma \proves \Atleast{(c \cup d)}{(a \cup b)}\\
\Updownarrow &
\Downarrow\\
(a,b) \provsub (c,d) &
(a,b) \provextended (c,d)\\
\Updownarrow &
\Updownarrow\\
S_a \cup S_b \subseteq S_c \cup S_d &
|S_a \cup S_b| \le |S_c \cup S_d|
\end{array}
\end{equation}

The $\Updownarrow$ on the upper-left  is the definition of $\provsub$.
The $\Downarrow$ on the upper-right comes from the fact that $\provextended$ is a linearization (hence an extension) of $\provle$.  The lower $\Updownarrow$-arrows follow from the representation lemma.

Let $\Model$ be the model determined by the $BT$-family $S$. By (\ref{arrows}), $\Model$ satisfies the $\AllNoArgs$- and $\AtleastNoArgs$-sentences in $\Gamma$.  We would like to ensure as well that $\Model$ does not satisfy $\varphi$.  We have two cases:

\begin{itemize}
    \item $\varphi$ is $\All{(a \cup b)}{(c \cup d)}$.
    Since $\Gamma \not \proves \All{(a \cup b)}{(c \cup d)}$, by (\ref{arrows}) we have $\semantics{a \cup b} = S_a \cup S_b \not \subseteq S_c \cup S_d = \semantics{c \cup d}$.  So $\Model \not \models \varphi$.
    
    \item $\varphi$ is $\Atleast{(a \cup b)}{(c \cup d)}$.
    Recall that when we applied Proposition~\ref{proposition-linearization}, we took $x^\star$ to be $(a, b)$.  Now take $y$ to be $(c, d)$.  Since $\Gamma \not \proves \Atleast{(a \cup b)}{(c \cup d)}$, we have $(c, d) \not \provle (a, b)$ by definition.  Thus, by Proposition~\ref{proposition-linearization}, $(a, b) \provextendedstrict (c, d)$.  So by (\ref{arrows}), $|\semantics{a \cup b}| = |S_a \cup S_b| < |S_c \cup S_d| = |\semantics{c \cup d}|$.  So $\Model \not \models \varphi$.
    
\end{itemize}

%\textbf{Old treatment of model building and completeness separately now hidden; I tried to streamline the proof and add transition phrases so that it is easier to follow}

\rem{ %8/9/2019
\begin{theorem}
\label{theorem-consistent-satisfiable}
    If $\Gamma$ is consistent, it has a model: if  $\provle$ and $\provsub$ are as in  Definition~\ref{def-provle},
     $\provextended$ is any linearization of $\provle$,  $S =(S_a)_{a\in BT}$ is any representation of $(\provextended,\provsub)$,
     and $\Model$  defined from $S$ as in Definition~\ref{def-family-model}, then $\Model\models\Gamma$.
\end{theorem} 

\begin{proof} 
%Use Definition~\ref{def-provle}
%to define $\provle$ and $\provsub$ from $\Gamma$. %
%Consider the preorder $(\Pairs,\provle)$.   
 %Use
%Proposition~\ref{prop-extension-first} to extend $\provle$ to $\provextended$.

   In the notation of our statement,   
 Lemma~\ref{lemma-representation} provides
 a $BT$-family of sets $(S_a)_{a\in BT}$ such that for all 
$(a,b),(c,d) \in \Pairs$:

\begin{equation}\label{arrows}
\begin{array}{lcccc}
\Gamma \proves \All{(a \cup b)}{(c \cup d)} & \iff & 
    (a,b) \provsub (c,d) & \iff & 
    S_a \cup S_b \subseteq S_c \cup S_d\\
\Gamma \proves \Atleast{(c \cup d)}{(a \cup b)} & \implies & 
    (a,b) \provextended (c,d) & \iff & 
    |S_a \cup S_b| \le |S_c \cup S_d| \\
\Gamma \proves \More{(c \cup d)}{(a \cup b)} & \implies & 
    (a,b) \provextendedstrict (c,d) & \iff & 
    |S_a \cup S_b| < |S_c \cup S_d|\\
& (\textrm{see below}) & & (\textrm{by Lemma \ref{lemma-representation}}) & 
\end{array}
\end{equation}
The $\iff$ on the left  is the definition of $\provsub$.
The first $\implies$ on the left
second comes from the fact that  $\provextended$ is a linearization (hence an extension) 
of 
$\provle$.
For the  second $\implies$, suppose that 
$\Gamma \vdash \More{(a\cup b)}{(c\cup d)}$.
We have  $\Gamma \vdash \Atleast{(a\cup b)}{(c\cup d)}$, by $\proverule{(more-atleast)}$.
Write $p$ for the pair $(a,b)$, and $q$ for $(c,d)$.
So $q \provle p$.
We cannot have $p \provle q$, since that would mean $\Gamma \vdash \Atleast{(c\cup d)}{(a\cup b)}$,
and  $\Gamma$ would be inconsistent.   So $q \provltstar p$.
By the definition of ``linearization,'' we have
  $q \provextendedstrict p$.

 We also have a model $\Model$.
Let $\psi\in\Gamma$, so that $\Gamma\proves \psi$.
By the $\implies$
 implications in (\ref{arrows}),  $\Model\models\psi$.
 % $\Model\models\Gamma$. 
\end{proof}
 
 \subsection{Completeness}
 \label{section-completeness}
 
\begin{theorem} If $\Gamma$ is a finite consistent set and $\Gamma\not\proves\phi$,
 then there is a model of $\Gamma$ where $\phi$ is false.
 \end{theorem}
 
 The rest of this section is devoted to the proof.
 We write $x$ as $a\cup b$ and $y$ as $c\cup d$.
 
 \paragraph{Case 1: $\phi$ is of the form $\All{(a\cup b)}{(c\cup d)}$.} 
 In this case, the work is already done. Indeed, let $\Model$ be the model of $\Gamma$ from
 the previous section. 
 Since $\Gamma\not\proves\phi$, $S_a\cup S_b$ is not a subset of $S_c\cup S_d$,
  by (\ref{arrows}).  So $\Model\not\models\phi$.

 \paragraph{Case 2: $\phi$ is of the form $\Atleast{(a\cup b)}{(c\cup d)}$.} 
 
 This time, we need a sharper form of Proposition~\ref{prop-extension-first}.
 
 \begin{proposition}
\label{prop-extension-second}
Let $(P,\leq_0)$ be a finite preorder.  
Let $x^*\in P$.  Then there is a linearization $\leq_1$ of $P$ 
such that for all $y$, if $y \nleq_0 x^*$, then $x^* <_1 y$.
%and
%(d) if $x^*\nless_0 y^*$, then $y^* \leq_1 x^*$.
\end{proposition}

 
Use Definition~\ref{def-provle}
to define $\provle$ and $\provsub$ from $\Gamma$.  
Consider the preorder $(\Pairs,\provle)$.   
 Use
Proposition~\ref{prop-extension-second},  taking $x^*$ to be the pair $(a,b)$, to 
 extend $\provle$ to a new relation $\provextended$.
 By Proposition~\ref{proposition-suitable-pair}, $(\provextended,\provsub)$  is a suitable pair.
Use 
  Lemma~\ref{lemma-representation} to represent the pair by a 
  $BT$-family of sets $(S_a)_{a\in BT}$, and then 
  use Definition~\ref{def-family-model} to define $\Model$ from this family.
As in  
 the proof of Theorem~\ref{theorem-consistent-satisfiable},
 $\Model\models\Gamma$.
We need to show that $\Model\not\models\phi$.
We have $(c,d) \nprovle (a,b)$.  
Take $y$ to be $(c,d)$ in 
  Proposition~\ref{prop-extension-second}.
Thus $(a,b) \provextendedstrict (c,d)$.   So by (\ref{arrows}), 
 $ |S_a \cup S_b| < |S_c \cup S_d|$.


%Now we followUse Lemma~\ref{lemma-representation} to get a 

 
  \paragraph{Case 3: $\phi$ is of the form $\More{(a\cup b)}{(c\cup d)}$.} 

 We start by doing what we did in Case 2:  
 Use Definition~\ref{def-provle}
to define $\provle$ and $\provsub$ from $\Gamma$.  
Consider the preorder $(\Pairs,\provle)$.   
 Use
Proposition~\ref{prop-extension-second},  taking $x^*$ to be the pair $(a,b)$, and thereby
 extending $\provle$ to a new relation $\provextended$.
If $(a,b) \provextended (c,d)$, then the model from Case 2 falsifies $\phi$.
So we may assume that $(c,d) \provextendedstrict (a,b)$.
Let 
\[ M = \set{(e,f)\in \Pairs : (e,f) \provextended (a,b) \mbox{ and } \Gamma\not\proves \More{(a\cup b)}{(e\cup f)}}.\]
Note that $(c,d), (a,b)\in M$.
Let $\Delta$ be the following set of sentences:
\begin{itemize}
\item $\All{x}{y}$, whenever this sentence is provable from $\Gamma$.
\item $\Atleast{(e\cup f)}{(g\cup h)}$, when $(e,f)$ and $(g,h)$ belong to $M$.
\item  $\Atleast{(e\cup f)}{(g\cup h)}$, when neither $(e,f)$ and $(g,h)$ belong to $M$,
and when $(g,h) \provextended (e,f)$.
\item $\Atleast{(e\cup f)}{(g\cup h)}$,  when $(e,f)\notin M$ and $(g,h)\in M$ and  $(a,b) \provextended (e,f)$.
\item $\Atleast{(e\cup f)}{(g\cup h)}$,  when $(e,f)\in M$ and $(g,h)\notin M$, and $(g,h) \provextended (a,b)$. 
\end{itemize}

Use Definition~\ref{def-provle}
to define $\provleDelta$ and $\provsubDelta$ from $\Delta$. 



\begin{claim}
\label{claimInMore}
\begin{enumerate}
\item $\provleDelta$  is a linear preorder, so $(\provleDelta,\provsubDelta)$ is a suitable pair.
\item  If  $(g,h)\provleDelta (e,f)$, then $\Atleast{(e\cup f)}{(g\cup h)}$ belongs to $\Delta$.
(The converse is immediate.)
\item If  $\Gamma\proves\Atleast{(e\cup f)}{(g\cup h)}$, then $(g,h)\provleDelta (e,f)$.
\item If  $\More{(e\cup f)}{(g\cup h)}$ belongs to $\Gamma$, then  $(g,h)\provltDelta (e,f)$.
\end{enumerate}
\end{claim}

\newcommand{\proofClaimInMore}{
\begin{enumerate}
\item It is easy to show that $\leq_2$ is reflexive.
The transitivity takes some case-by-case checking, and we omit this.
The work would be nearly the same as what we saw in the proof  of Proposition~\ref{prop-extension-second}.\footnote{LM:
I can of course write out the 16 cases.  It also would be possible to unify the arguments for the
work in this result with what we saw in  the proof  of Proposition~\ref{prop-extension-second}.}

\medskip

We discuss the linearity.   Let $(e,f), (g,h)\in \Pairs$.
If neither of these belong to $M$ or if both belong, then by linearity of $\provextended$, we
see that either $(e,f) \provleDelta (g,h)$ or $(g,h) \provleDelta (e,f)$ (or both).
So we may assume that $(e,f) \in M$ and $(g,h)\notin M$.
If  $(g,h) \provextended (a,b)$, then we have  $\Atleast{(e\cup f)}{(g\cup h)}$ in $\Delta$, so that 
$(g,h) \provleDelta  (e,f)$.   And if  $(g,h) \nprovextended (a,b)$, then 
by linearity, $(a,b) \provextended (g,h)$.  
By our fourth condition in the definition of $\Delta$, that set contains   $\Atleast{(g\cup h)}{(e\cup f)}$.
So  $(e,f) \provleDelta (g,h) $.

Since $\provle$ is a linear preorder, it is a linearization of itself.
Thus $(\provleDelta,\provsubDelta)$ is a suitable pair, by Proposition~\ref{proposition-suitable-pair}.


\item If $\Atleast{(e\cup f)}{(g\cup h)}$ belongs to $\Delta$, then clearly  $(g,h)\provleDelta (e,f)$.
In the other direction, we argue by induction on proofs from $\Delta$ that the following two facts hold:
\begin{itemize}
\item[(a)] If $\Delta \proves \Atleast{(e\cup f)}{(g\cup h)}$, then  $\Atleast{(e\cup f)}{(g\cup h)}$ belongs to $\Delta$.
\item[(b)] If $\Delta \proves \All{(e\cup f)}{(g\cup h)}$, then either $ \All{(e\cup f)}{(g\cup h)}$ belongs to $\Delta$, or
else both $\Delta\proves\All{(g\cup h)}{(e\cup f)}$ and
 $\Atleast{(g\cup h)}{(e\cup f)}$ belongs to $\Gamma$.
\end{itemize}

We need (b) in the induction step for (\size).

Before we begin, here is an observation.   If $\Delta$ contains $\All{x}{y}$ and $\All{y}{z}$, 
then it also contains $\All{x}{z}$.   The reason:  by definition, both of these are provable from $\Gamma$. 
Hence so is $\All{(a\cup b)}{c}$.  Thus, $\Delta$ contains this sentence.

The base case  (sentences in $\Delta$) is trivial.
For inferences with no premises in $\Delta$, that is, when we use (\unionl) of (\unionr), 
we use the fact that $\Delta$ contains the $\AllNoArgs$-sentences provable from $\Gamma$.
 

Most of the induction steps are routine.
Here is the work for (\barbara).     Assume that we have a derivation
$\Delta\proves\All{x}{z}$ ending with an application of (\barbara), so that 
$\Delta\proves \All{x}{y}$ and $\Delta\proves \All{y}{z}$ via shorter derivations.
So by induction hypothesis, $\Delta$ contains $\Atleast{y}{x}$ and $\Atleast{z}{y}$.
By (\trans),  $\Delta$ contains $\Atleast{z}{x}$.

 Next, consider (\unionall).
   Assume that we have a derivation
$\Delta\proves\All{(a\cup b)}{c}$ with an application of (\unionall), so that 
$\Delta\proves \All{a}{c}$ and $\Delta\proves \All{b}{c}$ via shorter derivations.
We have four cases.   Suppose first that $\Delta$ contains $\All{a}{c}$ and $\All{b}{c}$.
As we know from our observation before the induction started, 
$\Delta$ contains $\All{(a\cup b)}{c}$.
Continuing, suppose that $\Delta$ contains $\All{a}{c}$ and also that  $\Delta\proves\All{c}{b}$,
and that $\Delta\proves\Atleast{c}{b}$. \marginpar{CHECK that this is what's used}
Using (\size),  $\Delta\proves\All{a}{b}$.  
So $\Delta\proves\All{(a\cup b)}{b}$.
Then $\Gamma\proves \Atleast{b}{(a\cup b)}$.   And so by (\trans),  $\Gamma\proves \Atleast{c}{(a\cup b)}$.
Using (\unionr) to get $\All{b}{(a\cup b})$ and then (\barbara), $\Delta\proves\All{c}{(a\cup b)}$.

The other two cases concerning (\unionall) are similar.  \marginpar{I need to check them.}



Let us discuss (\mix).   Suppose that 
$\Delta \proves \All{(e\cup f)}{(g\cup h)}$ 
by a proof which ends with (\mix), so that 
$\Delta \proves \Atleast{(g\cup h)}{(e\cup f)}$
and  $\Delta \proves \All{(g\cup h)}{(e\cup f)}$.
The induction hypothesis applies to the first fact and tells us that 
$ \Atleast{(g\cup h)}{(e\cup f)}$ belongs to $\Delta$, as desired.

 Since $\Delta$ has no $\MoreNoArgs$ sentences, 
none of the rules involving  $\MoreNoArgs$  can be used in derivations. 

\item  Suppose that    $\Gamma\proves \Atleast{(e\cup f)}{(g\cup h)}$.
We show that this same sentence belongs to $\Delta$. 
Notice that 
$(g,h) \provle (e,f)$.
Since $\provextended$ is a linearization of $\provle$,
$(g,h) \provextended (e,f)$.


We 
have four cases as to whether  each of
$(e,f)$ and $(g,h)$ belongs to $M$ or not.

If both $(e,f)$ and $(g,h)$ belong to $M$, then since 
 $\Gamma\proves \Atleast{(e\cup f)}{(g\cup h)}$, our sentence belongs to $\Delta$.

Suppose that neither  $(e,f)$ nor $(g,h)$ belong to $M$.
Since  $\Gamma\proves \Atleast{(e\cup f)}{(g\cup h)}$,  
 our sentence belongs to $\Delta$.

The next, and most interesting, case: $(e,f)\notin M$, $(g,h)\in  M$.
We claim that 
$(a,b) \provextended (e,f)$, and this will finish this case.
For if not, we have $(e,f) \provextended (a,b)$, by linearity.
Since $(e,f)\notin M$, $\Gamma\proves  \More{(a\cup b)}{(e\cup f)}$.
By  (\morel), $\Gamma\proves \More{(a\cup b)}{(g\cup h)}$.
And this contradicts $(g,h)\in M$.


The final case: $(e,f)\in M$, $(g,h)\notin  M$.
We claim that 
$(g,h) \provextended (a,b)$, and this will finish this case.
If not, then $(a,b) \provextendedstrict (g,h)$ by linearity.  
As we have seen, $(g,h) \provextended (e,f)$.
So $(a,b) \provextendedstrict (e,f)$.
This contradicts $(e,f)\in M$.




\item  Let   $\More{(e\cup f)}{(g\cup h)}$ belong  to $\Gamma$.
By the last part and (\moreatleast),  $(g,h)\provleDelta (e,f)$.
We show that we do not have  $(e,f)\provleDelta  (g,h)$.
Suppose towards a contradiction that we do have  $(e,f)\provleDelta  (g,h)$.
In view of part (2) of this result, $\Delta$ contains
both 
$\Atleast{(e\cup f)}{(g\cup h)}$
and also 
$\Atleast{(g\cup h)}{(e\cup f)}$.
Looking back at the definition of $\Delta$,  we 
have several cases as to whether 
$(g,h)$ and $(e,f)$ belong to $M$ or not.

 
The most interesting case is the first one: $(e,f)$ and $(g,h)$ both belong to $M$.
Since $(e,f)\in M$, $(e,f)\provextended (a,b)$.
We claim that   $(e,f)\provle (a,b)$.  
[Here is the reason, if $(e,f)\nprovle (a,b)$, then 
by Proposition~\ref{prop-extension-second}, we have $(a,b)\provextendedstrict (e,f)$.
Since $(e,f)\in M$, we know that $(e,f)\provextended (a,b)$.  
So we have a contradiction.]   
That is, $\Gamma\proves \Atleast{(a\cup b)}{(e\cup f)}$.
By  (\morer), $\Gamma\proves \More{(a\cup b)}{(g\cup h)}$.  
And this contradicts $(g,h)\in M$.

The second case is when neither $(e,f)$ nor $(g,h)$ belong to $M$.
This time, we have $(e,f)\provextended (g,h)$ and $(g,h)\provextended (e,f)  $.
However, since $\More{(e\cup f)}{(g\cup h)}$ belongs  to $\Gamma$,
we have $(g,h)\provextendedstrict (e,f)  $.
(The argument is one we saw in the proof of Theorem~\ref{theorem-consistent-satisfiable}.)
So we have a contradiction.

The last case is  (without loss of generality) 
when $(g,h)\in M$, $(e,f)\notin M$, $(a,b) \provextended (e,f)$, and $(e,f) \provextended (a,b)$.
In this case, the argument which we saw in part (3) of this result gives the needed contradiction.

\end{enumerate}
This completes the proof. 
Incidentally, parts (3) and (4) in  this proof are the only places in our work where we use (\morel) and (\morer).
  }


  
To conclude,  use Lemma~\ref{lemma-representation} on the suitable pair $(\provleDelta,\provsubDelta)$.
There is a $BT$-family of sets $S= (S_a)_{a\in BT}$ with the properties in (\ref{arrows}), but 
with $\Delta$ replacing $\Gamma$.  We then turn this family $S$ into a model $\Model$; see Definition~\ref{def-family-model}.
By  Claim~\ref{claimInMore}, $\Model\models\Gamma$.
In more detail, for a sentence in $\Gamma$ of the form $\All{(e\cup f)}{(g\cup h)}$, this sentence is in $\Gamma$
and hence in $\Delta$, and thus $(e,f) \provsubDelta (g,h)$.   
For sentences using $\AtleastNoArgs$ or $\MoreNoArgs$,  use Claim~\ref{claimInMore}, parts (3) and (4).
Finally, 
because $(c,d) \provextended (a,b)$, we have $(a,b) \provleDelta (c,d)$.  In $\Model$, $|\semantics{c} \cup \semantics{d}| \geq
|\semantics{a} \cup \semantics{b}|$.
Thus,  $\Model\not\models\phi$.
}


%{\bf old stuff hidden}
\rem{
\hl{OLD STUFF AHEAD}
\section{Completeness of $\Aunion(\card)$}\label{s:comp_Aunion(card)}

In this section, we prove that every finite set $\Gamma$ which is consistent in 
$\Aunion(\card)$ has a model.  In logics with sentential negation and a
rule of \emph{reductio ad absurdum}, this is tantamount to the completeness
of the logic.  But $\Aunion(\card)$ has neither of these,  and so more 
work will be needed afterwards.


Up until now in this paper, our basic terms were symbols like $a$, $b$, $\ldots$.  
Let $BN$ be the set of basic nouns.  We fix a linear order $<$ on $BN$.


\subsection{Every Finite Consistent Set has a Model}

\begin{definition}
A \emph{\suitable}
pair of relations on $\Pairs$ 
is a pair of relations $(\preceq, \precsubseteq)$ such that for all
pairs $p$ and $q$, and all basic nouns $a$, $b$, $c$, and $d$:

\footnote{LM: we only have one kind of
suitable pair, so I suggest dropping the $\Aunion(\card)$.}

\begin{enumerate}
\item $\preceq$ and $\precsubseteq$ are preorders on $\Pairs$.
(That is, they are reflexive and transitive.)
\item   $\preceq$ is linear:
 either $p \preceq q$ or $q \preceq p$.
 \item If $a < b$, then $(a,a) \precsubseteq (a,b)$.  If $b < a$, then $(a,a) \precsubseteq (b,a)$. 
 \item If $(a,a) \precsubseteq (c,d)$ and $(b,b) \precsubseteq (c,d)$ and $a < b$,
 then $(a,b) \precsubseteq (c,d)$.
\item If $p \precsubseteq q$, then $p\preceq q$.
\item If $p \precsubseteq q$ and $q\preceq p$, then $q \precsubseteq p$.
\end{enumerate}
\label{def-suitable-pair-first}
\end{definition}

Here is an example:
Let $\Nodel$ is any model, and define $(a,b) \preceq (c,d)$ iff 
$|\semantics{a} \cup \semantics{b}| \leq
|\semantics{c} \cup \semantics{d}|$, and 
 $(a,b) \precsubseteq (c,d)$ iff 
$\semantics{a} \cup \semantics{b} \subseteq
\semantics{c} \cup \semantics{d}$.
This gives a suitable pair of relations.
Lemma~\ref{lemma-representation} below shows that every suitable pair of relations arises in
this way.

\begin{definition}\label{def-provle}
Let $\Gamma$ be any set in our logic.
Define relations $\provle$ and $\provsub$ on $\Pairs$ by:

\begin{itemize}%[parsep=0pt,partopsep=0pt]
    \item $(a,b) \provle (c,d)$ iff $\Gamma \vdash  \Atleast{(c \cup d){(a \cup b)}}$
    
    \item $(a,b) \provsub (c,d)$ iff $\Gamma \vdash \All{(a \cup b)}{(c \cup d)}$
\end{itemize}
for all $(a,b)$ and  $(c,d)$ in $\Pairs$.  \footnote{LM: Here's 
an explanation of \hl{my main notational choice, in case we want to 
discuss and/or change it}.  I resisted the temptation to 
import pairs into our original syntax.   So when it comes to 
sentences, I wrote $ \All{(a \cup b)}{(c \cup d)}$
and never ``$\All{x}{y}$ where $x$ and $y$ correspond to $p = (a,b)$
and $q=(c,d)$.''  I figure that this second usage would be too much for people.
Do you agree?
}
\end{definition}

Then $(\provle,\provsub)$ has all of the properties of a suitable pair,
except possibly for linearity of $\provle$.   To linearize it, we use the 
a general fact.  First, some background.
For any preorder relation $\leq$, we let $<$ be the \emph{strict part} of $\leq$:
  \[ x < y \quadiff x \leq y \mbox{ but } y \nleq x\]
 Recall that a \emph{total} (or \emph{linear}) preorder is one where every two points are comparable:
 $x\leq y$ or $y\leq x$ (and possibly both).
 
\begin{proposition}
\label{prop-extension-first}
Let $(P,\leq_0)$ be a finite preorder.  Then there is a linear preorder $\leq_1$ on $P$ 
such that for all $x,y\in P$:
(a) if $x \leq_0 y$, then also $x\leq_1 y$; and 
(b) if $x <_0 y$, then $x <_1 y$.
\rem{
Moreover, for fixed $x^*\in P$, we can also arrange that $\leq_1$ have the following 
additional property:
(c) For all $y$, if $y \nleq_0 x^*$, then $x^* <_1 y$.
%and
%(d) if $x^*\nless_0 y^*$, then $y^* \leq_1 x^*$.
}\end{proposition}

\begin{definition}
 A  \emph{$BN$-family} is  a family of  finite sets
$S = (S_a)_{a\in BN}$.
For a $BN$-family $S$, we write $S_{a,b}$ for $S_a \cup S_b$.  We also write $s_{a,b}$
for the number  $|S_a\cup S_b|$.  
We also write $s_a$ for $s_{a,a}$ (i.e., for $|S_a|$).

For $p\in\Pairs$, say with $p= (a,b)$, we often write $S_p$ instead of $S_{a,b}$.
\end{definition}

\begin{definition}
Let $p,q\in \Pairs$.
A $BN$-family $S$ is \emph{$\preceq$-preserving} if 
$p \preceq q$ implies that $s_{p} \le s_{q}$.  $S$ is \emph{$\preceq$-reflecting} if $s_{p} \le s_{q}$ implies that $p \preceq q$.

 $S$ is \emph{$\precsubseteq$-preserving}
if $p \precsubseteq q$ implies that $S_{p} \subseteq S_{q}$, and 
$S$ is \emph{$\precsubseteq$-reflecting} if 
$S_{p} \subseteq S_{q}$ implies that $p \precsubseteq  q$. 
\end{definition}

% Our hope is that $S$ preserves and reflects $\preceq$ and $\precsubseteq$, i.e. that we have a model whose terms preserve and relect the behavior of $\mbox{\sf All}$-, $\mbox{\sf More}$-, and $\mbox{\sf Atleast}$-sentences with term union.  
%We now state our representation lemma and use it to prove the completeness of  $\Aunion(\card)$.

\begin{lemma} [Representation Lemma]
Let $(\preceq, \precsubseteq)$ be a suitable pair of relations.
Then there is a $BN$-family of sets $S$
such that for all $p,q\in\Pairs$,
\begin{equation}
    \label{goal-main1-first}
p \preceq q \quadiff 
 s_{p}\leq s_q.
 \end{equation}
 And also
 \begin{equation}
    \label{goal-main2-first}
 p \precsubseteq  q \quadiff 
S_{p}\subseteq S_{q}.
 \end{equation}
 That is, $S$ preserves and reflects $\preceq$ and $\precsubseteq$.
 \label{lemma-representation-first}
 \end{lemma}
 
\begin{theorem}
    If $\Gamma$ is consistent, it has a model.
\end{theorem} 

\begin{proof}
Use Definition~\ref{def-provle}
to define $\provle$ and $\provsub$ from $\Gamma$.  Then use
Proposition~\ref{prop-extension-first} to extend $\provle$ to $\provextended$.

\begin{claim}
$(\provextended, \provsub)$ is an  \suitable{} pair of relations on $\Pairs$.
\end{claim}
To see this,
write $p$ as $(i,j)$ and $q$ as $(k,l)$.
    Suppose that $p \provsub q$ and $q \provextended p$.  
    So $\Gamma \vdash \All{(i \cup j)}{(k \cup l)}$.  
    By ($\proverule{subset-size}$), $\Gamma \vdash \Atleast{(k \cup l)}{(i \cup j)}$.  
    Thus  $(i,j)\provle (k,l)$.
   Now if $(i,j)\provlestrict (k,l)$, we would also have $(i,j)\provextendedstrict (k,l)$,
    since we used  Proposition~\ref{prop-extension} to obtain $\provextended$ (see part (b)).  
    This would contradict $(k, l) \provextended (i, j)$.  
     Thus $(k,l)\provle (i,j)$.  Using (\mix),  $q \provsub p$, as desired.
     
Now we use Lemma~\ref{lemma-representation-first}.
There is a $BN$-family of sets $(S_a)_{a\in BN}$ such that for all 
$(a,b),(c,d) \in \Pairs$:

\[
\begin{array}{lcccc}
\Gamma \proves \Atleast{(c \cup d)}{(a \cup b)} & \implies & 
    (a,b) \provextended (c,d) & \iff & 
    |S_a \cup S_b| \le |S_c \cup S_d|\\
\Gamma \proves \More{(c \cup d)}{(a \cup b)} & \implies & 
    (a,b) \provextendedstrict (c,d) & \iff & 
    |S_a \cup S_b| < |S_c \cup S_d|\\
\Gamma \proves \All{(a \cup b)}{(c \cup d)} & \iff & 
    (a,b) \provsub (c,d) & \iff & 
    S_a \cup S_b \subseteq S_c \cup S_d\\
& (\textrm{by construction}) & & (\textrm{by Lemma \ref{lemma-representation}}) & 
\end{array}
\]
Indeed, the $\implies$ arrows from the first column to the third column
hold for \emph{all} terms $a\cup b$ and $c\cup d$, even when $(a,b)$ or 
$(c,d)$ is not in $\Pairs$.  And versions of these hold as well for 
sentences using basic terms.


We build the model $\Model$ as follows.  Its universe $M$ is  $\bigcup_{a} S_a$.
This set is finite because $\Gamma$ is finite.
For a basic term  $a$, let $\semantics{a} = S_{a}$. 
So  for union terms $a\cup b$, we
automatically
have $\semantics{a\cup b} = S_{a} \cup S_{b} = S_{a,b}$.
Then by the implications outlined above, $\Model$ satisfies the $\AllNoArgs$-, $\MoreNoArgs$-, and $\AtleastNoArgs$-sentences in $\Gamma$. 
\end{proof}
 
 \subsection{Completeness}
 
 We show that if $\Gamma$ is a finite consistent set and $\Gamma\not\proves\phi$,
 then there is a model of $\Gamma$ where $\phi$ is false.
 
 \paragraph{Case 1: $\phi$ is $\All{x}{y}$}
 In this case, the work is already done.   Let $\Model$ be the model of $\Gamma$ from
 the previous section. Write $x$ as $a\cup b$ and $y$ as $c\cup d$.
 Since $\Gamma\not\proves\phi$, $S_a\cup S_b$ is not a subset of $S_c\cup S_d$.

 \paragraph{Case 2: $\phi$ is $\Atleast{x}{y}$} 
 
 \begin{proposition}
\label{prop-extension-second}
Let $(P,\leq_0)$ be a finite preorder. 
Let $x^*\in P$.  Then there is a linear preorder $\leq_1$ on $P$ 
such that for all $x,y\in P$:
(a) if $x \leq_0 y$, then also $x\leq_1 y$; and 
(b) if $x <_0 y$, then $x <_1 y$.
(c) For all $y$, if $y \nleq_0 x^*$, then $x^* <_1 y$.
%and
%(d) if $x^*\nless_0 y^*$, then $y^* \leq_1 x^*$.
\end{proposition}

 
  \paragraph{Case 3: $\phi$ is $\More{x}{y}$}
  
%%%%%%%%%%%%%%%%%%%%%%%%%%%%%%%%%%%%%%%%%%%%%%%%%%%%%%%%%%%%%%%%%%%%%%
\section{Completeness of $\Aunion(\card)$}\label{s:comp_Aunion(card)}
\label{section-completeness}
%%%%%%%%%%%%%%%%%%%%%%%%%%%%%%%%%%%%%%%%%%%%%%%%%%%%%%%%%%%%%%%%%%%%%%


We will now prove the completeness of $\Aunion(\card)$.  Our strategy is to show that if $\Gamma \not \proves \varphi$ then $\Gamma \not \models \varphi$.  We will build a model $\Model$ satisfying $\Gamma$ but not satisfying $\varphi$.  Now, the interaction between the $\MoreNoArgs$- and $\AtleastNoArgs$-sentences and term unions introduces combinatorial complications in building this model $\Model$.  We address the combinatorics head-on via a representation lemma that maps terms and their ordering as given by $\Gamma$ to pairs and certain preorders on pairs. 
%(Recall that a preorder is a set with a reflexive and transitive relation on it.\footnote{LM: I think that this point should be made someplace.})
In this section we will state the representation and use it to prove the completeness of $\Aunion(\card)$.  In the next section we will provide an outline of the proof of this representation lemma.

Before we start, 
we mention a general fact about topological sorts that we are going to use.  The proof of this fact is shown in Appendix B.
For any preorder relation $\leq$, we let $<$ be the \emph{strict part} of $\leq$:
  \[ x < y \quadiff x \leq y \mbox{ but } y \nleq x\]
 Recall that a \emph{total} (or \emph{linear}) preorder is one where every two points are comparable:
 $x\leq y$ or $y\leq x$ (and possibly both).
 
\begin{proposition}
\label{prop-extension}
Let $(P,\leq_0)$ be a finite preorder.  Then there is a linear preorder $\leq_1$ on $P$ 
such that for all $x,y\in P$:
(a) if $x \leq_0 y$, then also $x\leq_1 y$; and 
(b) if $x <_0 y$, then $x <_1 y$.

Moreover, for fixed $x^*\in P$, we can also arrange that $\leq_1$ have the following 
additional property:
(c) For all $y$, if $y \nleq_0 x^*$, then $x^* <_1 y$.
%and
%(d) if $x^*\nless_0 y^*$, then $y^* \leq_1 x^*$.
\end{proposition}

\rem{
\footnote{LM: I am not sure that we actually need (d) in Proposition~\ref{prop-extension}.
So \hl{maybe we can drop (d)} both from the statement and the proof.}
}

\rem{\begin{remark}\label{remark-identify}
\marginpar{LM: \textcolor{red}{look}.}
Up until now in this paper, we used letters $a$, $b$, $\ldots$
for basic nouns.  At this point, we need some additional notation.
Let $BN$ be the set of basic nouns.  We fix a linear order $<$ on $BN$.
alone, and we only return to variables in the proof of Theorem~\ref{theorem-completeness-Aunioncard}.
\end{remark}
}

Up until now in this paper, our basic terms were symbols like $a$, $b$, $\ldots$.  
Let $BN$ be the set of basic nouns.  We fix a linear order $<$ on $BN$.

\rem{
At this point, we need to work with an order on basic terms, say $a_1, \ldots, a_n$.  Going further, we will \emph{identify} each basic term $a_i$ with its subscript $i$.  We  represent
binary terms $a_i \cup a_j$ with the pair $(i, j)$.  Similarly, we will represent basic terms $a_i$ with the pair $(i, i)$.  We present our representation lemma exclusively in terms of pairs of numbers, and we only return to ordinary term presentation in the proof of Theorem~\ref{theorem-completeness-Aunioncard}.
}
\begin{definition}
Let $\Pairs = \set{(a,b)\in BN:  a \leq b}$.
 We use letters $p$ and $q$ to range over $\Pairs$ whenever possible.
 That is, we prefer to denote an element of $\Pairs$ with a single symbol,
 to simplify our notation.
\end{definition}

\begin{remark}\label{point-pairs}
The point of this set $\Pairs$ is that 
for every term $x$, there is some $(a,b)\in\Pairs$ so that $\proves \All{x}{(a\cup b)}$
and $\proves \All{(a\cup b)}{x}$.  Indeed, if $x$ is basic, we take $a = x = b$.
If $x$ is $c\cup d$, we let $a$ be the smaller of $c$ and $d$ in the fixed order $<$,
and we let $b$ be the larger.   For work in this section, it is best to think of
the terms in all sentences as being $a\cup b$ with $(a,b)\in\Pairs$.
\end{remark}

\rem{We think of $(a,b)$ as the union term $a\cup b$.
And so $(a,a)$ then corresponds to the basic term $a$.}

For a fixed $\Gamma$ in the completeness theorem,
we shall introduce orderings $\preceq, \precsubseteq$ on pairs such that $\preceq$ corresponds to the behavior of $\MoreNoArgs$-sentences and $\AtleastNoArgs$-sentences inferrable from $\Gamma$ and $\precsubseteq$ corresponds to $\AllNoArgs$-sentences inferrable from $\Gamma$.  We abstract the properties that we need such orderings to have in the following definition.

\rem{We abstract the properties 
of interpretations in models in the following definition.}

\begin{definition}
An \emph{\suitable}
pair of relations on $\Pairs$ 
is a pair of relations $(\preceq, \precsubseteq)$ such that for all
pairs $p$ and $q$, and all basic nouns $a$, $b$, $c$, and $d$:
\footnote{LM: we only have one kind of
suitable pair, so I suggest dropping the $\Aunion(\card)$.}

\begin{enumerate}
\item $\preceq$ and $\precsubseteq$ are preorders on $\Pairs$.
(That is, they are reflexive and transitive.)
\item   $\preceq$ is linear:
 either $p \preceq q$ or $q \preceq p$.
 \item If $a < b$, then $(a,a) \precsubseteq (a,b)$.  If $b < a$, then $(a,a) \precsubseteq (b,a)$. 
 \item If $(a,a) \precsubseteq (c,d)$ and $(b,b) \precsubseteq (c,d)$ and $a < b$,
 then $(a,b) \precsubseteq (c,d)$.
\item If $p \precsubseteq q$, then $p\preceq q$.
\item If $p \precsubseteq q$ and $q\preceq p$, then $q \precsubseteq p$.
\end{enumerate}
\label{def-suitable-pair}
\end{definition}


Here is an example:
Let $\Nodel$ is any model, and define $(a,b) \preceq (c,d)$ iff 
$|\semantics{a} \cup \semantics{b}| \leq
|\semantics{c} \cup \semantics{d}|$, and 
 $(a,b) \precsubseteq (c,d)$ iff 
$\semantics{a} \cup \semantics{b} \subseteq
\semantics{c} \cup \semantics{d}$.
This gives a suitable pair of relations.
Lemma~\ref{lemma-representation} below shows that every suitable pair of relations arises in
this way.


\begin{definition}
 A  \emph{$BN$-family} is  a family of  finite sets
$S = (S_a)_{a\in BN}$.
For a $BN$-family $S$, we write $S_{a,b}$ for $S_a \cup S_b$.  We also write $s_{a,b}$
for the number $|S_a\cup S_b|$. 
We also write $s_a$ for $s_{a,a}$ (i.e., for $|S_a|$).

For $p\in\Pairs$, say with $p= (a,b)$, we often write $S_p$ instead of $S_{a,b}$.
\end{definition}

\begin{definition}
Let $p,q\in \Pairs$.
A $BN$-family $S$ is \emph{$\preceq$-preserving} if 
$p \preceq q$ implies that $s_{p} \le s_{q}$.  $S$ is \emph{$\preceq$-reflecting} if $s_{p} \le s_{q}$ implies that $p \preceq q$.


 $S$ is \emph{$\precsubseteq$-preserving}
if $p \precsubseteq q$ implies that $S_{p} \subseteq S_{q}$, and 
$S$ is \emph{$\precsubseteq$-reflecting} if 
$S_{p} \subseteq S_{q}$ implies that $p \precsubseteq  q$. 
\end{definition}

% Our hope is that $S$ preserves and reflects $\preceq$ and $\precsubseteq$, i.e. that we have a model whose terms preserve and relect the behavior of $\mbox{\sf All}$-, $\mbox{\sf More}$-, and $\mbox{\sf Atleast}$-sentences with term union.  
%We now state our representation lemma and use it to prove the completeness of  $\Aunion(\card)$.

\begin{lemma} [Representation Lemma]
Let $(\preceq, \precsubseteq)$ be a suitable pair of relations.
Then there is a $BN$-family of sets $S$
such that for all $p,q\in\Pairs$,
\begin{equation}
    \label{goal-main1}
p \preceq q \quadiff 
 s_{p}\leq s_q.
 \end{equation}
 And also
 \begin{equation}
    \label{goal-main2}
 p \precsubseteq  q \quadiff 
S_{p}\subseteq S_{q}.
 \end{equation}
 That is, $S$ preserves and reflects $\preceq$ and $\precsubseteq$.
 \label{lemma-representation}
 \end{lemma}

\begin{theorem}
    The logic $\Aunion(\card)$ is complete.
    \label{theorem-completeness-Aunioncard}
\end{theorem}
\begin{proof}
If $\Gamma$ were inconsistent, then every sentence $\varphi$ would be provable, and 
we would be done.  So we may assume $\Gamma$ is a  finite consistent
 set of $\AllNoArgs$-, $\MoreNoArgs$-, and $\AtleastNoArgs$-sentences,  allowing binary union terms.
Suppose that $\Gamma \not \proves \varphi$.  Our plan is to use Lemma \ref{lemma-representation} to build a model $\Model$ 
satisfying $\Gamma$ and falsifying $\varphi$.

In this part of the paper, we are going to deal with the cases
that $\phi$ is $\All{x}{y}$ or $\Atleast{x}{y}$.   We defer to Appendix \hl{section}
the proof when  $\phi$ is $\More{x}{y}$, since that case requires additional work.
We shall assume that $x$ and $y$ are binary union terms $a\cup b$ with $(a,b)\in\Pairs$
(see Remark~\ref{point-pairs}).

Define relations $\provle$ and $\provsub$ on $\Pairs$ by:

\begin{itemize}%[parsep=0pt,partopsep=0pt]
    \item $(a,b) \provle (c,d)$ iff $\Gamma \vdash  \Atleast{(c \cup d){(a \cup b)}}$
    
    \item $(a,b) \provsub (c,d)$ iff $\Gamma \vdash \All{(a \cup b)}{(c \cup d)}$
\end{itemize}
for all $(a,b)$ and  $(c,d)$ in $\Pairs$.  

The relation $\provle$ is a preorder, due to our logic.
Recall that in a suitable pair, $\preceq$ must be a \emph{linear} preorder.
Note that $\provle$ is not a linear preorder on $\Pairs$,
so we must change this preorder.




Recall that we started with $\Gamma$ and $\phi$ such that 
$\Gamma \not \proves \varphi$. 
Suppose that $\phi$ is $\All{x}{y}$.
In this case, we simply use the first part of 
Proposition~\ref{prop-extension} with $\leq_0$ being $\provle$
to find a linear preorder $\provextended$.
 

Suppose that
 $\phi$ is $\Atleast{x}{y}$, again with $x$ the term $c\cup d$ and $y$ the term $a\cup b$.
Then we use (c) in 
Proposition~\ref{prop-extension}, and   we take $\leq_0$ to be $\provle$, and $x^*$ to be $(c,d)$.

\rem{
Suppose that $\phi$ is $\More{x}{y}$,
with $x$ being $c\cup d$ and $y$ being $a\cup b$.  We assume both $(a,b)$ and $(c,d)$ are in $\Pairs$.
We are assuming that $\Gamma\not\proves\phi$.   If $\Gamma\not\proves\Atleast{x}{y}$,
then we get $\leq_1$ from the previous paragraph.   If $\Gamma\proves\Atleast{x}{y}$,
we have two subcases.  If also $\Gamma\proves\Atleast{y}{x}$, we use Proposition~\ref{prop-extension}
without the ``Moreover'' part.   If $\Gamma\not\proves\Atleast{y}{x}$,
  use (c) in 
Proposition~\ref{prop-extension} with $\leq_0$ being $\provle$, $x^*$ being $(a,b)$,
and $y^*$ being $(c,d)$.
}

\bigskip

Now we have a linear preorder $\provextended$.
This preorder has the following properties, for $(a,b), (c,d)\in \Pairs$:
(1) if $\Gamma \vdash \Atleast{(a\cup b)}{(c\cup d)}$, 
    then $(c,d) \provextended (a,b)$;
(2) if $\Gamma \vdash \More{(a\cup b)}{(c\cup d)}$, then $(c,d) \provextendedstrict (a,b)$.
  

Point (1) just above is immediate.
Here is a verification of (2): Suppose that $\Gamma \vdash \More{(a\cup b)}{(c\cup d)}$.
We have  $\Gamma \vdash \Atleast{(a\cup b)}{(c\cup d)}$, by $\proverule{(more-atleast)}$.
Write $p$ for the pair $(a,b)$, and $q$ for $(c,d)$.
So $q \provle p$.
We cannot have $p \provle q$, since that would mean $\Gamma \vdash \Atleast{(c\cup d)}{(a\cup b)}$,
and  $\Gamma$ would be inconsistent.   So $q \provlestrict p$.
  Thus $q \provextended p$.  By (b) in Proposition~\ref{prop-extension}, we have
  $q \provextendedstrict p$.
We verify Lemma \ref{lemma-provpair}, i.e. that our pair of relations is $\Aunion(\card)$-suitable, in Appendix B.

\begin{lemma}
\label{lemma-provpair}
$(\provextended, \provsub)$ is an  \suitable{} pair of relations on $\Pairs$.
\end{lemma}



We may now apply Lemma \ref{lemma-representation}. 
(Please note that while we extended $\provle$ to $\provextended$, we did not modify $\provsub$.)
By the lemma, there is a $BN$-family of sets $(S_a)_{a\in BN}$ such that for all 
$(a,b),(c,d) \in \Pairs$:

\[
\begin{array}{lcccc}
\Gamma \proves \Atleast{(c \cup d)}{(a \cup b)} & \implies & 
    (a,b) \provextended (c,d) & \iff & 
    |S_a \cup S_b| \le |S_c \cup S_d|\\
\Gamma \proves \More{(c \cup d)}{(a \cup b)} & \implies & 
    (a,b) \provextendedstrict (c,d) & \iff & 
    |S_a \cup S_b| < |S_c \cup S_d|\\
\Gamma \proves \All{(a \cup b)}{(c \cup d)} & \iff & 
    (a,b) \provsub (c,d) & \iff & 
    S_a \cup S_b \subseteq S_c \cup S_d\\
& (\textrm{by construction}) & & (\textrm{by Lemma \ref{lemma-representation}}) & 
\end{array}
\]
Indeed, the $\implies$ arrows from the first column to the third column
hold for \emph{all} terms $a\cup b$ and $c\cup d$, even when $(a,b)$ or 
$(c,d)$ is not in $\Pairs$.  And versions of these hold as well for 
sentences using basic terms.


We build the model $\Model$ as follows.  Its universe $M$ is  $\bigcup_{a} S_a$.
This set is finite because $\Gamma$ is finite.
For a basic term  $a$, let $\semantics{a} = S_{a}$. 
So  for union terms $a\cup b$, we
automatically
have $\semantics{a\cup b} = S_{a} \cup S_{b} = S_{a,b}$.
Then by the implications outlined above, $\Model$ satisfies the $\AllNoArgs$-, $\MoreNoArgs$-, and $\AtleastNoArgs$-sentences in $\Gamma$. 

We also verify that $\Model$ does not satisfy $\varphi$.  We divide into cases, depending on the form of $\varphi$.
 Suppose that $\phi$ is $\All{(a\cup b)}{(c\cup d)}$. 
Recall that we have assumed that $(a,b)$ and $(c,d)$ belong to $\Pairs$.
 Suppose towards a contradiction that
 in $\Model$, $\semantics{a\cup b}\subseteq \semantics{c\cup d}$.
 Then $(a\cup b) \provsub (c\cup d)$.
 But then $\Gamma\proves\phi$, contradicting our assumption from above.

\rem{
Suppose that $\phi$ is $\More{(c \cup d)}{(a\cup b)}$, with the same notational assumptions as above.
Write $x$ for $c\cup d$ and $y$ for $a\cup b$.
We have several cases, in parallel with what we saw when
 we defined $\provextended$ from $\provle$.
 The most interesting is 
 when $\Gamma\proves \Atleast{x}{y}$
 and  $\Gamma\not\proves \Atleast{y}{x}$.
We took $x^*$ to be $a \cup b$
and $y^*$ to be $c\cup d$, thereby 
arranging that $(c,d) \provextended (a,b)$.
Thus $|\semantics{c\cup d}| \leq |\semantics{a\cup b}|$ in our model.
So $\phi$ is false in $\Model$.
}

Suppose that  
 $\phi$ is $\Atleast{(c \cup d)}{(a\cup b)}$.
Write $x$ for $(c,d)$ and $y$ for $(a,b)$.
We have $y \nprovle x$.
And since we used Proposition~\ref{prop-extension} part (c) with $x^* = x$,
we see that $x \provextendedstrict y$.
So in $\Model$, $|S_{c,d}| < |S_{a,b}|$.
That is, $\phi$ is false.

\rem{
\begin{enumerate}%[(a),parsep=0pt,partopsep=0pt]
    \item $\varphi$ is $\More{t}{u}$.
    \item 
    \item 
\end{enumerate}

For (a), since $\Gamma \not \proves \More{t}{u}$, $\Gamma \cup \set{\Atleast{u}{t}}$ is consistent.  So in the previous exposition, we may have instead built a model $\Model^*$ satisfying $\Gamma \cup \set{\Atleast{u}{t}}$.  This model $\Model^*$ does not satisfy $\More{t}{u}$, since $|\semantics{u}| \le |\semantics{t}|$.  Similarly for case (b), not satisfying $\Atleast{t}{u}$ amounts to building a model satisfying $\Gamma \cup \set{\More{u}{t}}$.

Now consider (c).
}

This completes the proof.
\end{proof}
}


%%%%%%%%%%%%%%%%%%%%%%%%%%%%%%%%%%%%%%%%%%%%%%%%%%%%%%%%%%%%%%%%%%%%%%
\section{Outline of the Proof of the Representation Lemma}\label{s:representation}
%%%%%%%%%%%%%%%%%%%%%%%%%%%%%%%%%%%%%%%%%%%%%%%%%%%%%%%%%%%%%%%%%%%%%%

For the proof of completeness of $\Aunion(\card)$, it remains to justify our representation lemma (Lemma \ref{lemma-representation}).  Given a \suitable{} pair of relations $(\preceq, \precsubseteq)$, we want to build a $BT$-family $S = (S_a)_{a\in BT}$ that preserves and reflects $\preceq$ and $\precsubseteq$.  Our plan is to start with a family that preserves and reflects $\precsubseteq$.  We then build our family of sets iteratively, ensuring that at each stage our family preserves and reflects $\precsubseteq$ and at the final stage our family also preserves and reflects $\preceq$.

\begin{proof}[Proof of Lemma \ref{lemma-representation}]

Consider the preorder $\preceq$.  We call a maximal set of $\preceq$-equivalent pairs a \emph{size class}.  We list the size classes in order, from  $\prec$-largest to $\prec$-smallest. 
Let's say the size classes in this order are
    \[  C_1, C_2, \ldots, C_K \]
Since we are listing them from 
$\prec$-largest to 
$\prec$-smallest, we have the following fact:
if $(a,b) \prec (c,d)$, and also  $(a,b)\in C_i$, and finally
$(c,d)\in C_j$,
then $j < i$.

We will inductively construct a sequence of $BT$-families $S^0, S^1, \ldots, S^K$ such that at each step $i$ we ensure that in $S^i$:

\begin{enumerate}[(1)]
    \item The pairs preceding or in the size class $C_i$ are correctly ordered according to $\preceq$, i.e. for all $(a,b)$, $(c,d)\in \bigcup_{j\leq i} C_i$,
 $
        (a,b) \preceq  (c,d)$ iff 
        $s^i_{a,b}\leq s^i_{c,d}
        $.
    
    \item The sizes of all pairs strictly preceding $C_i$ are larger than the sizes of all pairs in or following $C_i$.
    That is, for $j, k \in \{1, \ldots, K\}$ such that $j < i \le k$, $q \in C_j$, and $p \in C_k$, $s^i_q > s^i_p$.
    
    \item $S^i$ preserves and reflects $\precsubseteq$.
\end{enumerate}
If we do this for $i = 0, 1, \ldots, K$, then $S^K$ will
be a family of sets that preserves and reflects $\preceq$ and $\precsubseteq$.

\paragraph{Constructing $S^K$.} 
We begin by taking $S^0$ to be any family
which preserves and reflects $\precsubseteq$.
%The canonical choice is to take $S^0_i$ to be the join-prime up-closed subsets of $(X,\preceq)$ that contain $(i,i)$ as an element, where $X$ is the set of all unary and binary union terms.  
One choice is to take, for each basic term $a$, $S^0_a = \set{(c, d) \in \Pairs : (a, a) \not \precsubseteq (c, d)}$.  $S^0$ trivially satisfies assertions (1) and (2) from above and satisfies (3) by construction.

We now consider step $1 \le i \le K$.  Suppose we have a family $S^{i-1}$ that satisfies assertions (1)-(3) above for $i-1$.  In order to appropriately manipulate the sizes of sets $S^i_a$, we will apply the lemmas to follow.  Both lemmas make use of a basic construction on families of sets, which we call \emph{$\Clamp$}.

\begin{definition}  Let $S$ be a $BT$-family.  Let $(c, d) \in \Pairs, r \in \omega$. 
We define a new family $\Clamp(S, c, d, r)$ as follows: 
Let $R = \set{*_1, \ldots, *_r}$ be fresh points.  For all $a \in BT$, let
\[ \begin{array}{lcl}
\Clamp(S,c,d,r)_a = \left\{
\begin{array}{ll}
S_a \cup R & \mbox{if $ (a,a) \not \precsubseteq (c,d)$}\\
 S_a & \mbox{otherwise}\\ 
 \end{array}
 \right.
\end{array}
\]
In words, we add $r$ new points
simultaneously to all sets $S_a$, except when 
$\precsubseteq$ ``wants 
$S_a$ to be a subset of  $S_c\cup S_d$.''

\end{definition}

The following proposition summarizes the basic properties
of the $\Clamp$ operation.  %Arxiv:The proof of this fact is left to Section~\ref{s:supp:completeness-Aunioncard}.
\begin{proposition}
Let $S$ be a $BT$-family, and fix $(c, d) \in \Pairs$ and $r \in \omega$.  Write $T$ for $\Clamp(S, c, d, r)$.  Then:

\begin{enumerate}
    \item For $(a,b)\precsubseteq (c,d)$, $T_{a,b} = S_{a,b}$. \label{part-easy}

    \item For $(a,b)$ such that $(c,d) \prec (a,b)$, $T_{a,b} =  S_{a,b}\cup\set{*_1,\ldots, *_r}$. \label{part-bigger}

    \item If $S$ preserves and reflects $\precsubseteq$, then so does $T$. \label{part-preserve}
\end{enumerate}

\label{proposition-sClamp}

\end{proposition}

Our first lemma allows us to equalize sizes of unions of pairs in the same size class.  The second lemma ensures that the pairs in our size class $C_i$ have greater size than all pairs in 
$C_j$ for $j > i$.

\begin{lemma}

Let $S$ be a family which preserves and reflects $\precsubseteq$.  Let $k\geq 2$, and let $C = \set{p_1, \ldots, p_k}$ be a size class of  $\preceq$.  Then there is a $BT$-family $T$ such that: 

\begin{enumerate}
    \item The unions corresponding to the pairs in $C$ have equal size in $T$; i.e., for $1\leq r,s \leq k$, $t_{p_r} = t_{p_s}$.
    
    \item If $(a,b)$ and $(c,d)$ are any pairs 
    which belong to larger size classes than $C$,
    then 
    \[ \mbox{$t_{a,b} \leq t_{c,d} $ if and only if $s_{a,b} \leq s_{c,d} $}.\]
    
    \item $T$ preserves and reflects $\precsubseteq$. 
\label{equalize2}
\end{enumerate}

\label{lemma-equalizing}
\end{lemma}
\begin{proof}
Before we begin the construction of $T$,
we have an observation.
Let $\approx$ be the equivalence relation induced by $\precsubseteq$.
$C$, being a size class of $\approx$, splits into one or more
$\approx$-classes.
The observation is that if $q_1$ and $q_2$ are members of $C$ which are in different $\approxsubset$
classes, then neither $q_1 \precsubseteq q_2$ nor  $q_2 \precsubseteq q_1$.
To see this, suppose towards a contradiction that  $q_1 \precsubseteq q_2$.
Then since we also have $q_2 \preceq q_1$, we have 
$q_1 \precsubseteq q_2$ by one of the properties of the suitable pair $(\preceq, \precsubseteq)$.
This gives a contradiction, since now $q_1 \approx q_2$.


Let us choose one pair in each $\approx$-class of 
$C$, and list the chosen pairs in size order according to $S$.
That is, we have pairs $(a_1, b_1), \ldots, (a_k,b_k)$
so that every element of $C$ is related by 
$\approx$ to exactly one pair on this list,
and the order is chosen so that
$s_{a_1, b_1} \leq s_{a_2, b_2} \leq \cdots\leq s_{a_k, b_k}$.
 Let 
\[ \begin{array}{lcl}
 T^1  & = &  \Clamp(S,a_2,b_2,s_{a_2, b_2} - s_{a_1, b_1})\\
T^2 & = & \Clamp(T^1,a_3, b_3, s_{a_3, b_3} - s_{a_2, b_2} )\\
  & \vdots   & \\
T^{k-1} & = & \Clamp(T^{k-2},a_k,b_k,
s_{a_k, b_k} - s_{a_{k-1}, b_{k-1}})\\
\end{array}
\]
We take $T = T^{k-1}$.  One may carefully verify that this $T$ works.  %Arxiv:The verification that this $T$ works is left to Section~\ref{s:supp:completeness-Aunioncard}.
\end{proof}


\begin{lemma}

Let $S$ be a family which  preserves and reflects $\precsubseteq$.  
Let $q_1$, $\ldots$, $q_{\ell}$ be a sequence of pairs in $\Pairs$.  
Then there is a family $T$ such that:

\begin{enumerate} 
    \item For $1\leq i,j \leq k$, $s_{q_i} \leq s_{q_j}$ iff $t_{q_i} \leq t_{q_j}$. \label{competitor1}
    
    \item For all pairs $p $ which are $\prec$-below all $q_j$, we also have $t_{p}  <  t_{q_i}$ for all $i$. \label{competitor2}
  
    \item $T$ preserves and reflects $\precsubseteq$. 

\end{enumerate}

\label{lemma-sizeadjustment}
\end{lemma}
\begin{proof}
Let $m =  \min_i s_{i} = \min_i s_{q_i}$.
We call a pair $p$ a \emph{size competitor} if 
$p\prec q_j$ for all $j$, and yet  $t_p \geq m$.

List the size competitors as $p_1, \ldots, p_k$. 
 Note that for all size competitors $p_i$ and all of the original points
 $q_j$,  we have $q_j \not \precsubseteq p_i$.
For if we did have $q_j \precsubseteq p_i$, then we would have 
$q_j \preceq p_i$; and the definition of a size competitor
ensures that that $p_i \prec q_j$ for all $i, j$.


 Let \[ \begin{array}{lcl}
 T^1  & = &  \Clamp(S,p_1,  s_{p_1} -m + 1)\\

T^2 & = & \Clamp(T^1,p_2,  s_{p_2}-m + 1 )\\
  & \vdots   & \\
T^{k} & = & \Clamp(T^{k-1},p_k, s_{p_k}-m + 1 )\\
\end{array}
\]
We take $T = T^{k}$.  Again, one may carefully verify that this $T$ works.  %Arxiv:Again, we verify that this $T$ works in Section~\ref{s:supp:completeness-Aunioncard}.
\end{proof}


We may finally return to constructing the next family $S^i$.  Consider our currently examined size class $C_i$.  If $C_i$ contains $\ge 2$ pairs then we may apply Lemma \ref{lemma-equalizing} with size class $C_i$ and family $S^{i-1}$ in order to obtain a new family which we'll call $T$.  (If $C_i$ contains only $1$ pair, let $T = S^{i-1}$).  In $T$, all pairs in $C_i$ have the same size (by part 1 of Lemma \ref{lemma-equalizing}).  By (1) for $i-1$ and by part 2 of Lemma \ref{lemma-equalizing}, we have (1) for $T$.  Lemma \ref{lemma-equalizing} also ensures that $T$ preserves and reflects $\precsubseteq$.

We list the pairs preceding or in size class $C_i$ as $p_1, \ldots, p_k$.  We apply Lemma \ref{lemma-sizeadjustment} with these pairs $p_1, \ldots, p_k$ and family $T$.  We let $S^i$ be the resulting family.  Lemma \ref{lemma-sizeadjustment}, part 1 ensures that (1) holds for $S^i$, since it held for $T$.  And Lemma \ref{lemma-sizeadjustment}, part 2 ensures that (2) holds for $S^i$.  Lemma \ref{lemma-sizeadjustment} also ensures (3): that $S^i$ preserves and reflects $\precsubseteq$.

This completes the proof of our representation lemma and hence the completeness of $\Aunion(\card)$.
\end{proof}



%%%%%%%%%%%%%%%%%%%%%%%%%%%%%%%%%%%%%%%%%%%%%%%%%%%%%%%%%%%%%%%%%%%%%%
\section{Completeness of Logics with \\Intersection Terms}\label{s:completenes_intersections}
%%%%%%%%%%%%%%%%%%%%%%%%%%%%%%%%%%%%%%%%%%%%%%%%%%%%%%%%%%%%%%%%%%%%%%

We obtain completeness of $\Ainter(\card)$ for free from the completeness of  $\Aunion(\card)$.
Let $\lang^\cup$ be the language 
of $\Aunion(\card)$, and
%with binary union terms along with $\AllNoArgs$- and $\AtleastNoArgs$-sentences.
let $\lang^\cap$ be the language of $\Ainter(\card)$. 
%with binary intersection terms, and again with $\AllNoArgs$- and $\AtleastNoArgs$-sentences.
$\lang^\cup$ and $\lang^\cap$ share the same basic terms.


We  translate $\lang^\cap$ terms $x$ and sentences  $\phi$ to $\lang^\cup$ as follows:
 For basic terms $a$, let $a^\cup = a$. 
For an intersection term, say $a \cap b$, let $(a \cap b)^\cup = a \cup b$.  Let 
$\mathsf{R}$  be $\AllNoArgs$ or $\AtleastNoArgs$.
Then for the $\lang^\cap$-sentence
$\psi = \R{x}{y}$, we let     $\psi^\cup$ be  
 $\R{(y^\cup)}{(x^\cup)}$.  For a set $\Gamma$ of $\lang^\cap$ sentences,
 let $\Gamma^\cup = \set{\psi^\cup : \psi\in \Gamma}$.
 


\begin{lemma}
\label{lemma-proof-translation}
$\Gamma\proves\phi$ in $\Ainter(\card)$
iff $\Gamma^\cup\proves\phi^\cup$ in $\Aunion(\card)$.
\end{lemma}

\noproof{
\begin{proof}
Our mapping $\phi\mapsto\phi^\cup$ extends to instances of inference rules; instances of rules in $\Ainter(\card)$ are mapped to instances of the rules in $\Aunion(\card)$. 
In particular, instances of (\interl), (\interr), and (\interall) are sent to instances of (\unionl), (\unionr), and (\unionall), respectively.  For every other rule, an instance of the rule is sent to an instance of the same rule.  So any proof tree in $\Aunion(\card)$ witnessing $\Gamma \proves \varphi$ is sent to a proof tree in $\Aunion(\card)$ witnessing $\Gamma^\cup \proves \varphi^\cup$.  Showing that $\Gamma^\cup \proves \varphi^\cup \implies \Gamma \proves \varphi$ is similar.
\end{proof}
}

\rem{An instance of (\interl) 
gets sent to an instance of (\unionl), similarly for (\interr) and (\unionr), 
and an instance of (\interall) gets sent to an instance of (\unionall).  Instance
of every other rule get sent to an an instance of the same rule.  This means that every
proof tree for  $\Ainter(\card)$ showing that $\Gamma\proves\phi$ gets sent to a proof tree
in $\Aunion(\card)$ showing that $\Gamma^\cup\proves\phi^\cup$.   This shows half of our result,
and similar work the other way shows the other half.}


We also need a semantic construction in the other direction.
For a model $\Model$ of $\lang^\cup$, define a model 
 $\Model^\cap$ of $\lang^\cap$ as follows:
Let $\Model^\cap$ use the same underlying universe $M$. 
For basic terms $a$, let $\semantics{a}_{\Model^\cap} = \overline{\semantics{a}_{\Model}}$.  That is, each
basic term's semantics 
in $\Model^\cap$
is the complement of its semantics in $\Model$.



\begin{lemma}
For all  models $\Model$ of $\lang^\cup$ 
and all $\lang^\cap$-sentences $\psi$, 
$\Model^\cap \models \psi$ iff $\Model \models \psi^\cup$.
\label{proposition-union-inter-conversion}
\end{lemma}

\noproof{
\begin{proof}
Note that for all basic terms $a$ and $b$, 
$\semantics{a\cap b}_{\Model^\cap} =\semantics{a}_{\Model^\cap}
\cap \semantics{b}_{\Model^\cap}
=  \overline{\semantics{a}_{\Model}
\cup \semantics{b}_{\Model}  }
=  \overline{\semantics{(a\cap b)^{\cup}}_{\Model}  }
$.
Thus, for all $\lang^\cap$-terms $x$, 
$\semantics{x}_{\Model^\cap} = \overline{\semantics{x^\cup}}_{\Model}$.
Let $\psi$ be the $\lang^\cap$-sentence $\All{x}{y}$.
Then 
$\Model^\cap \models \psi$
iff
$\semantics{x}_{\Model^\cap} \subseteq \semantics{y}_{\Model^\cap}$
iff
$\overline{\semantics{y}_{\Model^\cap}} \subseteq \overline{\semantics{x}_{\Model^\cap}}$
iff
$\semantics{y^\cup}_\Model \subseteq \semantics{x^\cup}_\Model$
iff
$\Model\models\All{y^\cup}{x^\cup}$
iff
$\Model\models\psi^\cup$.
The same argument works for $\AtleastNoArgs$-sentences.
\end{proof}
}

From Lemmas~\ref{lemma-proof-translation} and \ref{proposition-union-inter-conversion}, completeness of $\Ainter(\card)$ follows.  %Arxiv:We prove this and the above lemmas in Section~\ref{s:supp:completeness-Aintercard}.

\begin{theorem}
    The logic $\Ainter(\card)$ is complete.
    \label{theorem-completeness-intersection}
\end{theorem}

\noproof{
\begin{proof} Suppose that $\Gamma\cup\set{\phi}$ are $\lang^\cap$
sentences, and $\Gamma \not \proves \varphi$ in $\Ainter(\card)$.
By Lemma~\ref{lemma-proof-translation},  $\Gamma^\cup\not\proves\phi^\cup$ in $\Aunion(\card)$.
By Theorem \ref{theorem-completeness-Aunioncard},
we have a model $\Model$ for $\lang^\cup$
such that $\Model \models \Gamma^\cup$ and $\Model \not \models \varphi^\cup$. 
Consider $\Model^\cap$ as defined  above.
By Proposition~\ref{proposition-union-inter-conversion},
$\Model^\cap\models\Gamma$
and $\Model^\cap\not\models\phi$.
\end{proof}
}

\rem{%Add to the Arxiv version:
We can use this same trick to show the completeness of the logic $\Ainter$, described in Remark~\ref{remark-related-logics}.  Unfortunately, Lemma \ref{proposition-union-inter-conversion} does not hold for $\SomeNoArgs$-sentences, so we cannot use this approach to show the completeness of $\Sinter$. (discussed in Section~\ref{s:supp:completeness_Aunion_Sunion}).  
Indeed, we have not proved that result.
}


%%%%%%%%%%%%%%%%%%%%%%%%%%%%%%%%%%%%%%%%%%%%%%%%%%%%%%%%%%%%%%%%%%%%%%
\section{Complexity of Our Logics}\label{s:complexity}
%%%%%%%%%%%%%%%%%%%%%%%%%%%%%%%%%%%%%%%%%%%%%%%%%%%%%%%%%%%%%%%%%%%%%%

We now turn our attention towards the complexity of $\Aunion(\card)$ and $\Ainter(\card)$.  As mentioned in Section~\ref{subsection-arbitraryterms}, one of our primary motivations for restricting our language to binary terms was to argue that our logics are decidable in polynomial time.
\noproof{
% NOTE: Include this sentence at the beginning of the section in Supplemental Material for these proofs.
The proofs of our claims in this section apply equally well to either of these logics, so for the purposes of this discussion we consider $\Aunion(\card)$.
}
The proof %Arxiv:(Section~\ref{s:supp:complexity-proofs})
is based on Theorem $1.5$ of \cite{exploring_the_landscape}, which is a variant of the proof of McAllester's Tractability Lemma \cite{recognition_of_tractability}.


\rem{
The sketch of the proof is this:  Because we can polynomially bound (relative to the combined length of $\Gamma$ and $\varphi$) the number of terms and the number of sentences in our language, we may polynomially bound the number of applications of our proof rules.  Hence we may polynomially bound the height of proof trees in our logics.  There are subtleties involved; the full proof is given in Section~\ref{s:supp:complexity-proofs}.
}
\noproof{
In this section, we need to be a bit more careful about which basic terms are used in proofs. So we fix a background language, built from a set of basic terms. Suppose $\mathcal{L}$ is a sublanguage, built from a subset of the basic terms, $\Gamma$ is a set of $\mathcal{L}$-sentences, and $\varphi$ is an $\mathcal{L}$-sentence. Then we write $\Gamma\vdash_{\mathcal{L}} \varphi$ if $\Gamma\vdash \varphi$ and this is witnessed by a proof tree using only $\mathcal{L}$-sentences. Similarly, we write $\Gamma\models_{\mathcal{L}} \varphi$ if every $\mathcal{L}$-model of $\Gamma$ satisfies $\varphi$. 
}

\noproof{
\begin{lemma}\label{lemma-language}
Let $\Gamma$ be a set of sentences, and let $\varphi$ be a sentence. Let $\mathcal{L}$ be the language containing only the basic terms appearing in $\Gamma \cup \{\varphi\}$. Then $\Gamma\vdash \varphi$ if and only if $\Gamma\vdash_{\mathcal{L}} \varphi$. 
\end{lemma}
}

\noproof{
\begin{proof}
One direction is clear. For the other, we assume $\Gamma\vdash \varphi$ and we want to show $\Gamma\vdash_{\mathcal{L}}\varphi$. By soundness in the full language and completeness in the restricted language $\mathcal{L}$, it suffices to show that $\Gamma\models \varphi$ implies  $\Gamma\models_{\mathcal{L}} \varphi$.

So assume $\Gamma\models \varphi$, and let $\Model$ be an $\mathcal{L}$-model of $\Gamma$. We extend $\Model$ to a structure $\Model'$ in the full language by assigning the basic terms which are not in $\mathcal{L}$ arbitrary interpretations. Then $\Model'\models \Gamma$, so $\Model'\models \varphi$, and $\Model\models \varphi$, since satisfaction of $\mathcal{L}$-sentences does not depend on the basic terms which are not in $\mathcal{L}$.  
\end{proof}
}

\begin{theorem}
\label{theorem-ptime}
The relation $\vdash$ is decidable in polynomial time. 
\end{theorem}
\noproof{
\begin{proof}
Let $\Gamma$ be a set of sentences, and let $\varphi$ be a sentence. Let $n$ be the combined length of $\Gamma$ and $\varphi$. Furthermore, we let $\mathcal{L}$ be the language with the set of basic terms restricted to those appearing in $\Gamma$ and $\varphi$. The number of terms and the number of sentences in $\mathcal{L}$ are each bounded by a polynomial in $n$, and by Lemma~\ref{lemma-language}, $\Gamma\vdash \varphi$ if and only if $\Gamma\vdash_{\mathcal{L}} \varphi$.

Now we have a finite set of rules, and a substitution instance of a rule is obtained by substituting at most three terms for term variables in the rule. So there is a polynomial $p(x)$ and a set $R$ of substitution instances of rules of size at most $p(n)$ such that if $\Gamma\vdash \varphi$, then there is a proof tree such that each leaf and node is labeled by an element of $R\cup \Gamma$. Further, we may assume that no element of $R$ appears twice along any path through the proof tree from the root to a leaf. Otherwise, we could shorten the path by replacing the subtrees above the premises of the lower instance of the rule by the subtrees above the premises of the higher instance of the rule. It follows from the pigeonhole principle that if $\Gamma\vdash \varphi$, then this is witnessed by a proof tree of height at most $p(n)$. 

We can now decide if $\Gamma\vdash \varphi$ as follows: Let $\Gamma_0 = \Gamma$. Given $\Gamma_i$, let $\Gamma_{i+1}$ be $\Gamma_i$ together with all sentences which can be deduced from premises in $\Gamma_i$ by a proof rule in $R$. Each set $\Gamma_i$ has size bounded by a polynomial in $n$ (since every element of $\Gamma_i$ is either in $\Gamma$ or is the conclusion of an element of $R$), and $\Gamma_{i+1}$ can be computed from $\Gamma_i$ in polynomial time. It follows that $\Gamma_{p(n)}$ can be computed from $\Gamma$ in polynomial time. By induction, $\Gamma_i$ is the set of all sentences $\psi$ in $\mathcal{L}$ such that $\Gamma\vdash_{\mathcal{L}} \psi$ by a proof tree of height at most $i$. Then $\Gamma\vdash \varphi$ if and only if $\varphi\in \Gamma_{p(n)}$, 
\end{proof}
}

Furthermore, for $\Aunion(\card)$ and $\Ainter(\card)$ the following also holds:


%%%%%%%%%%%%%%
\begin{theorem}
If $\Gamma \not \proves \varphi$ in either $\Aunion(\card)$ or $\Ainter(\card)$, then we can construct a countermodel $\Model$ satisfying $\Gamma$ but falsifying $\varphi$ in polynomial time.
\label{theorem-ptime-model-building}
\end{theorem}

\noproof{
\begin{proof}

Let $n$ be the combined length of $\Gamma$ and $\varphi$. For $\Ainter(\card)$, one may use the translation from intersection terms to union terms in order to build a countermodel of $\varphi$ from one for $\varphi$ in $\Ainter(\card)$, as is done in the proof of Theorem~\ref{theorem-completeness-intersection}.  It is easily seen that this translation can be done in polynomial time.

For $\Aunion(\card)$, we wish to show that the model $\Model$ used in the proof of Theorem~\ref{theorem-completeness-Aunioncard} can be constructed in polynomial time.  First, we may construct $\provle$ and $\provsub$ over $\Pairs$ in polynomial time, since deciding whether $\Gamma \proves \Atleast{(a \cup b)}{(c \cup d)}$ and whether $\Gamma \proves \All{(a \cup b)}{c \cup d)}$ can be done in polynomial time by Theorem~\ref{theorem-ptime}.  One may also check that extending $\provle$ to a linear preorder $\provextended$ can be done in polynomial time.
\hl{depends on the proof of Prop. 3.6}

It remains to ensure that the application of Lemma~\ref{lemma-representation} can be done in polynomial time, since no more work is needed to build the countermodel $\Model$.  Let $K$ be the number of size classes listed in the proof of Lemma~\ref{lemma-representation}.  Note that $K$ is bounded by a polynomial in $n$.  Our procedure for constructing $S^K$ involves $K$ steps; in each step, we apply Lemmas \ref{lemma-equalizing} and \ref{lemma-sizeadjustment} in sequence.  The former lemma involves selecting pairs from $\provsub$-equivalence classes, which can be done in polynomial time.  Both lemmas otherwise involve fewer than $K$ applications of the $\Clamp$ construction.  

For each application of $\Clamp$, we must first check whether $(a, a) \not \provsub (i, j)$, i.e. whether $\Gamma \not \proves \All{(a \cup a)}{(i \cup j)}.$   Again, this check can be done in polynomial time by Theorem~\ref{theorem-ptime}.
Finally, in each application of $\Clamp$, we must verify that the number of points added is bounded by $K$.  We proceed by induction on $i$, the index denoting the current stage $S^i$ of our family of sets.  We consider only those applications of $\Clamp$ in Lemma \ref{lemma-equalizing}, although the argument follows similarly for those in Lemma \ref{lemma-sizeadjustment}.

Consider the number of points added in a given instance of $\Clamp$ just prior to extending family $S^i$.  By inductive hypothesis, each instance of $\Clamp$ applied to obtain $S^i$ added a number of points bounded by a polynomial in $K$ to each set $S^i_a$.  Hence the \textit{total} number of points in each set $S^i_a$ is polynomial in $K$.  When extending $S^i$ to $T = S^{i+1}$, the number of points added in a given $\Clamp$ instance is $|S^i_{a_k} \cup S^i_{b_k}| - |S^i_{a_{k-1}} \cup S^i_{b_{k-1}}|$.  This is consequently bounded by a polynomial in $K$.
With this, we are done.
\end{proof}
}

The proof involves observing that the model-building procedure described in the proof of Theorem~\ref{theorem-completeness-Aunioncard} can be performed in polynomial time (relative to the combined length of $\Gamma$ and $\varphi$).  We may first construct $\provle$ and $\provsub$ over $\Pairs$ in polynomial time, since $\Aunion(\card)$ and $\Ainter(\card)$ are polynomial-time decidable (by Theorem~\ref{theorem-ptime}).  Of course, extending $\provle$ into the appropriate linear preorder $\provextended$ can be done in polynomial time.  The rest of the work is to carefully check that the algorithm described in Section~\ref{s:representation} can be done in polynomial time.  %Arxiv:The details are worked out in Section~\ref{s:supp:complexity-proofs}. 
%%%%%%%%%%%%%%%%%%%%%

\begin{remark}
It follows from Theorem~\ref{theorem-ptime} that our logics with arbitrarily large finite union (or intersection) terms are also decidable in polynomial time.  
Given $\Gamma, \varphi$ with arbitrary finite union terms (say), our decision procedure for $\Gamma \provesarbitrary \varphi$ is simply to construct $\Gamma^\star$ and $\varphi^\star$ and then decide whether $\Gamma^\star \proves \varphi^\star$ (see Section~\ref{subsection-arbitraryterms}).  
Constructing $\Gamma^\star$ and $\varphi^\star$ takes polynomial steps in the size of $\Gamma, \varphi$.  To verify that this is in fact a decision procedure for $\Gamma \provesarbitrary \varphi$, we must check that $\Gamma \provesarbitrary \varphi$ if and only if $\Gamma^\star \proves \varphi^\star$.  $\Gamma^\star \proves \varphi^\star \implies \Gamma \provesarbitrary \varphi$ is handled in Section \ref{subsection-arbitraryterms}.  As for the converse, a proof tree $\mathcal{T}$ for $\Gamma \provesarbitrary \varphi$ is transformed into a proof tree $\mathcal{T}^\star$ for $\Gamma^\star \proves \varphi^\star$ by introducing the term substitutions $t_i$.  One can verify that this is in fact a proof tree.  %Arxiv:Checking that this is in fact a proof tree is similar to the check in Section~\ref{s:supp:completeness-Aintercard}.

\label{remark-complexity}
\end{remark}

\rem{
\footnote{LM:
One thing that we should consider saying is that our logics have a property that's
 a little \hl{better than {\sc ptime}-decidability}.   There is a polynominal time algorithim
 which takes $\Gamma$ and $\phi$ and either returns a proof that $\Gamma \proves\phi$ in 
 one of our systems or else gives a countermodel.   For this, we'd need to check that the
 counter-model can be built in polynomial time.   I think we checked this once $\ldots$.
}}
% So if we can show that each set of rules for the respective logics ($\Aunion, \Sunion, \Aunion(\card), \Ainter, \Ainter(\card)$) is local, then it follows by completeness that whether $\Gamma \models \varphi$ is $\Ptime$-decidable in each logic!

% Suppose $\Gamma \proves_\Ruleset \varphi$.  We may assume $\Gamma$ is consistent, since if $\Gamma$ were inconsistent \hl{...}.  

% So if we can show that our proof relation $\proves$ for our syllogistic logics ($\Aunion, \Sunion, \Aunion(\card), \Ainter, \Ainter(\card)$) is boundedly complete, then it follows that whether $\Gamma \models \varphi$ is $\Ptime$-decidable in each logic!

% Suppose $\Gamma$ is consistent.  
% \hl{why can we make this assumption?\footnote{LM:
% \hl{Were $\Gamma$  inconsistent, it would prove anything, due to 
% the rule ({\sc x}).}}}  We claim that, in all of our logics, if $\Gamma \models \varphi$ then any proof tree $\mathcal{T}$ witnessing $\Gamma \proves \varphi$ only involves basic terms in $\Gamma$ and $\varphi$ and binary unions or intersections of these terms. 

%Since $\Gamma$ is consistent, rule $(\proverule{x})$ is not needed for any such proof tree $\mathcal{T}$, so we may assume it never occurs in $\mathcal{T}$.
% One may check the table of rules for our logics (Figure \ref{fig-rules}) that (with the exception of rule $(\proverule{x})$), a basic term occurs in a rule's conclusion if and only if that basic term occurs in that rule's premises.  It follows that any proof tree $\mathcal{T}$ with premises in $\Gamma$ and conclusion $\varphi$ must only involve basic terms in $\Gamma$ and $\varphi$.  Regarding union and intersection terms, this of course means that all union or intersection terms involved in $\mathcal{T}$ will also only involve these basic terms.

% So we wish to take:

% \[ \begin{array}{lcl}
% f(S) & = & \left\{ \R{x}{y} \quad \middle| \quad 
% \begin{array}{ll}
% \R \in \set{\AllNoArgs, \SomeNoArgs, \MoreNoArgs, \AtleastNoArgs}, \textrm{and} \\
% x, y \textrm{ are either basic terms in } S \textrm{ or binary unions of such terms.} \\
%  \end{array}
%  \right\}.
% \end{array}
% \]

% One can show that $f$ is $\Ptime$-computable, and as justified above that if $\Gamma \models \varphi$, then $\Gamma \proves \varphi$ via a proof tree $\mathcal{T}$ such that all sentences in $\mathcal{T}$ belong to $f(\Gamma \cup \set{\varphi})$.  Thus our syllogistic proof relation $\proves$ is boundedly complete, and we are done.

% Rules for \Munion(\card)
\begin{figure*}[ht!]
\begin{equation*}
\boxed{
\small
\begin{array}{c}
\begin{array}{ccc}
\\ 
\infer[(\morel)]
    {\More{x}{z}}
    {\More{x}{y} & \Atleast{y}{z}} 
&
\infer[(\morer)]
    {\More{x}{z}}
    {\Atleast{x}{y} & \More{y}{z}}
\end{array}
\\% \\
\begin{array}{ccc}
% TOGGLE FOR SPACING:
%\\ 
\infer[(\moreatleast)]
    {\Atleast{x}{y}}
    {\More{x}{y}}
&
\infer[(\x)]
    {\varphi}
    {\More{z}{z}}
&
\infer[(\raa)]
     {\More{x}{y}}
     {\inferLineSkip=0pt
     \infer*
        {\More{z}{z}}
        {[\Atleast{y}{x}]}}
\end{array}
\\
\end{array}
}
\end{equation*}
\caption{The additional rules for the logics $\Munion(\card)$ and $\Minter(\card)$.
\label{fig-more-rules}}
\end{figure*}

%%%%%%%%%%%%%%%%%%%%%%%%%%%%%%%%%%%%%%%%%%%%%%%%%%%%%%%%%%
\section{Completeness of Logics with \\$\MoreNoArgs$-Sentences}\label{s:completeness-more}
%%%%%%%%%%%%%%%%%%%%%%%%%%%%%%%%%%%%%%%%%%%%%%%%%%%%%%%%%%

In this section, we consider the extension of our logic $\Aunion(\card)$ with $\MoreNoArgs$-sentences.  We call the resulting logic $\Munion(\card)$ (similarly, we call the corresponding logic with $\cap$-terms $\Minter(\card)$).  We may extend the argument in Section \ref{s:comp_Aunion(card)} to prove the completeness of $\Munion(\card)$ (and hence, by Section \ref{s:completenes_intersections}, the completeness of $\Minter(\card)$).  Unfortunately, we cannot extend the complexity argument of Section \ref{s:complexity} to obtain polynomial decidability for $\Munion(\card)$.  We discuss avenues for ameliorating this situation in Section \ref{s:future_work}.

\noproof{\subsection{The Logics $\Munion(\card)$ and $\Minter(\card)$}}

We extend $\Aunion(\card)$ to $\Munion(\card)$ by allowing, in addition to $\AllNoArgs$- and $\AtleastNoArgs$-sentences, the sentence $\More{x}{y}$ (where $x$ and $y$ are terms).  The semantics of $\MoreNoArgs$-sentences is similar to that for $\AtleastNoArgs$-sentences:

\[
\begin{array}{lclcl}
    \Model \models \More{x}{y} & \textrm{ iff } & 
        |\semantics{x}| > |\semantics{y}|
\end{array}
\] 



$\Munion(\card)$ employs the rules listed in Figure \ref{fig-more-rules}, in addition to those rules used in $\Aunion(\card)$ (similarly for $\Minter(\card)$).  The rules in Figure \ref{fig-more-rules}, with the exception of (\raa), we borrow from the logic $\mathscr{S}(\card)$ described in \cite{syllogistic_cardinality_comparisons}.  Again, one may verify that each of the rules in Figure \ref{fig-more-rules} is individually sound for our semantics.

Note in particular the rules (\x) and (\raa).  The (\x) rule is \emph{ex falso} (or \emph{explosion}), fitted for our language of $\MoreNoArgs$- and $\AtleastNoArgs$-sentences.  Similarly, (\raa) is a special instance of \emph{reductio ad absurdum}.

\noproof{
% NOTE: Include this in the Supplemental Material section where we prove the completeness of \Munion(\card).
With $\MoreNoArgs$-sentences in our language, we must now worry about expressing inconsistencies within our logics.  We say that a set $\Gamma$ of sentences in $\Munion(\card)$ (or $\Minter(\card)$) is \emph{inconsistent} whenever every sentence $\varphi$ in the logic is provable from $\Gamma$ (otherwise, we say that $\Gamma$ is \emph{consistent}).  Note that for these particular logics, using rule (\x), $\Gamma$ is inconsistent if and only if there is a term $z$ such that $\Gamma \proves \More{x}{y}$ and $\Gamma \proves \Atleast{y}{x}$.
}

\noproof{\subsection{Completeness}}

In contrast to $\Aunion(\card)$ and $\Ainter(\card)$, we must worry about expressing inconsistencies within $\Munion(\card)$ and $\Minter(\card)$.  We say that a set $\Gamma$ of sentences in $\Munion(\card)$ (or $\Minter(\card)$) is \emph{inconsistent} whenever every sentence $\varphi$ in the logic is provable from $\Gamma$ (otherwise, we say that $\Gamma$ is \emph{consistent}).  Note that for $\Munion(\card)$, using rule (\x), $\Gamma$ is inconsistent if and only if there is a term $z$ such that $\Gamma \proves \More{z}{z}$.

\begin{theorem}
The logic $\Munion(\card)$ is complete.
\label{theorem-completeness-Munioncard}
\end{theorem}

The proof is a straightforward extension of the proof of Theorem~\ref{theorem-completeness-Aunioncard}; observe that in order to build a model of $\Gamma$ refuting $\More{x}{y}$, it suffices to construct a model of $\Gamma \cup \set{\Atleast{y}{x}}$. If $\Gamma\not\vdash \More{x}{y}$, then (\raa) ensures that $\Gamma \cup \set{\Atleast{y}{x}}$ is consistent, which is necessary since we can only apply the modified version of the model construction to consistent $\Gamma$. %Arxiv:The full proof of completeness is given in Section~\ref{s:supp:completeness-Munioncard}.



\noproof{
\begin{proof}
Suppose that $\Gamma$ is a finite, consistent set of sentences in $\Munion(\card)$, and suppose that $\Gamma \not \proves \varphi$.  Our plan is again to build a model of $\Gamma$ where $\varphi$ is false.
When $\varphi$ is an $\AllNoArgs$- or $\AtleastNoArgs$-sentence, we build our model as in the proof of Theorem \ref{theorem-completeness-Aunioncard}.  We deal here with the case where $\varphi$ is $\More{x}{y}$

Since $\Gamma \not \proves \More{x}{y}$, we cannot have a proof of $\More{x}{y}$ via (\raa) in particular.  That is, $\Gamma \cup \set{\Atleast{y}{x}} \not \proves \More{z}{z}$.  This means that $\Gamma \cup \set{\Atleast{y}{x}}$ is consistent.  We now only need to construct a model $\Model$ of $\Gamma \cup \set{\Atleast{y}{x}}$; such a model is a model of $\Gamma$, and in addition satisfies $\Model \not \models \More{x}{y}$, since $|\semantics{y}| \ge |\semantics{x}|$.

For what follows, let $\Gamma^\star = \Gamma \cup \set{\Atleast{y}{x}}$.  In order to construct the model $\Model$ of $\Gamma^\star$, we first obtain the suitable pair $(\provextendedstar, \provsubstar)$ as before.  We obtain a $BT$-family of sets $(S_{a})_{a \in BT}$ such that, in addition to the implications in (\ref{arrows}) (with $\Gamma^\star$ in place of $\Gamma$), for all $(a, b), (c, d) \in \Pairs$ we have:

\begin{equation}
\label{arrows-more}
\begin{array}{lcccc}
\Gamma^\star \proves \More{(a \cup b)}{(c \cup d)} & \implies & 
    (c,d) \provextendedstrictstar (a,b) & \iff & 
    S_c \cup S_d < S_a \cup S_b\\
& (\textrm{see below}) & & (\textrm{by Lemma \ref{lemma-representation}}) & 
\end{array}
\end{equation}

For the $\implies$ on the left, suppose that $\Gamma^\star \proves \More{(a \cup b)}{(c \cup d)}$.  We have $\Gamma^\star \proves \Atleast{(a \cup b)}{(c \cup d)}$ by (\moreatleast).  Write $p$ for the pair $(a, b)$ and $q$ for $(c, d)$.  So we have $q \provlestar p$.  We cannot have $p \provlestar q$, since that would mean $\Gamma^\star \proves \Atleast{(c \cup d)}{(a \cup b)}$, and $\Gamma$ would be inconsistent.  So $q \provlestrictstar p$.  By the definition of ``linearization'', we have $q \provextendedstrictstar p$.

We build $\Model$ from our family $S$ exactly as in the proof of Theorem \ref{theorem-completeness-Aunioncard}:  For every basic term $a$, let $\semantics{a} = S_a$.  By the implications in (\ref{arrows}), with $\Gamma$ replaced with $\Gamma^\star$, $\Model$ satisfies the $\AllNoArgs$- and $\AtleastNoArgs$-sentences in $\Gamma^\star$.  Additionally, by (\ref{arrows-more}), $\Model$ staisfies the $\MoreNoArgs$-sentences in $\Gamma^\star$.  So $\Model \models \Gamma^\star$, and we are done.

\end{proof}
}

%%%%%%%%%%%%%%%%%%%%%%%%%%%%%%%%%%%%%%%%%%%%%%%%%%%%%%%%%%%%%%%%%%%%%%
\section{Discussion and Future Work}\label{s:future_work}
%%%%%%%%%%%%%%%%%%%%%%%%%%%%%%%%%%%%%%%%%%%%%%%%%%%%%%%%%%%%%%%%%%%%%%

This paper has presented two complete, polynomial-time decidable logics for reasoning about the sizes of sets alongside the union or intersection of terms, respectively.  These logics are the most basic for reasoning of this kind; $\Aunion(\card)$ and $\Ainter(\card)$ are both minimally expressive and decidable in polynomial time.  Our logics may be viewed as more efficient fragments of $\BAPA$ and $\CardCompLogic$, two more expressive $\NP$-complete logics for reasoning about sizes with union and intersection.  A direct corollary of our work is the completeness of the logic additionally permitting $\MoreNoArgs$-sentences.

\begin{nextsteps*}
Since decidability in both $\BAPA$ and $\CardCompLogic$ is $\NP$-complete, it would be interesting to steadily build our fragment towards these logics and note at which point decidability is no longer decidable in polynomial time.  The first step in this direction is to attempt to extend our polynomial decidability argument in Section \ref{s:complexity} to the logic $\Munion(\card)$.  The main issue here is that $\Munion(\card)$ makes use of the (\raa) rule, and we cannot put a bound on the height of an (\raa) application.  One could remove the (\raa) rule from $\Munion(\card)$ and attempt to replace it with simpler rules that may also ensure completeness.  One such rule, not derivable from the rules of $\Munion(\card)$ sans (\raa) is:
\[
\infer[(\proverule{diamond})]
    {\More{a}{x}}
    {\All{x}{a} & \All{x}{b} & \More{(a \cup b)}{b}}
\]
(This rule is so-named because the terms $x$, $a$, $b$, and $a \cup b$ form a $\subseteq$-diamond.)
\noproof{To see that this rule of inference is sound, suppose $\Gamma \models \All{x}{a}$, $\Gamma \models \All{x}{b}$, and $\Gamma \models \More{(a \cup b)}{b}$.  Since $\Gamma \models \More{(a \cup b)}{b}$, there must be an element of $\semantics{a}$ not in $\semantics{b}$.  Since $\semantics{x} \subseteq \semantics{b}$, this element is not in $\semantics{x}$ either.  So $\semantics{x}$ must be a proper subset of $\semantics{a}$.  But then $\Gamma \models \More{a}{x}$.
}
Observe that this rule of inference is sound.  If (\proverule{diamond}) and rules like it ensure completeness without (\raa), then our argument for polynomial decidability follows without issue.

The next step would be to integrate union and intersection terms. %But it is very likely that 
But the further step of integrating union and intersection with term complement will likely result in an $\NP$-complete logic, so this is where the road ends.

% However, this is where the road ends:  It is very likely that the natural step of integrating union with term complement will result in an $\NP$-complete logic.  

There is historical precedent in the syllogistic logic
literature to allow $\SomeNoArgs$ sentences alongside
$\AllNoArgs$-sentences.   As mentioned in Remark \ref{remark-related-logics}, the usual semantics for $\SomeNoArgs$-sentences is that $\Model \models \Some{x}{y}$ whenever $\semantics{x} \cap \semantics{y} \ne \emptyset$.
The main trouble with introducing $\SomeNoArgs$-sentences is in addressing the following pesky rule:
\[
%\infer[(\proverule{pesky})]
\infer[]
    {\Some{a}{c}}
    {\More{a}{b} & \Atleast{c}{d} & \Atleast{(b \cup d)}{(a \cup c)}}
\]
This rule simultaneously involves $\AtleastNoArgs$-, $\MoreNoArgs$-, and $\SomeNoArgs$-sentences with term union.
Observe that this rule is sound as well, and that it is not provable from the rules of either $\Aunion(\card)$ or $\Sunion$.
One could hope to extend our model construction to model sets $\Gamma$ which also include $\SomeNoArgs$-sentences, but it is far from obvious how to integrate the above rule into the model-building process.

\noproof{
To see that this is a sound rule of inference, suppose that $\Gamma \not \models \Some{a}{c}$.  So
\[
\begin{array}{lclr}
|\semantics{a \cup c}| & = & |\semantics{a}| + |\semantics{c}| & (\textrm{since } \semantics{a} \cap \semantics{c} = \emptyset) \\

& > & |\semantics{b}| + |\semantics{d}| &
(\textrm{since } \Gamma \models \More{a}{b}, \Gamma \models \Atleast{c}{d}) \\

& \ge & |\semantics{b \cup d}| & 
\end{array}
\]
but $\Gamma \models \Atleast{(b \cup d)}{(a \cup c)}$, so $|\semantics{b \cup d}| \ge |a \cup c|$.  This is a contradiction.  So $\Gamma \models \Some{a}{c}$.  This rule is also not provable from those of either $\Aunion(\card)$ or $\Sunion$.
}
\noproof{
Note that a special case of this rule is
\[
\infer[(\proverule{more-some})]
    {\Some{a}{a}}
    {\More{a}{b}}
\]
which is a key rule relating $\SomeNoArgs$- and $\MoreNoArgs$-sentences in \cite{syllogistic_cardinality_comparisons}.
}

%One could hope to extend our model construction (given by our representation lemma) to model sets $\Gamma$ which also include $\SomeNoArgs$-sentences, but it is far from obvious how to do this.  Any such approach must account for the above rule, but new ideas are needed to integrate this rule into the model-building process.

Finally, our two main logics can be integrated with SMT solvers in order to efficiently automate those inferences which just involve sizes and subset alongside union or intersection.  A particularly appropriate SMT solver with which to test this is \cite{cardinality_constraints_smt}, which extends the SMT solver Z$3$ with reasoning in quantifier-free $\BAPA$.

% The most immediate next step would be, of course, combining the related logics $\Sunion$ and $\Sinter$ with the main logics presented in this paper.  To this end, one could try to obtain a complete axiomatization for $\Sunion(\card)$, the logic $\Aunion(\card)$ augmented with $\SomeNoArgs$-sentences.  Similarly, one could wish to do the same for $\Sinter(\card)$, the same logic but instead involving intersection terms.  We have entertained this thought; the main trouble is in addressing the following pesky rule (that simultaneously involves $\AtleastNoArgs$-, $\MoreNoArgs$-, and $\SomeNoArgs$- sentences with term union):
% \[
% \infer[(\proverule{pesky})]
%     {\Some{a}{c}}
%     {\More{a}{b} & \Atleast{c}{d} & \Atleast{(b \cup d)}{(a \cup c)}}
% \]

% To see that this is a sound rule of inference in $\Sunion(\card)$, suppose that $\Gamma \not \models \Some{w}{y}$.  So
% \[
% \begin{array}{lclr}
% |\semantics{a \cup c}| & = &
% |\semantics{a}| + |\semantics{c}| & 
% (\textrm{since } \semantics{a} \cap \semantics{c} = \emptyset) \\

% & > & |\semantics{b}| + |\semantics{d}| &
% (\textrm{since } \Gamma \models \More{a}{b}, \Gamma \models \Atleast{c}{d}) \\

% & \ge & |\semantics{b \cup d}| & 
% \end{array}
% \]
% but $\Gamma \models \Atleast{(b \cup d)}{(a \cup c)}$, so $|\semantics{b \cup d}| \ge |a \cup c|$.  This is a contradiction.  So $\Gamma \models \Some{a}{c}$.  This rule is also not provable from those of either $\Aunion(\card)$ or $\Sunion$.
% Note that a special case of this rule is
% \[
% \infer[(\proverule{more-some})]
%     {\Some{a}{a}}
%     {\More{a}{b}}
% \]
% which is a key rule relating $\SomeNoArgs$- and $\MoreNoArgs$-sentences in \cite{syllogistic_cardinality_comparisons}.

% \rem{
% \footnote{LM: I like the ({\sc pesky}) rule.
% And, after looking at it, I am reminded that the first connection rule between
% $\SomeNoArgs$ and the size quantifiers is that $\Some{x}{x}$ follows from $\More{x}{y}$.
% Also, we could get a complete logic for $\AtleastNoArgs + \MoreNoArgs + \SomeNoArgs$ using
% all our rules that just have those symbols, together with the rule I just mentioned
% (that $\Some{x}{x}$ follows from $\More{x}{y}$), and the rule that
% if $\Some{x}{x}$ and $\Atleast{y}{x}$, then $\Some{y}{y}$.
% }
% }

% One could hope to simply extend our model construction (given by our representation lemma)
% for $\Aunion(\card)$, but it is far from obvious how to do this.  Any such approach must account for the $(\proverule{pesky})$ rule, but new ideas are needed to integrate this rule into the model-building process.

% Since decidability in both $\QFBAPA$ and $\CardCompLogic$ is $\NP$-complete, it would be interesting to steadily build our fragment towards these logics and note at which point decidability is no longer $\Ptime$.  One could start by integrating both union and intersection terms.  This logic would be strictly weaker than the logic that also permits term complement, which would be the next candidate to try.  Term complement alone in syllogistic logics has been studied by \cite{syllogistic_cardinality_comparisons}.  The author gives a complete axiomatization of the $\Ptime$-decidable logic $\mathscr{S}^{\dagger}(\card)$, which is the logic $\Sunion(\card)$ with complement terms rather than union terms.  One path towards $\QFBAPA$ and $\CardCompLogic$ (restricted to finite interpretations of terms) is to try merging the logics $\Sunion(\card)$, $\Sinter(\card)$, and $\mathscr{S}^{\dagger}(\card)$.  Interestingly, such a logic would also express, say:
% \[
% \Most{a}{b} \textrm{ is } \More{(a \cap b)}{(a \cap \overline b)}
% \]
% and hence would encompass certain $\MostNoArgs$-logics of interest to natural logicians \cite{syllogistic_logic_most}.
\end{nextsteps*}

\paragraph{Acknowledgements} We thank the anonymous reviewers, as well as Vikraman Choudhury, Matthew Heimerdinger, and Chaitanya Koparkar, for their careful reviews and helpful comments.
%We thank the anonymous reviewers, as well as Vikraman Choudhury, Matthew Heimerdinger, and Chaitanya Koparkar, for their careful reviews and helpful comments.

%We thank the people who provided feedback to be named in the final form of the paper.
%Vikraman Choudhury, Matthew Heimerdinger, Chaitanya Koparkar,
%\emph{and others in alphabetical order} for useful comments.


%%%%%%%%%%%%%%%%%%%%%%%%%%%%%%%%%%%%%%%%%%%%%%%%%%%%%%%%%%%%%%%%%%%%%%
%\section{References}
%%%%%%%%%%%%%%%%%%%%%%%%%%%%%%%%%%%%%%%%%%%%%%%%%%%%%%%%%%%%%%%%%%%%%%

\bibliographystyle{aaai}
\bibliography{AAAI-KisbyC.6200}
%\bibliography{easychair}


\end{document}